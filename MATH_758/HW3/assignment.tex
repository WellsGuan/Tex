%!TEX program = xelatex
\documentclass[lang=en,11pt,a4paper,citestyle =authoryear]{elegantpaper}

% 标题
\title{Homework03 - MATH 758}
\author{Boren(Wells) Guan}

% 本文档命令
\usepackage{array,url,stix,enumerate}
\usepackage{subfigure}
\newcommand{\ccr}[1]{\makecell{{\color{#1}\rule{1cm}{1cm}}}}
\newcommand{\code}[1]{\lstinline{#1}}
\newcommand{\prvd}{$\hfill \qedsymbol$}
\newcommand{\Z}{\mathbb{Z}}
\newcommand{\R}{\mathbb{R}}
\newcommand{\N}{\mathbb{N}}
\newcommand{\C}{\mathbb{C}}
\newcommand{\Q}{\mathbb{Q}}
\newcommand{\M}{\mathcal{M}}
\newcommand{\B}{\mathcal{B}}
\newcommand{\X}{\mathcal{X}}
\newcommand{\Hil}{\mathcal{H}}
\newcommand{\range}{\mathcal{R}}
\newcommand{\nul}{\mathcal{N}}

% 文档区
\begin{document}

% 标题
\maketitle

\subsection*{Before Reading:}\par
To make the proof more readable, I will miss or gap some natural or not important facts or notations during my writing. If you feel it hard to see, you can refer the appendix after the proof, where I will try to explain some simple conclusions (will be marked) more clearly. In case that you misunderstand the mark, I will add the mark just after those formulas between \$ and before those between \$\$.\par
And I have to claim that the appendix is of course a part of my assignment, so the reference of it is required. Enjoy your grading!

\subsection*{Problem.3 Worksheet 5} 
Suppose that $(\mu_n)$ are Borel probability measures on a compact metric space $X$ that weak* converge to $\mu$. Show that for any compact subset $K$m $\limsup \mu_n(K) \leq \mu(K)$.
\vspace{0.5em}\\
\textbf{Sol.} \par
(Here we consider $\mu$ should be a Borel measure since it is consider as a bounded linear map of $C(X)$), we know there is $f_n$ in $C(X)$ such that $f_n\geq \chi_K$ and $\mu(K) = \lim \int f_n d\mu$, and hence
\[
\limsup \mu_n(K) \leq \limsup \int f_k \mu_n = \int f_k \mu \to \mu(K)
\]
and hence the inequality holds.\par
Then consider any sector centered at the origin with angle of $\pi$, then we know all $\mu_n = \dfrac{1}{4}$ for any integer $n$ from the last problem and hence the equality may not hold.
\prvd
\vspace{0.5em}

\subsection*{Problem.4 Worksheet 5} 
Under the same assumptions as the previous problem, show that for any open set $U$ with $\mu$-measure zero boundary $\lim_n \mu_n(U) = \mu(U)$. Finally, show that if $(X,T,\mu)$ is an ergodic topological p.m.p.s. and $x\in X$ is generic, the nfor any open set $U$ with measure zero boundary
\[\lim_{n\to\infty} \dfrac{1}{n}\sum\limits_{i=1}^n \mu(T^{-k}(U)) = \mu(U)\]
\vspace{0.5em}\\
\textbf{Sol.} \par
    Denote $K$ the closure of $U$ which is compact, then we know
    \[\limsup \mu_n(K) \leq \mu(K)\]
    and there are also $g_n$ in $CX$ such that $g_n \leq \chi_U$ and $\lim g_n \mu \to \mu(U)$, then we know 
    \[\liminf \mu_n(U) \geq \liminf \int g_k \mu_n = \int g_k\mu \to \mu(U)\]
    and hence
    \[\limsup \mu_n(K) \leq \mu(K) = \mu(U) \leq \liminf \mu_n(U)\]
    which means $\limsup \mu_n(K) \geq \liminf \mu_n(U)$ since $K\supset U$ and the limite exists and equals to $\mu(U)$.\par
    The second conclusion is a direct corollary of the Birkhoff Ergodic Theorem.
\prvd
\vspace{0.5em}

\subsection*{Problem.1}
a. Show that if $G$ is an amenable group that acts continuously
and linearly on a compact convex subset $C$ of a locally convex
topological vector space, then $G$ has a fixed point.\par
b. Then show that if $N$ is an amenable normal subgroup of $G$ and $G/N$ is amenable, then so is $G$.\par
c. Conclude that solvable groups are amenable.
\vspace{0.5em}\\
\textbf{Sol.} \par
a. Here we assume $\mu$ is a $G$-invariant measure on $C$, and $c$ is the barycenter of $C$, if $gc \neq c$ for some $g\in G$, let $f$ be an element in the dual of the topological vector space such that $f(c) \neq f(gc)$. And then we know 
\[f(c) = \mu(f) = \int_C fd\mu = \int_C fdg_*\mu  = \mu(g_*f) = f(gc)\mu\]
which is a contradiction and hence $c$ is a $G$-fixed point.\par
b. Assume $G$ acts continuously on $X$ which is a compact metric space, then consider $C \subset \M^1(X)$ all the $N$-invariant measure which is a convex subset of $\M^1$, and since $G/N$ is amenable and it actss continuously and linearly on $C$, notice for any measurable set in $X$, we have
\[
g_*\mu(n^{-1}A) = \mu(g^{-1}n^{-1}A) = \mu((ng)^{-1}A) = \mu((gn')^{-1}A) = \mu(n'^{-1}g^{-1}A) = \mu(g^{-1}A) = g_*\mu(A)
\]
and hence $g_*\mu$ is also $N$-invariant for any $g\in G$ and hence $G/N$ acts continously and linearly on $C$, which means there is a fixed point in $C$ which is obviously a $G$-invariant measure.\par
c. Notice $\{e\}$ is abelian and hence it is amenable, so all normal subgroup in the subnormal series are amenable according to the conclusion in (b) and hence all solvable groups are amenable. 
\prvd
\vspace{0.5em}


\addappheadtotoc

\end{document}
