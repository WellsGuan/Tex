%!TEX program = xelatex
\documentclass[lang=en,11pt,a4paper,citestyle =authoryear]{elegantpaper}

% 标题
\title{Homework04 - MATH 758}
\author{Boren(Wells) Guan}

% 本文档命令
\usepackage{array,url,stix,enumerate}
\usepackage{subfigure}
\newcommand{\ccr}[1]{\makecell{{\color{#1}\rule{1cm}{1cm}}}}
\newcommand{\code}[1]{\lstinline{#1}}
\newcommand{\prvd}{$\hfill \qedsymbol$}
\newcommand{\Z}{\mathbb{Z}}
\newcommand{\R}{\mathbb{R}}
\newcommand{\N}{\mathbb{N}}
\newcommand{\C}{\mathbb{C}}
\newcommand{\Q}{\mathbb{Q}}
\newcommand{\M}{\mathcal{M}}
\newcommand{\B}{\mathcal{B}}
\newcommand{\X}{\mathcal{X}}
\newcommand{\Hil}{\mathcal{H}}
\newcommand{\range}{\mathcal{R}}
\newcommand{\nul}{\mathcal{N}}

% 文档区
\begin{document}

% 标题
\maketitle

\subsection*{Before Reading:}\par
To make the proof more readable, I will miss or gap some natural or not important facts or notations during my writing. If you feel it hard to see, you can refer the appendix after the proof, where I will try to explain some simple conclusions (will be marked) more clearly. In case that you misunderstand the mark, I will add the mark just after those formulas between \$ and before those between \$\$.\par
And I have to claim that the appendix is of course a part of my assignment, so the reference of it is required. Enjoy your grading!

\subsection*{Problem.1} 
A topological flow is a continuous $\R$-action on a topological space $X$. Show that every topological flow on the circle either has a fixed poiint or is conjugate to a flow where for some fixed $\lambda$, and for all $t\in \R$, $t$ acts by rotating the circle by $t\lambda$ radians.
\vspace{0.5em}\\
\textbf{Sol.} \par
Consider a topological flow $F$ acts on $S^1$, if $F$ is not minimal, then we know it has a nowhere dense minimal set $M$ and assume $x\in M$, if there exists $t\in \R, t\neq 0$ such that $t(x)\ neq x$, then we know $[0,t]x$ is a simple curve in $M$ which contradictions to $M$ is nowhere dense, so $x$ is a fiexed point of $F$ or $F$ is a minimal action on $S^1$.\par

\prvd
\vspace{0.5em}

\subsection*{Problem.3} 
Show taht rotation numer is a class function on $Homeo(S^1)$ i.e. two homeomorphisms have the same rotation number if they are in the same conjugacy class.
\vspace{0.5em}\\
\textbf{Sol.} \par
    Consider $f$ is conjugate to $h$ by $H$ and if $f$ does not have irrational rotation number, then $f$ has fixed points or period point, which mean $\tau(f)$ has irrational and minimial where $\tau(f)$ is $f$ added with an irrational constant rotation, and notice $\tau(f)$ is 
\prvd
\vspace{0.5em}

\subsection*{Problem.1}
a. Show that if $G$ is an amenable group that acts continuously
and linearly on a compact convex subset $C$ of a locally convex
topological vector space, then $G$ has a fixed point.\par
b. Then show that if $N$ is an amenable normal subgroup of $G$ and $G/N$ is amenable, then so is $G$.\par
c. Conclude that solvable groups are amenable.
\vspace{0.5em}\\
\textbf{Sol.} \par
a. Here we assume $\mu$ is a $G$-invariant measure on $C$, and $c$ is the barycenter of $C$, if $gc \neq c$ for some $g\in G$, let $f$ be an element in the dual of the topological vector space such that $f(c) \neq f(gc)$. And then we know 
\[f(c) = \mu(f) = \int_C fd\mu = \int_C fdg_*\mu  = \mu(g_*f) = f(gc)\mu\]
which is a contradiction and hence $c$ is a $G$-fixed point.\par
b. Assume $G$ acts continuously on $X$ which is a compact metric space, then consider $C \subset \M^1(X)$ all the $N$-invariant measure which is a convex subset of $\M^1$, and since $G/N$ is amenable and it actss continuously and linearly on $C$, notice for any measurable set in $X$, we have
\[
g_*\mu(n^{-1}A) = \mu(g^{-1}n^{-1}A) = \mu((ng)^{-1}A) = \mu((gn')^{-1}A) = \mu(n'^{-1}g^{-1}A) = \mu(g^{-1}A) = g_*\mu(A)
\]
and hence $g_*\mu$ is also $N$-invariant for any $g\in G$ and hence $G/N$ acts continously and linearly on $C$, which means there is a fixed point in $C$ which is obviously a $G$-invariant measure.\par
c. Notice $\{e\}$ is abelian and hence it is amenable, so all normal subgroup in the subnormal series are amenable according to the conclusion in (b) and hence all solvable groups are amenable. 
\prvd
\vspace{0.5em}


\addappheadtotoc

\end{document}
