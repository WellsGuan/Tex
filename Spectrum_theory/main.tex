
%%%%%%%%%%%%%%%%中文%%%%%%蓝色标题%%%    
\documentclass[lang=en, color=blue, ]{elegantbook}
%%%使用包
\usepackage{amsmath, amssymb, amstext,mathrsfs}

%%%标题
\title{Notes for Func Analysis Rudin}
%%%作者
\author{Wells Guan}
%%%封面中间色块
\definecolor{customcolor}{RGB}{102,102,255}
\colorlet{coverlinecolor}{customcolor}
%%%封面图

%%%自定义符号区
    %%% 组合数, 在数学环境中使用
\newcommand{\per}[2]{\left(\begin{array}{c} #1 \\ #2 \end{array}\right)}
\newcommand{\proba}[1]{\mathsf{P}(#1)}
%%%文档
\newcommand{\cov}{\text{cov}}
\newcommand{\var}{\text{var}}
\newcommand{\E}{\mathbb{E}}
\newcommand{\WN}{\varepsilon}
\newcommand{\pushop}{\mathscr{B}}
\newcommand{\F}{\mathcal{F}}
\newcommand{\R}{\mathbb{R}}
\newcommand{\Q}{\mathbb{Q}}
\newcommand{\N}{\mathbb{N}}
\newcommand{\C}{\mathbb{C}}
\newcommand{\B}{\mathcal{B}}
\newcommand{\ParZ}{\dfrac{\partial}{\partial z}}
\newcommand{\ParbZ}{\dfrac{\partial}{\partial \bar{z}}}
\newcommand{\ParX}{\dfrac{\partial}{\partial x}}
\newcommand{\ParY}{\dfrac{\partial}{\partial y}}
\begin{document}

%%%封面页

%%%正文

%%% Stochastic Processes
\chapter{}

\section*{Fundamental Concepts}

\begin{definition}
A $complex\ algebra$ is a complex v.s. $A$ with a multiplication with $x(yz) = (xy)z, (x+y)z = xz+yz, x(y+z) = xy+xz$ and $cxy = (cx)y = x(cy)$ for any $x,y,z\in A$ and $c\in \C$.\par
If $A$ is a Banach space with $||xy|| \leq ||x||||y||$ and there is a unit element $e$ in $A$ i.e. $xe = ex = x, ||e||=1$ for any $x\in A$, then we call $A$ is a Banach algebra.
\end{definition}

\begin{lemma}
    If $Y$ is a complete n.v.s, then so is $L(X,Y)$.
\end{lemma}

\begin{lemma}
    (The Open Mapping Theorem) Let $X,Y$ be Banach spaces. If $T\in L(X,Y)$ is surjective, then $T$ is open.
\end{lemma}

\begin{theorem}
    Assume that $A$ is a Banach space as well as a complex algebra with unit element $e \neq 0$, in which multiplication is left-continuous and right-continuous. Then there is a norm on $A$ which induces the same topology as the given one and which makes $A$ into a Banach algebra.
\end{theorem}
\begin{proof}\par
    Consider $T: X\to B$ by $x\mapsto M_x$ where $M_x(y) = xy$ which is obviously a bounded linear map and $B$ is the subspace of $L(X,X)$ of all these maps. Then we have
    \[||M_xM_y|| \leq  ||M_x||||M_y||\quad ||M_e|| = 1\]
    and notice
    \[||x|| \leq ||M_x||||e||\]
    which means $T^{-1}$ is continuous, and notice if $M_{x_n}$ Cauchy, then
    \[M' y = xy\]
    and hence $M' = M_x$, which means $B$ will become a Banach space, and hence $T$ is open by the Open Mapping Theorem, which means $T$ is continuous and then a isometry from $X$ to $B$, where $B$ is a Banach algebra and so does $X$.
\end{proof}

\begin{definition}
    Suppose $A$ is a complex algebra and $\phi$ is a linear functional on $A$ which is not identically $0$. If
    \[\phi(xy) = \phi(x)\phi(y)\]
    for all $x,y\in A$, then $\phi$ is called a complex homomorphism on $A$.
\end{definition}

\begin{proposition}
    If $\phi$ is a complex homomorphism on a complex algebra $A$ with unit $e$, then $\phi(e) = 1$ and $\phi(x) \neq 0$ for every invertible $x\in A$.
\end{proposition}
\begin{proof}\par
    There is $y$ such that $\phi(y) \neq 0$, then it is easy to check $\phi(e) = 1$ and hence the invertible element can not be mapped to $0$.
\end{proof}

\begin{theorem}
    Suppose $A$ is a Banch algebra, $x\in A$, $||x||<1$. Then\par
    a. $e-x$ is invertible,\par
    b. $||(e-x)^{-1}-e-x||\leq \dfrac{||x||^2}{1-||x||}$.\par
    c. $|\phi(x)|<1$ for every complex homomorphism $\phi$ on $A$.
\end{theorem}
\begin{proof}
    We only need to prove c, since for any $|\lambda| \geq 1$, we know
    \[
    \phi(e-\lambda^{-1}x) = 1 - \lambda^{-1}\phi(x)\neq 0
    \]
    which means $\phi(x)$ has to be strictly less than $1$. 
\end{proof}

\begin{lemma}
    Suppose $f$ is an entire function of one complex variable, $f(0) = 1, f'(0) = 0$, and
    \[0 < |f(\lambda)| \leq e^{|\lambda|}\]
    Then $f(\lambda) = 1$ for all $\lambda \in \C$.
\end{lemma}

\begin{theorem}
    If $\phi$ is a linear functional on a Banach algebra $A$, such that $\phi(e)=1$ and $\phi(x) \neq 0$ for every invertible $x\in A$, then
    \[\phi(xy) = \phi(x)\phi(y)\]
    and $\phi$ is continuous.
\end{theorem}
\begin{proof}
    Here consider $N$ to be the null space of $\phi$, then for any $x\in A$, $x = a + \phi(x)e$ where $a \in N$ and then
    \[\phi(xy) = \phi(ab) + \phi(x)\phi(y)\]
    so it suffices to show $ab\in N$ for any $a\in N, b\in N$, which is equivalent to $a^2 \in N$ for any $a\in N$, since if so, then $\phi(x)^2 = \phi(x)^2$, and we have
    \[
    \phi(xy+yx) = 2\phi(x)\phi(y)
    \]
    which means $ax+xa \in N$ for any $a\in N, x\in A$, then consider
    \[
    (xy-yx)^2+(xy+yx)^2 = 2x(yxy)+2(yxy)x
    \] 
    and hence if $x\in N$, then $xyxy + yxyx, (xy+yx)^2 \in N$ which means $(xy-yx)^2 \in N$, and then $xy - yx\in N$.\par
    Now we will show that $a^2 \in N$ for any $a\in N$, consider since $x$ invertible is not in $N$, so $||e-x|| \geq 1$ for any $x\in N$ and hence
    \[||\lambda e-x|| \geq |\lambda| = (\phi(\lambda e-x))\]
    for any $x\in N, \lambda \in C$ which means $||\phi(x)|| \leq ||x||$ for any $x\in A$ and hence $\phi$ is continuous.\par
    Then we may assume $a\in N, ||a|| = 1$ and consider
    \[f(\lambda) = \sum\limits \dfrac{\phi(a^n)}{n!}\lambda^n\]
    where $f(0) = 1, f'(0) = 0$ entire.\par
    To show $f$ is nonzero, consider
    \[E(\lambda) == \sum \dfrac{\lambda^n}{n!}a^n\]
    where $a^0 = e$ and then we know $f(\lambda) = \phi(E(\lambda))$, and notice $E(\lambda+\mu) = E(\lambda)E(\mu)$ since
    \[
    ||x_ny_n - xy|| \leq ||x_n||||y_n-y||+||y||||x-x_n||
    \]
    and then $E(\lambda)E(-\lambda) = e$ and hence $E(\lambda)$ is invertible, so $\phi(E(\lambda)) \neq 0$. To sum up, $f = 1$ on $\C$ and hence $0 = f'' = 0\phi(a^2)$. 
\end{proof}

\begin{definition}
    Let $A$ be a Banach algebra, let $GL(A)$ be the set of all invertible elements of $A$, which is a group under the multiplication.\par
    For $x\in A$, the $spectrum$ $\sigma(x)$ of $x$ is the set of all complex numbers $\lambda$ such that $\lambda e -x$ is not invertible.\par
    The $spectral\ radius$ of $x$ is the number
    \[\rho(x) = \sup \{|\lambda|: \lambda \in \sigma(x)\}\]
\end{definition}

\begin{theorem}
Suppose $A$ is a Banach algebra, $x\in GL(A), h \in A$ and $||h|| < \tfrac{1}{2}||x^{-1}||^{-1}$. Then $x+h \in G(A)$ and
\[||(x+h)^{-1} - x^{-1} + x^{-1}hx^{-1}|| \leq 2||x^{-1}||^3||h||^2\]
\end{theorem}
\begin{proof}
    $(x+h) = x(e+x^{-1}h)$ and hence $x+h \in GL(A)$ since $||x^{-1}h|| < \tfrac{1}{2}$. And
    \[
    ||(x+h)^{-1} - x^{-1}+x^{-1}hx^{-1}|| \leq ||(e+x^{-1}h)^{-1}-e+hx^{-1}||||x^{-1}|| \leq \dfrac{||x^{-1}h||^2}{1 - ||x^{-1}h||} \leq 2||x^{-1}||^3||h||^2
    \]
\end{proof}

\begin{corollary}
    If $A$ is a Banach algebra, then $G(A)$ is an open subset of $A$ and the mapping $x\to x^{-1}$ is a homeomorphism of $GL(A)$ onto $GL(A)$.
\end{corollary}

\begin{theorem}
    If $A$ is a Banach algebra and $x\in A$, then\par
    a. the spectrum $\sigma(x)$ of $x$ is compact and nonempty\par
    b. the spectral readius $\rho(x)$ of $x$ satisfies
    \[\rho(x) = \lim_{n\to\infty} ||x^n||^{1/n} = \inf_{n\geq 1} ||x^n||^{1/n}\]
\end{theorem}
\begin{proof}
    
\end{proof}
    a. Consider $\phi:\C \to A$ by $\lambda\mapsto \lambda e-x$ and it is easy to check $\phi$ is continuous, and then consider $\phi^{-1}{\sigma(x)^c}$ is an open set and hence $\sigma(x)$ is closed. For $|\lambda| > ||x||$, $e- \lambda^{-1}x \in GL(A)$ and hence $\sigma(x) \subset \overline{B(0,||x||)}$ and hence $\sigma(x)$ is a bounded closed set in $\C$, so it is compact.\par
    Now denote $ U = \sigma(x)^c$ and define $f:U \to A$ by $f(\lambda) = (\lambda e -x)^{-1}$ which is continuous, then we know
    \[f(\mu) - f(\lambda) + (\mu-\lambda)f(\lambda)^2  \leq 2||f(\lambda)||^3||\mu-\lambda||^2\]
    for $\mu$ close to $\lambda$ and hence
    \[\lim_{\mu\to\lambda} \dfrac{f(\mu)-f(\lambda)}{\mu-\lambda} = - f^2(\lambda)\]
    which means $f$ is a holomorphic $A$-valued function, then notice for $|\lambda| > ||x||$
    \[
    f(\lambda) = \lambda^{-1} \sum \lambda^{-n}x^n
    \]
    which converges uniformly on $\partial D(0,r)$, then we may know
    \[
    \dfrac{1}{2\pi i}\int_{\partial D(0,r)} \lambda^k f(\lambda) d\lambda= \dfrac{1}{2\pi i} \sum \int_{\partial D(0,r)} \lambda^{k-n-1}x^n = x^k
    \]
    but if $\sigma(x)$ is empty, then let $k = 0$ we know the integral should be $0$ since $f$ is holomorphic on $\C$, but then we will know $e = 0$ which is a contradiction, so $\sigma(x)$ is nonempty.\par
    b. We use the $f$ above, and we know $f$ is holomorphic on $\partial D(0,r)$ for any $r>\rho(x)$, then we consider
    \[M(r) = \sup_{\theta}|f(re^{i\theta})| < \infty\]
    and $||x^n||\leq r^{n+1}M_r$ and hence
    \[\limsup_{n\to\infty} ||x^n||^{1/n} \leq r\]
    for any $r> \rho(x)$, which means $\rho(x) \leq \limsup ||x^n||^{1/n}$.\par
    Notice
    \[(\lambda e^n - x^n)y = (\lambda e -x)(\lambda^{n-1} e +\cdots + x^{n-1})y\]
    for any $y\in A$, and hence $\lambda \in \sigma(x)$ will imply $\lambda^n \in \sigma(x^n)$, so we have
    \[\lambda \leq ||x^n||^{1/n}\]
    for any integer $n$ and hence $\rho(x) = \sup\{|\lambda|, \lambda \in \sigma(x)\} \leq \inf ||x^n||^{1/n}$.
\end{document}