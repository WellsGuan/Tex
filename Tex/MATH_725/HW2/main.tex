%!TEX program = xelatex
\documentclass[lang=en,11pt,a4paper,citestyle =authoryear]{elegantpaper}

% 标题
\title{Homework02 - MATH 725}
\author{Boren(Wells) Guan}

% 本文档命令
\usepackage{array,url,stix}
\usepackage{subfigure}
\newcommand{\ccr}[1]{\makecell{{\color{#1}\rule{1cm}{1cm}}}}
\newcommand{\code}[1]{\lstinline{#1}}
\newcommand{\prvd}{$\hfill \qedsymbol$}
\newcommand{\Z}{\mathbb{Z}}
\newcommand{\R}{\mathbb{R}}
\newcommand{\N}{\mathbb{N}}
\newcommand{\C}{\mathbb{C}}
\newcommand{\Q}{\mathbb{Q}}
\newcommand{\M}{\mathcal{M}}
\newcommand{\B}{\mathcal{B}}
\newcommand{\X}{\mathcal{X}}
\newcommand{\Hil}{\mathcal{H}}
\newcommand{\range}{\mathcal{R}}
\newcommand{\nul}{\mathcal{N}}

% 文档区
\begin{document}

% 标题
\maketitle

\subsection*{Before Reading:}\par
To make the proof more readable, I will miss or gap some natural or not important facts or notations during my writing. If you feel it hard to see, you can refer the appendix after the proof, where I will try to explain some simple conclusions (will be marked) more clearly. In case that you misunderstand the mark, I will add the mark just after those formulas between \$ and before those between \$\$.\par
And I have to claim that the appendix is of course a part of my assignment, so the reference of it is required. Enjoy your grading!

\subsection*{Section 6.1 Ex.5} 
Suppose $0<p<q<\infty$. Then $L^p$ is not contained by $L^q$ iff $X$ contains sets of arbitrarily small positive measure, and $L^q$ is not in $L^p$ iff $X$ contains sets of arbitratily large finite measure. What about $q = \infty$?
\vspace{0.5em}\\
\textbf{Sol.} \par
We start with the proof of the first conclusion. To see the sufficiency, there exists $f\in L^p$ such that $||f||_q = \infty$. If there does not exists arbitrarily small positive measure sets, then we consider
\[ 0= \mu(\{|f|=\infty\}) = \mu(\bigcap_{n\geq 0} \{|f|>n\}) = \lim_{n\to\infty} \mu(\{|f|>n\})\]
and hence there exists an integer $N$ such that $\mu(\{|f|>N\}) = 0$ or there will be a contradiction. Then we know
\[\int |f|^q d\mu \leq \int |f|^pN^{q-p}d\mu\ = ||f||_p^pN^{q-p}\]
and hence $f\in L^q$, which is a contradiction. So see the necessity, consider there exists disjoint measurable sets $\{E_n\}_{n\geq 0}$ such that $ 0< \mu(E_n) < 2^{-n}$, then let $f= \mu(E_n)^{-\tfrac{1}{q}}\chi_{E_n}$ and we know
\[\int |f|^p d\mu = \sum\limits_{n\geq 0}\mu(E_n)^{\tfrac{q-p}{q}} < \sum\limits_{n\geq 0}(2^{\tfrac{p-q}{p}})^n < \infty\]
since $0 < 2^{\tfrac{p-q}{q}} < 1$. Also, notice
\[\int |f|^q d\mu = \sum\limits_{n\geq 0}1\]
which is not convergent and hence $f\in L^p$ but not in $L^q$.\par
Then we will show the second conclusion. Assume $f\in L^q$ but not in $L^p$. To show the sufficiency, firstly, we claim that there is no simple function $\phi\geq 0$ such that $\phi\leq |f|^p$ and $\int \phi d\mu= \infty$, since if $\int\phi = \infty$, then we know there exists $E$ measurable such that $\mu(E) = \infty$ and $|f|^p \geq \phi(x) > 0$ where $x\in E$ and hence $|f|^q \geq \phi(x)^{\tfrac{q}{p}} > 0$ on $E$ and then $\int |f|^q d\mu = \infty$, which is a contradiction. Then we notice that
\[\infty > \int |f|^q d\mu = \int_{|f|\geq 1} |f|^q d\mu + \int_{|f|<1} |f|^q d\mu \geq \int_{|f|\geq 1} |f|^p d\mu\]
and hence $\int_{|f|<1} |f|^p d\mu = \infty$ and then we may find simple functions $\phi_n$ such that $\phi_n \leq |f|^p \leq 1$ and $\int \phi_n d\mu \geq n, n\in \N$. Assume $\phi_n$ is defined on $E_n$, then we know
\[\mu(E_n) = \int \chi_{E_n} d\mu \geq \int \phi_n d\mu \geq n\]
and $\mu(E_n) \leq \inf_{E_n}\{\phi_n\}^{-1}\int \phi_n d\mu <\infty$ and hence there are arbitrarily large finite measure sets.\par
To show the necessity, we consider there exists disjoint $\{E_n\}_{n\geq 0}$ measurable sets such that $\infty > \mu(E_N) \geq 1$, then define $f = \sum\limits_{n\geq 1} n^{-\tfrac{1}{p}}\mu(E_n)^{-\tfrac{1}{p}}\chi_{E_n}$, then
\[\begin{aligned}
    \int |f|^p &= \sum\limits_{n\geq 1} \dfrac{1}{n} = \infty \\
    \int |f|^q &= \sum\limits_{n\geq 1} \dfrac{1}{n^r}\mu(E_n)^{1-r} \leq \sum\limits_{n\geq 1} \dfrac{1}{n^r} < \infty
\end{aligned}\]
where $r = \tfrac{q}{p} > 1$, which means $f$ is in $L^q$ but not in $L^p$.\par
If there exists $f\in L^p$ but $\mu(\{|f|>n\}) > 0$ for any integer $n$, then we know
\[\int |f|^p d\mu \geq n^p\mu(\{|f|>n\})\]
which means $\mu(\{|f|>n\}) \leq ||f||_p^p/n^p \to 0, n\to\infty$ and hence there exists arbitrarily small postive measure sets. If there exists disjoint $\{E_n\}_{n\geq 0}$ such that $0< \mu(E_n) \leq 2^{-2n}$, then consider $f = \sum\limits_{i=0}^{\infty} 2^{n/p}\chi_{E_n}$ and we know
\[\int |f|^p d\mu \leq \sum\limits_{n\geq 0} <\infty\]
and hence $f\in L^p$ but not in $L^{\infty}$.\par
For the second part, consider a dirac measure on $\R$ at $0$ such that $\mu(E) = 0$ if $0\notin E$, $\mu(E=\infty)$ if $0\in E$, and $f = 1$ on $\R$, then $f\in L^{\infty}$ but not in $L^p$, however, there is not any finite postive measure set. which is a counterexample for the sufficiency.\par
But the necessity is true, since there exists disjoint $\{E_n\}_{n\geq 0}$ such that $\infty > \mu(E_n)\geq 1$ and define $f = \sum\limits_{n\geq 0}\chi_{E_n}$, then $f\in L^{\infty}$ and $\int |f|^p dm = \infty$, which means $L^{\infty}$ is not contained in $L^p$.\par 
\vspace{0.5em}

\subsection*{Section 6.1 Ex.7} 
If $f\in L^p\cap L^{\infty}$ for some $p<\infty$, so that $f\in L^q$ for all $q>p$, then $||f||_{\infty} = \lim_{q\to\infty} ||f||_q$.
\vspace{0.5em}\\
\textbf{Sol.} \par
For any $\epsilon > 0$, assume $N_{\epsilon} = ||f||_{\infty} - \epsilon$, then we have $m_{\epsilon} = \mu(\{|f|>N_{\epsilon}\}) > 0$, where $m_{\epsilon} \leq N_{\epsilon}^{-p}||f||_p^p < \infty$. Notice that
\[||f||_q^q = \int |f|^q d\mu \geq \int_{|f|>N_{\epsilon}} |f|^q d\mu \geq N_{\epsilon}^q m_{\epsilon}\]
and hence $||f||_q \geq N_{\epsilon} m_{\epsilon}^{\tfrac{1}{q}}$, which means $\liminf_{q\to\infty} ||f||_q \geq N_{\epsilon}$ for any $\epsilon > 0$. Therefore, $\liminf_{q\to\infty}||f||_q \geq ||f||_{\infty}$.\par
Then we know
\[||f||_q \leq ||f||_p^{\lambda}||f||_{\infty}^{1-\lambda}\]
for all $\infty>q>p$ where $\lambda = \tfrac{p}{q}$ by the Proposition 6.10 on Folland and hence
\[\limsup_{q\to\infty} ||f||_q \leq \limsup_{q\to\infty}||f||_p^{\lambda}||f||_{\infty}^{1-\lambda} = ||f||_{\infty}.\]
Therefore, $\lim_{q\to\infty} ||f||_q = ||f||_{\infty}$.
\prvd
\vspace{0.5em}

\subsection*{Section 6.1 Ex.13} 
$L^p(\R^n,m)$ is separable for $1\leq p < \infty$. However, $L^{\infty}(\R^n,m)$ is not separable.
\vspace{0.5em}\\
\textbf{Sol.} \par
Here we still recall that if $f\in L^p(\R^n,m)$, then there exists simple functions $\phi_n \to f$ in $L^p$ and $\phi_n$ is defined on finite measure set $E_n$, for each $E_n, \epsilon > 0$ there exists $K_n \subset E_n$ campact such that $\mu(E_n-K_n) < 2^{-n}\epsilon$. For any indicator function of finite measure set $E$, we know there exists $K$ campact, $U$ open such that $K\subset E \subset U$ and $\mu(U-K)$ can be arbitrarily small, then we will know that there exists $f\in C(\R^n)$ such that $f = 1$ on $K$, $f=0$ outside of $U$ and $f\geq 0$ by the Urysohn's Lemma, so then
\[\int |f-\chi_E|^p dm \leq \int |\chi_U - \chi_K|^p dm\]
where the right-hand-side can be small arbitrarily. Then by the Stone-Weierstrass' Theorem on locally compact hausdorff space, we know the polynomials on $\R^n$ is dense in $C_0(\R^n)$ in $L^{\infty}$ and hence the polynomials on $\R^n$ can be arbitrarily close to a simple function in $L^p(\R^n)$.\par
Now consider all rational coefficients polynomials defined on the rectangle $E_{n} = [-n, n]^d,n\geq 0$ which are countable denoted as $P_n$, then for any polynomial $p$ on $\R^n$, we know $p|_{E_{n}}$ can be approached by the rational polynomials defined on $E_{\alpha}$ arbitrarily close in $L^p(\R^n)$ by the Dominated Convergence Theorem, then we know for $\epsilon > 0, f\in L^p$, there exists a simple function $\phi_n$ such that
\[||f-\phi_n||_p < \epsilon\]
and we may find a simple function $\phi$ with compact support such that
\[||\phi_n-\phi||_p < \epsilon\]
then we may find a continuous function $c$ on a campact, a polynomial $p$ on a compact set to approach $\phi$ and hence there exists an element $q$ in some $P_n$ close to $p$ and hence $||f-q||_p$ can be small arbitrarily. Notice $\bigcup_n P_n$ is countable and the problem goes.
\vspace{0.5em}

\subsection*{Section 6.2 Ex.18} 
The self-duality of $L^2$ follows from Hilbert space theory, and this fact can be used to prove the Lebesgue-Radon-Nikodym theorem by the following argument due to von Neumann. Suppose that $\mu,\nu$ are positive finite measures on $(X,\M)$ and let $\lambda = \mu+\nu$.\par
a. The map $f\mapsto \int f d\nu$ is a bounded linear functional on $L^2(\lambda)$, so $\int f d\nu = \int fg d\lambda$ for some $g\in L^2(\lambda)$. Equivalently, $\int f(1-g)d\nu = \int fg d\mu$ for $f\in L^2(\lambda)$.\par
b. $0\leq g \leq 1$ $\lambda$-a.e., so we may assume $0\leq g \leq 1$ everywhere.\par
c. Let $A=\{x,g(x)<1\},B=\{x,g(x)=1\}$ and set $\nu_a(E)=\nu(A\cap E),\nu_s(E)=\nu(B\cap E)$. Then $\nu_s \perp \mu$ and $\nu_a \ll \mu$; in fact, $d\nu_a = g(1-g)^{-1}\chi_Ad\mu$.\par
\vspace{0.5em}
\textbf{Sol.} \par
a. Notice
\[|\int f d\nu| \leq \int |f|d\nu \leq |f|d\lambda = ||f\cdot 1||_1 \leq ||1||_2||f||_2 = \sqrt{\lambda(X)}||f||_2\]
by the Holder's inequality and hence $f\mapsto \int fd\nu\in L^2(\lambda)^*$. Then we know there always exists $g\in L^2(\lambda)$ such that 
\[\int f d\nu = \int fg d\lambda = \int fg d(\nu+\mu) = \int fgd\nu + \int fg d\mu\]
and hence $\int f(1-g)d\nu = \int fg d\mu$ since $\int fg d\nu$ is always finite.\par
b. Let $f = \overline{sgn(g)}$ and then we know $f\in L^2(\lambda)$ since $|f|\leq 1$, then we know 
\[ \nu(E) = \int_{E} d\nu \geq |\int_{E} fd\nu| = |\int_{E} fg d\lambda| = \int_{E} |g| d\lambda\]
for any measurable set $E$, let $E = \{|g|\geq 1+n^{-1}\}$ then we will get $\nu(E) \geq (1+n^{-1})\lambda(E) > \lambda(E)$ if $\lambda(E) > 0$, which is a contradiction and hence $\lambda(E) = 0$ for any integer $n$, and hence $|g|\leq 1$ $\lambda$-a.e.\par
Then let $f = \chi_E$, we know $\nu(E) = \int_E g d\lambda$ for any $E$ measurable and hence $g$ is always real and not negative which can be shown easily by choosing $E$ as $\{Im(g) > n^{-1}\}, \{Im(g) < -n^{-1}\}, \{g<n^{-1}\},n\in\N$. Therefore, we know $0\leq g\leq 1$ $\lambda$-a.e.\par
c. Consider $F$ measurable and $\mu(F) = 0$, then we know $\lambda(F) = \nu(F) = \int gd\lambda$ and hence $g=1$ $\lambda$-a.e. on $F$ and hence $\nu_a(F) = \nu(A\cap F) = \nu(F-B) \leq \lambda(F-B) = 0$, which means $\nu_a(F) = 0$ and $\nu_a\ll \mu$.\par
Then we know $\nu_s$ is null on $A$ and $\mu$ is null on $B$ since
\[\lambda(E) = \int_E g d\nu = \int_E d\nu = \nu(E)\]
if $E\subset B$, so $\nu_s \perp \mu$ since $A\cup B = X, A\cap B = \emptyset$.\par 
Now we recall
\[\int f(1-g)d\nu = \int fgd\mu\]
for all $f\in L^2(\lambda)$, then let $f = \chi_E\chi_A/(1-g)$ for some measurable set $E$, we will have
\[\chi_Ad\nu_a + \chi_A(1-g)d\nu_s= g(1-g)^{-1}\chi_A d\mu\]
and hence 
\[\nu_a(E) = \nu_a(EA) + \nu_s(EA) = \int_{EA} g(1-g)^{-1}d\mu = \int_E g(1-g)^{-1}\chi_Ad\mu\]
which means $d\nu_a = g(1-g)^{-1}\chi_A d\mu$.
\prvd
\vspace{0.5em}

\subsection*{Section 6.2 Ex.20} 
Suppose $\sup_n||f_n||_p < \infty$ and $f_n\to f$ a.e.\par
a. If $1<p<\infty$, then $f_n \to f$ weakly in $L^p$.\par
b. The result of (a) is false in general for $p=1$. It is, however, true for $p=\infty$ if $\mu$ is $\sigma$-finite
and weak convergence is replaced by weak* convergence.
\vspace{0.5em}\\
\textbf{Sol.} \par
a. Firstly, by Fatou's lemma, we know
\[\sup_n||f_n||_p^p \geq \liminf \int |f_n|^p d\mu \geq \int \liminf |f_n|^p d\mu = ||f||_p^p\]
and hence $f\in L^p$.\par
If suffices to show that for any $g\in L^q$,
\[\int f_ng d\mu \to \int fg d\mu\]
which can be implied from
\[\int |f_n-f||g|d\mu \to 0, n\to\infty.\]
For any $\epsilon > 0$, we know there exists $\delta > 0$ such that $\int |g|^q d\mu < \epsilon$ whenever $\mu(E)<\delta$ since $\int_E |g|^q d\mu = 0$ when $\mu(E)=0$. Then we have already mentioned several times that any simple function $\phi$ such that $0 \leq \phi \leq |g|^q$ will have $\int \phi d\mu<\infty$ and hence it can only defined on a finite measure set. Then we consider $\phi_n \uparrow |g|^q$ and $\int \phi_n d\mu \to \int |g|^q d\mu$ where $\phi_n$ simple, then we assume $\phi_n$ is defined on $E_n$, then we know
\[\int |g|^q d\mu \geq \int_{E_n} |g|^q d\mu\geq \int \phi_nd\mu \to \int |g|^q d\mu\]
and hence $\int_{X-E_n} |g|^q d\mu \to 0, n\to\infty$ since $g\in L^q$. So there is an integer $N$ such that $\int_{X-E_N}|g|^q d\mu < \epsilon$ and denote $E_N$ as $A$, and we know $\mu(A)<\infty$. So we have Egoroff's theorem and we know there exists $B\subset A$ such that $\mu(A-B)<\delta$ and $f_n\to f$ uniformly on $B$, then we have
\[\begin{aligned}
\limsup_{n\to\infty} \int |f_n-f||g|d\mu &\leq \limsup_{n\to\infty} \Big(\int_A^c |f_n-f||g|d\mu + \int_{A-B} |f_n-f||g|d\mu + \int_B |f_n-f||g|d\mu\Big) \\ &\leq (\sup_n||f_n||_p+||f||_p)\epsilon + \limsup_{n\to\infty}||f_n-f||_{L^p(B)}||g||_q \to 0, \epsilon \to 0
\end{aligned}\] 
by the Holder's inequality. Therefore, $f_n\to f$ weakly in $L^p$.\par
b. Let $f_n = n\chi_{[0,n^{-1}]}$ and we know $f_n \to 0$ a.e. and $f\mapsto \int f$ is obviouslt a bounded linear functionla on $L^1$, then we know $f_n\mapsto 1, f\mapsto 0$ which is a counterexample.\par
For $l^1$, consider $f_n = (\tfrac{1}{n},\tfrac{1}{n},\cdots,\tfrac{1}{n},\cdots), |f_n|_1 = 1$ and we know $f_n\to 0$ a.e. and define $f\mapsto \int f d\mu$ and we know $f_n\mapsto 1, f\mapsto 0$ which is a counterexample.\par
Consider $\{g_n\}_{n\geq 0}\subset L^{\infty}$ and $g_n\to g$ a.e., we know $g\in L^{\infty}$ since $\sup_n ||g_n||_{\infty} < \infty$. It is suffice to show that for any $f\in L^1$, $\int fg_n \to \int fg, n\to\infty$, and we find $A,B$ the same as the proof of (a), then we will have
\[\begin{aligned}
\limsup_{n\to\infty} \int |g_n-g||f|d\mu &\leq \limsup_{n\to\infty} \Big(\int_A^c |g_n-g|||fd\mu + \int_{A-B} |g_n-g||f|d\mu + \int_B |g_n-g||f|d\mu\Big) \\ &\leq (\sup_n||g_n||_{\infty}+||f||_{\infty})\epsilon + \limsup_{n\to\infty}||g_n-g||_{L^{\infty}(B)}||g||_1 \to 0, \epsilon \to 0
\end{aligned}\] 
and hence $g_n\to g$ in weak* topology.\par
\prvd
\vspace{0.5em}

\addappheadtotoc

\end{document}
