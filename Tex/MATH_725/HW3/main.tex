%!TEX program = xelatex
\documentclass[lang=en,11pt,a4paper,citestyle =authoryear]{elegantpaper}

% 标题
\title{Homework03 - MATH 725}
\author{Boren(Wells) Guan}

% 本文档命令
\usepackage{array,url,stix}
\usepackage{subfigure}
\newcommand{\ccr}[1]{\makecell{{\color{#1}\rule{1cm}{1cm}}}}
\newcommand{\code}[1]{\lstinline{#1}}
\newcommand{\prvd}{$\hfill \qedsymbol$}
\newcommand{\Z}{\mathbb{Z}}
\newcommand{\R}{\mathbb{R}}
\newcommand{\N}{\mathbb{N}}
\newcommand{\C}{\mathbb{C}}
\newcommand{\Q}{\mathbb{Q}}
\newcommand{\M}{\mathcal{M}}
\newcommand{\B}{\mathcal{B}}
\newcommand{\X}{\mathcal{X}}
\newcommand{\Hil}{\mathcal{H}}
\newcommand{\range}{\mathcal{R}}
\newcommand{\nul}{\mathcal{N}}

% 文档区
\begin{document}

% 标题
\maketitle

\subsection*{Before Reading:}\par
To make the proof more readable, I will miss or gap some natural or not important facts or notations during my writing. If you feel it hard to see, you can refer the appendix after the proof, where I will try to explain some simple conclusions (will be marked) more clearly. In case that you misunderstand the mark, I will add the mark just after those formulas between \$ and before those between \$\$.\par
And I have to claim that the appendix is of course a part of my assignment, so the reference of it is required. Enjoy your grading!

\subsection*{Section 6.3 Ex.31} 
(A Generalized Holder Inequality) Suppose that $1\leq p_j \leq \infty$ and $\sum\limits_{j=1}^n p_j^{-1} = r^{-1} \leq 1$. If $f_j \in L^{p_j}$ for $j = 1,\cdot,n$ then $\prod_{i=1}^n f_j \in L^r$ and $||\prod_1^n f_j||_r \leq \prod_{i=1}^n ||f_j||_{p_j}$.
\vspace{0.5em}\\
\textbf{Sol.} \par
If $n=2$, we have
\[
(\int |f_1f_2|^r)^{r^{-1}} \leq [(\int (|f_1|^r)^{p_1/r})^{r/p_1}(\int (|f_2|^r)^{p_2/r})^{r/p_2}]^{r^{-1}} = ||f_1||_{p_1}^r||f_2||_{p_2}
\]
by the Holder's inequality and hence the inequality holds.\par
Use the induction to $n$ and assume the inquality holds when $n=k$, if $n=k+1$, notice
\[
\sum\limits_{j=1}^k {p_j/r}^{-1} = (\dfrac{p_{k+1}}{p_{k+1}-r})^{-1}
\]
and then we have
\[
||\prod_{j=1}^k g_j||_{p_{k+1}/(p_{k+1}-r)} \leq \prod_{j=1}^k ||g_j||_{p_j/r}
\]
for any $g_j \in L^{p_j/r}$ and hence
\[
\begin{aligned}
||\prod_{j=1}^{k+1} f_j||_{r} &\leq ||f_{k+1}||_{p_{k+1}}||\prod_{j=1}^{k} f_j||_{rp_{k+1}/(p_{k+1}-r)} \\ &= ||f_{k+1}||_{p_{k+1}}||||\prod_{j=1}^{k} f_j^r||_{p_{k+1}/(p_{k+1}-r)}^{r^{-1}} \\ &\leq ||f_{k+1}||_{p_{k+1}}||\prod_{j=1}^k ||f_j^r||_{p_j/r}^{r^{-1}} = \prod_{j=1}^{k+1} ||f_j||_{p_j}
\end{aligned}
\]
Therefore, the inequality holds.
\prvd
\vspace{0.5em}

\subsection*{Section 6.3 Ex.32} 
Suppose that $(X,\M,\mu)$ and $(Y,\N,\nu)$ are $\sigma$-finite measure spaces and $K\in L^2(\mu\times\nu)$. If $f\in L^2(\nu)$, the integral $Tf(x) = \int K(x,y)f(y)d\nu(y)$ converges absolutely for a.e. $x\in X$; moreover ,$Tf \in L^2(\mu)$ and $||Tf||_2 \leq ||K||_2||f||_2$.
\vspace{0.5em}\\
\textbf{Sol.} \par
Notice $||K(x,\cdot)||_{L^2(\nu)} < \infty$ for a.e. $x$ by the Fubini's theorem and then we have
\[
\begin{aligned}
||Tf||_2^2 = \int (\int |K(x,y)f(y)| d\nu(y))^2 d\mu(x) &\leq \int \int ||f||^2_{L^2(\nu)} \int K(x,y)^2 d\nu(y) d\mu(x) \\ &= ||f||_{L^2(\nu)}^2 ||K||_{L^2(\mu\times\nu)}^2
\end{aligned}
\]
by the Holder's inequality and hence $Tf \in L^2(\mu)$, in particular $Tf(x) < \infty$ for a.e. $x\in X$.
\prvd
\vspace{0.5em}

\subsection*{Section 6.4 Ex.36} 
If $f\in \text{weak }L^p$ and $\mu(\{x:f(x)\neq 0\}) < \infty$, then $f\in L^q$ for all $q<p$. On the other hand, if $f\in\text{weak }L^p \cap L^{\infty}$, then $f\in L^q$ for all $q>p$.
\vspace{0.5em}\\
\textbf{Sol.} \par
To show the first conclusion, notice
\[
\begin{aligned}
\int |f|^q d\mu &=q\int_0^{\infty} x^{q-1}\lambda_f(x)dx \\ 
& \leq q\int_0^{1} \lambda_f(x)dx + q\int_1^{\infty} x^{-1-(p-q)}[f]_p^pdx \\
& \leq \mu(\{x:f(x)\neq 0\}) + qx^{q-p}|^{1}_{\infty} < \infty
\end{aligned}
\]
by the proposition 6.24 on Folland and hence $f\in L^q$ for all $q<p$.\par
Then we will show the second conclusion, notice
\[
\begin{aligned}
\int |f|^q d\mu = q\int_0^{\infty} x^{q-1}\lambda_f(x)dx &= q\int_0^{||f||_{\infty}} x^{q-1}\lambda_f(x)dx \\
&\leq q\int_0^{||f||_{\infty}} x^{q-p-1}[f]_p^pdx = \dfrac{q[f]_p^p}{q-p}x^{q-p}|^{||f||_{\infty}}_0 < \infty
\end{aligned}
\]
and hence $f\in L^q$ for all $q>p$ with the conditions.
\vspace{0.5em}

\subsection*{Section 6.4 Ex.40} 
If $f$ is a measurable function on $X$, its decreasing rearrangement is the function $f^*:(0,\infty) \to [0,\infty]$ defined by
\[ f^*(t) = \inf\{\alpha:\lambda_f(\alpha) \leq t\}\]\par
a. $f^*$ is decreasing. If $f^*(t) < \infty$ then $\lambda_f(f^*(t)) \leq t$, and if $\lambda_f(\alpha) < \infty$ then $f^*(\lambda_f(\alpha)) \leq \alpha$.\par
b. $\lambda_f = \lambda_{f^*}$, where $\lambda_{f^*}$ is defined with respect to Lebesgue measure on $(0,\infty)$.\par
c. If $\lambda_f(\alpha) < \infty$ for all $\alpha > 0$ and $\lim_{\alpha \to \infty} \lambda_f(\alpha) = 0$, and $\phi\geq 0$ is a Borel measurable function on $(0,\infty)$, then $\int_X \phi \circ |f|d\mu = \int_0^{\infty} \phi\circ f^*(t)dt$. In particular, $||f||_p = ||f^*||_p$ for $0<p<\infty$.\par
d. If $0<p<\infty$, $[f]_p = \sup_{t>0} t^{1/p} f^*(t)$.\par
e. The name "rearrangement" for $f^*$ comes from the case where $f$ is a nonnegative function on $(0,\infty)$. To see why it is appropriate, pick a step function on $(0,\infty)$ assuming four or five different values and draw the graphs of $f$ and $f^*$.\par
\vspace{0.5em}
\textbf{Sol.} \par
a. Consider $s\geq t$, then for any $\alpha > f^*(t)$, if suffices to show $f^*(t) \geq f^*(s)$ and we know
\[ \lambda_f(\alpha) \leq t \leq s\]
and hence $\alpha \geq f^*(s)$ for any $\alpha > f^*(t)$, which means $f^*(t) \geq f^*(s)$.\par
If $f^*(t)<\infty$, for any $\alpha > f^*(t)$, we know $\lambda_f(\alpha) \leq t$ and hence $\lambda_f(f^*(t)) \leq t$ since $\lambda_f$ is right-continuous.\par
If $\lambda_f(\alpha) < \infty$, then we know $\lambda_f{\alpha} \leq \lambda_f{\alpha}$ and hence
\[\alpha \geq \int\{\beta, \lambda_f(\beta) \leq \lambda_f(\alpha)\} = f^*(\lambda_f(\alpha))\]\par
b. If $\lambda_f(\alpha) < \infty$, then we know $\lambda_{f^*}(\alpha) = m(\beta, f^*(\beta) > \alpha) \leq \lambda_f(\alpha)$ since $f^*(\lambda_f(\alpha)) \leq \alpha$ and $f^*$ decreasing. Then since for any $\epsilon > 0$,
\[
f^*(\lambda_{f}(\alpha)-\epsilon) = \inf\{\beta, \lambda_f(\beta) \leq \lambda_{f}(\alpha) - \epsilon\} \geq \alpha
\]
and hence $\lambda_{f^*}(\alpha) \geq \lambda_f(\alpha)-\epsilon$ for any $\epsilon > 0$, which means $\lambda_f(\alpha) = \lambda_{f^*}(\alpha)$ if $\lambda_f(\alpha) < \infty$.\par
If $\lambda_f(\alpha) = \infty$, then we know for any $t\in(0,\infty)$, there exists $\epsilon > 0$ such that $\lambda_f(\alpha+\epsilon) > t$ and then $f^*(t) \geq \alpha+\epsilon > \alpha$, which means $f^*(t) > \alpha$ for all $t\in(0\infty)$ and hence $\lambda_{f^*}(\alpha) = \infty = \lambda_{f}(\alpha)$.\par
c. Notice
\[ \int_X \phi\circ |f| d\mu = - \int_0^{\infty} \phi(\alpha)d\lambda_f(\alpha) = -\int_0^{\infty} \phi(\alpha) d\lambda_{f^*}(\alpha) = \int_0^{\infty} \phi \circ f^*(t) dt\]
by the Proposition 6.23 on Folland. Let $\phi(x) = |x|^p$ and then we may know $||f||_p = ||f^*||_p$.\par
d. If $[f]_p < \infty$, then $\lambda_f(\alpha) < [f]_p^p/\alpha^p \to 0, \alpha\to\infty,\lambda_f(\alpha)<\infty$ for any $\alpha \in (0,\infty)$, and hence $f^*(t) < \infty$ for any $t\in(0,\infty)$, then consider if $f^*(t)>0$, then it is easy to check
\[
\lim_{n\to\infty} \lambda_f(f^*(t)-n^{-1})(f^*(t) - n^{-1})^p = \lambda_f(f^*(t)-)f^*(t)^p \geq tf^*(t)^p
\]
where $\lambda_f(f^*(t)-)$ is the left limit of $\lambda_f$ at $f^*(t)$ and hence $[f]_p^p \geq tf^*(t)^p$ for any $t>0$, which means $[f]_p^p \geq \sup_{t>0} tf^*(t)^p$.\par
then for $\alpha > 0$, we know
\[
\lim_{n\to\infty} f^*(\lambda_f(\alpha)+n^{-1})^p(\lambda_f(\alpha+n^{-1})) = f^*(\lambda_f(\alpha)+)^p\lambda_f(\alpha) \geq \alpha^p\lambda_f(\alpha)
\]
where $f^*(\lambda_f(\alpha)+)$ is the right limit of $f^*$ at $\lambda_f(\alpha)$ and hence  $\alpha^p\lambda_f(\alpha) \leq \sup_{t>0} tf^*(t)^p$ for any $\alpha>0$, which means $[f]_p^p \leq \sup_{t>0} tf^*(t)^p$. To sum up, $[f]_p = \sup_{t>0} t^{1/p}f^*(t)$.\par
e. We choose $f = \sum\limits_{i=1}^4 i\chi_{[i-1,i)}$, then we know $\lambda_f = \sum\limits_{i=1}^4 (5-i)\chi_{[i-1,i)}$, then $f^* = \sum\limits_{i=1}^4 (5-i)\chi_{[i-1,i)}$ which is exactly a rearrangement of $f$.
\prvd
\vspace{0.5em}

\subsection*{Section 6.5 Ex.41} 
Suppose $1<p\leq \infty$ and $p^{-1}+q^{-1} = 1$. If $T$ is a bounded operator on $L^p$ such that $\int(Tf)g = \int f(Tg)$ for all $f,g\in L^p\cap L^q$, then $T$ extends uniquely to a bounded operator on $L^r$ for all $r\in [p,q]$ or $[q,p]$.
\vspace{0.5em}\\
\textbf{Sol.} \par
Consider $T:L^p\cap L^q \to L^q$ is a bounded linear operator, since
\[
|\int(Tg)f| = |\int g(Tf)| \leq ||g||_q ||Tf||_p \leq ||g||_q||T||||f||_p
\]
by the Holder's inequality and hence $T$ is bounded by the theorem 6.14 on Folland. Then since we know $L^p\cap L^q$ is dense in $L^q$, so there has to be a unique extension of $T$ on $L^q$. Then $T$ is a linear operator on $L^r$ since $L^r \subset L^p+L^q$ for any $r\in [p,q]$ or $[q,p]$, then by the Riesz-Thorin Interpolation Theorem, $T$ is bounded as a $L^r$ operator by taking $p_0 = q_0 = \min(p,q), p_1 = q_1 = \max(p,q)$.
\par
\prvd
\vspace{0.5em}

\addappheadtotoc

\end{document}
