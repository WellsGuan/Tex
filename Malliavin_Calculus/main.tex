
%%%%%%%%%%%%%%%%中文%%%%%%蓝色标题%%%    
\documentclass[lang=en, color=blue, ]{elegantbook}
%%%使用包
\usepackage{amsmath, amssymb, amstext,mathrsfs}

%%%标题
\title{Notes for Durrett Ed5}
%%%作者
\author{Wells Guan}
%%%封面中间色块
\definecolor{customcolor}{RGB}{102,102,255}
\colorlet{coverlinecolor}{customcolor}
%%%封面图

%%%自定义符号区
    %%% 组合数, 在数学环境中使用
\newcommand{\per}[2]{\left(\begin{array}{c} #1 \\ #2 \end{array}\right)}
\newcommand{\proba}[1]{\mathsf{P}(#1)}
%%%文档
\newcommand{\cov}{\text{cov}}
\newcommand{\var}{\text{var}}
\newcommand{\E}{\mathbb{E}}
\newcommand{\WN}{\varepsilon}
\newcommand{\pushop}{\mathscr{B}}
\newcommand{\F}{\mathcal{F}}
\newcommand{\R}{\mathbb{R}}
\newcommand{\Q}{\mathbb{Q}}
\newcommand{\N}{\mathbb{N}}
\newcommand{\B}{\mathcal{B}}
\begin{document}

%%%封面页

%%%正文

%%% Stochastic Processes
\chapter{}
\section{Brownian Motion}

\begin{definition}
A real-valued stochastic process $B=(B_t)_{t\geq 0}$ defined on a probability space $(\Omega,\F;P)$ is called a Brownian motion if it satisfies the following conditions:\par
a. Almost surely $B_0 = 0$.\par
b. For all $0\leq t_1 < \cdots t_n$ the increments $B_{t_n}-B_{t_{n-1}},\cdots,B_{t_2}-B_{t_1}$ are independent random variables.\par
c. If $0\leq s < t$, the increment $B_t-B_s$ is a Gaussian random variable with mean zero and variance $t-s$.\par
d. With probability one, the map $t\to B_t$ is continuous.\par
A $d$-dimensional Brownian motion is defined as an $\R^d$-valued stochastic process $B=(B_t)_{t\geq 0}$, $B_t = (B_t^1,\cdots,B_t^d)$, where $B^1,\cdots,B^d$ are $d$ independent Brownian motions.
\end{definition}

\begin{proposition}
    Properties (a),(b),(c) are equivalent to that $B$ is a Gaussian process,i.e. for any finite set of indices $t_1,\cdots,t_n$, $(B_{t_1},\cdots,B_{t_n})$ is a multivariate Gaussian random variable, equivalently, any linear combination of $B_{t_i}$ is normal distributed r.v., with mean zero and covariance function
    \[\Gamma(s,t) = \min(s,t)\]
\end{proposition}
\begin{proof}\par
    Suppose (a),(b),(c) holds, then we know $(B_{t_1},\cdots,B_{t_n})$ is normal for any finite indices and then
    \[
    \begin{aligned}
        m(t) &= E(B_t) = 0\\
        \Gamma(s,t)&=E(B_sB_t) =E(B_{\min(s,t)}^2) = \min(s,t) 
    \end{aligned}
    \]\par
    Conversly, we know $E(B_0^2) = 0$ and hence $B_0 = 0$ a.s., then we know $E(B_s^2) = s$ and for any $0<s<t$,
    \[
    E(B_s(B_t-B_s)) = 0
    \]
    and it is easy to check (c), and (b) is deduced by computing the covariance of the increments, notice that two r.v.s are independent iff $\phi_{(X_1,X_2,\cdots,X_n)} = \phi_{X_1}\phi_{X_2}\cdots\phi_{X_n}$ which implies that normal r.v.s are independent iff they have zero covariances.
\end{proof}

\begin{theorem}
    (Kolmogorov's continuity theorem) Suppose that $X= (X_t)_{t\in[0,T]}$ satisfies
    \[E(|X_t-X_s|^{\beta}) \leq K|t-s|^{1+\alpha}\]
    for all $s,t\in[0,T]$ and some constant $\beta,\alpha,K>0$. Then there exists a version $\tilde{X}$ of $X$ such that if
    \[\gamma < \alpha/\beta\]
    then
    \[|\tilde{X}_{t}-\tilde{X}_s| \leq G_{\gamma}|t-s|^{\gamma}\]
    for all $s,t\in[0,T]$ where $G_{\gamma}$ is a random variable. The trajectories of $\tilde{X}$ are Holder continuous of order $\gamma$ for any $\gamma < \alpha/\beta$.
\end{theorem}

\begin{proposition}
    There exists a version of $B$ with Holder-continuous trajectories of order $\gamma$ for any $\gamma < (k-1)/2k$ on any interval $[0,T]$.
\end{proposition}
\begin{proof}\par
    Since we know $B_t-B_s$ has the normal distribution $\mathcal{N}(0,t-s)$ and then we know
    \[
    E\Big((B_t-B_s)^{2k}\Big) = \dfrac{1}{\sqrt{2\pi(t-s)}}\int_{-\infty}^{\infty}x^{2k} \exp^{-\dfrac{x^2}{2(t-s)}} dx = (2k-1)!!(t-s)^k = \dfrac{(2k)!}{2^kk!}(t-s)^k
    \]
    and by the theorem 1.1, the proposition holds.
\end{proof}


\begin{proposition}
    Brownian motion are basic properties:\par
    a. For any $a>0$, the process $(a^{-1/2}B_{at})_{t\geq 0}$ is a Brownian motion.\par
    b. For any $h>0$, the process $(B_{t+h}-B_h)_{t\geq 0}$ is a Brownian motion.\par
    c. The process $(-B_t)_{t\geq 0}$ is a Brownian motion.\par
    d. Almost surely $\lim_{t\to\infty} B_t/t = 0$ and the process $X_t = tB_{1/t}$ for $t>0$, $X_t = 0$ for $t=0$ is a Brownian motion.
\end{proposition}
\begin{proof}\par
    a. Consider $0\leq t_1<t_2<\cdots<t_n$ and we may calculate the covariance matrix for
    \[a^{-1/2}B_{at_n}-a^{-1/2}B_{at_{n-1}},\cdots,a^{-1/2}B_{at_2}-a^{1/2}B_{at_1}\]
    by
    \[
    \begin{aligned}
    &E[(a^{-1/2}B_{at_j}-a^{-1/2}B_{at_{j-1}})(a^{-1/2}B_{at_k}-a^{-1/2}B_{at_{k-1}})]\\ = &a^{-1}(at_j\wedge at_k)-a^{-1}(at_j\wedge at_{k-1})-a^{-1}(at_{j-1}\wedge at_k) + a^{-1}(at_{j-1}\wedge at_{k-1}) \\
    = &\begin{cases}
        t_j-t_{j-1}-t_{j-1}+t_{j-1} = t_j - t_{j-1}\quad&\text{if }j=k \\
        t_j-t_j - t_{j-1}+t_{j-1} = 0 &\text{if }j<k \\
        0&\text{if }j>k
    \end{cases}
    \end{aligned}
    \]
    and hence $(a^{-1/2}B_{at})_{t\geq 0}$ satisfies the property (b) in definition 1.1, a,d are obvious and c is easy to be checked.\par
    b. Obvious.\par
    c. Obvious.\par
    d. Notice $B$ is Holder continuous. Now we only need to check that
    \[
    E(tB_{1/t}sB_{1/s}) = ts(1/t\wedge 1/s) = (t\wedge s)
    \]
    and the rest is easy to be checked.
\end{proof}

\begin{theorem}
    (The law of the iterated logarithm)
    \[\limsup_{t\to s^+} \dfrac{|B_t-B_s|}{\sqrt{2|t-s|\ln\ln|t-s|}} = 1,\quad a.s.\]
\end{theorem}

\begin{proposition}
    Fix a time interval $[0,t]$ and consider the following subvision $\pi$ of this interval:
    \[0 = t_0 < t_1 < \cdots<t_n = t\]
    The norm of the subdivision $\pi$ is defined as $|\pi| = \max_{0\leq j \leq n-1}(t_{j+1}-t_j)$. Then
    \[\lim_{|\pi|\to 0} \sum\limits_{j=0}^{n-1}(B_{t_{j+1}}-B_{t_j})^2 = t\]
    in $L^2(\Omega)$.
\end{proposition}  
\begin{proof}\par
    Consider let $\xi_j =(B_{t_{j+1}}-B_{t_j})^2 - (t_{j+1}-t_j)$ and we know $\xi_j$ are independent with mean $0$ and hence
    \[
    \begin{aligned}
    E\Big(\sum\limits_{j=0}^{n-1}(B_{t_{j+1}}-B_{t_j})^2-t\Big)^2 = \sum\limits_{j=0}^{n-1}E\xi_j^2 &= \sum\limits_{j=0}^{n-1}(3(t_{j+1}-t_j)^2 - 2(t_{j+1}-t_j)^2+(t_{j+1}-t_j)^2) \\ &= 2\sum\limits_{j=0}^{n-1}(t_{j+1}-t_j)^2 \leq 2t|\pi| \to 0
    \end{aligned}
    \]
\end{proof}

\begin{proposition}
    The total variation of Browian motion on an interval $[0,t]$ defined By
    \[V = \sup_{\pi}\sum\limits_{j=0}^{n-1}|B_{t_{j+1}}- B_{t_j}|\]
    where $\pi = \{0 = t_0<t_1<\cdots<t_n\}$ is infinite with probability one.
\end{proposition}
\begin{proof}\par
    Here we know
    \[
    \sum\limits_{j=0}^{n-1}(B_{t_{j+1}}-B_{t_j})^2 \leq V\sup_{j}|B_{t_{j+1}}-B_{t_j}|
    \]
    and hence if $V<\infty$, then
    \[
    \lim_{|\pi|\to 0}\sum\limits_{j=0}^{n-1}(B_{t_{j+1}}-B_{t_j})^2 = 0
    \]
    which means $P(V<\infty) = 0$.
\end{proof}



\end{document}