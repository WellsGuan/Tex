%!TEX program = xelatex
\documentclass[lang=en,11pt,a4paper,citestyle =authoryear]{elegantpaper}

% 标题
\title{Homework0 - Paquette}
\author{Boren(Wells) Guan}

% 本文档命令
\usepackage{array,url,stix}
\usepackage{subfigure,listings}
\newcommand{\ccr}[1]{\makecell{{\color{#1}\rule{1cm}{1cm}}}}
\newcommand{\code}[1]{\lstinline{#1}}
\newcommand{\prvd}{$\hfill \qedsymbol$}
\newcommand{\Z}{\mathbb{Z}}
\newcommand{\R}{\mathbb{R}}
\newcommand{\N}{\mathbb{N}}
\newcommand{\C}{\mathbb{C}}
\newcommand{\Q}{\mathbb{Q}}
\newcommand{\M}{\mathcal{M}}
\newcommand{\B}{\mathcal{B}}
\newcommand{\X}{\mathcal{X}}
\newcommand{\Hil}{\mathcal{H}}
\newcommand{\range}{\mathcal{R}}
\newcommand{\nul}{\mathcal{N}}
\newcommand{\F}{\mathcal{F}}

% 文档区
\begin{document}

% 标题
\maketitle

\subsection*{Exercise.1}Show the Woodfury formulam for $n$-dimensional vectors $U,V$ and a square amtrix $A$
\[R(z;A+UV^T) - R(z;A) = - \dfrac{R(z;A)UV^TR(z;A)}{1+V^TR(z;A)U}\]
with $z\in \text{Spec}(A+UV^T),\text{Spec}(A)$.
\begin{proof}
    Notice
    \[
    (A+UV^T-zI)(R(z;A+UV^T) - R(z;A))(A-zI) = -UV^T
    \]
    and hence
    \[
    (A+UV^T-zI)(R(z;A+UV^T) - R(z;A)) = -UV^TR(z;A)
    \]
    and it remains to show
    \[
    R(z;A)U = (1+U^TR(z;A)V)R(z;A+UV^T)U
    \]
    assume $R(z;A)U = a, R(z;A+UV^T)U = b$ and we may have
    \[
    b = a - bV^Ta
    \]
    and hence
    \[
    b = a/(1+V^Ta)
    \]
    and we are done.
\end{proof}

\subsection*{Exercise.2}Show the directional derivative of $R(z;A)$ in its $A$ variable in the direction of $V$ is
\[
\lim_{\epsilon \to 0}\epsilon^{-1}(R(z;A+\epsilon))
\]
which there fore gives us an expression for all partial derivatives in $A$.
\begin{proof}
    We know
    \[
    \epsilon^{-1}(R(z;A+\epsilon B) - R(z;A)) = -R(z;A+\epsilon B)BR(z:A)
    \]
    and we are done.
\end{proof}

\subsection*{Exercise.3}Suppose that $S$ is a symmetric matrix, $G$ is GOE and set $A = SGS$. Show that for $z$ with $h$
\begin{proof}
    We know
    \[
    \epsilon^{-1}(R(z;A+\epsilon B) - R(z;A)) = -R(z;A+\epsilon B)BR(z:A)
    \]
    and we are done.
\end{proof}

\addappheadtotoc

\end{document}
