%!TEX program = xelatex
\documentclass[lang=en,11pt,a4paper,citestyle =authoryear]{elegantpaper}

% 标题
\title{Homework0 - Kuijlaars}
\author{Boren(Wells) Guan}

% 本文档命令
\usepackage{array,url,stix}
\usepackage{subfigure,listings}
\newcommand{\ccr}[1]{\makecell{{\color{#1}\rule{1cm}{1cm}}}}
\newcommand{\code}[1]{\lstinline{#1}}
\newcommand{\prvd}{$\hfill \qedsymbol$}
\newcommand{\Z}{\mathbb{Z}}
\newcommand{\R}{\mathbb{R}}
\newcommand{\N}{\mathbb{N}}
\newcommand{\C}{\mathbb{C}}
\newcommand{\Q}{\mathbb{Q}}
\newcommand{\M}{\mathcal{M}}
\newcommand{\B}{\mathcal{B}}
\newcommand{\X}{\mathcal{X}}
\newcommand{\Hil}{\mathcal{H}}
\newcommand{\range}{\mathcal{R}}
\newcommand{\nul}{\mathcal{N}}
\newcommand{\F}{\mathcal{F}}

% 文档区
\begin{document}

% 标题
\maketitle

\subsection*{Exercise 0.1} The semi-circle law is given by
\[
d\mu_{sc} = \dfrac{1}{2\pi}\sqrt{4-x^2}dx\quad\text{on }[-2,2]
\]\par
a. Compute its Stieltjes transform $F(z):=\int_{-2}^2 \dfrac{d\mu_{sc}(x)}{z-x}$ for $z \in \C-[-2,2]$ by the means of contour integration.
\begin{proof}
a.We know
\[
\begin{aligned}
    F(z) = \int_{-2}^2 \dfrac{d\mu_{sc}(x)}{z-x} = \dfrac{1}{2\pi}\int_{-2}^2 \dfrac{\sqrt{4-x^2}}{z-x}dx
\end{aligned}
\]
and hence
\[
F(z) = \dfrac{z+\sqrt{z^2-4}}{2}
\]\par
b. We know
\[
\begin{aligned}
\lim_{\epsilon\to 0^+}\text{Im}F(x-i\epsilon) &= \lim_{\epsilon \to 0}\text{Im}\left(\dfrac{x-i\epsilon+\sqrt{x^2-4-\epsilon^2-2ix\epsilon}}{2}\right) \\
&= Im\left(\dfrac{\sqrt{x^2-4}}{2}\right) \\
& = \dfrac{\sqrt{}{}}{}
\end{aligned}
\]
\end{proof}

\subsection*{Exercise 0.2} Let $\mu$ be a measure on $\C$ with compact support. Show that $U^{\mu}$ is superharmonic on $\C$ and harmonic on $C - \text{supp}(\mu)$
\begin{proof}
We know by Fubini's theorem
\[
\begin{aligned}
\dfrac{1}{2\pi}\int_0^{2\pi}U^{\mu}(z_0+re^{i\theta})d\theta &= \dfrac{1}{2\pi} \int_0^{2\pi}\int \ln\dfrac{1}{|z_0+re^{i\theta}-s|}d\mu(s)d\theta \\
& = \int \dfrac{1}{2\pi} \int_0^{2\pi}\ln\dfrac{1}{|z_0+r^{i\theta} - s|}d\theta d\mu(s) \\
& \leq \int \ln \dfrac{1}{|z_0-s|} d\mu(s) \\ & = U^{\mu}(z_0)
\end{aligned}
\]
when $z_0 \in$
\end{proof}

\subsection*{Exercise 0.3}In our study of equilibrium measures we will associate with a function $V:\R \to \R\cup\{+\infty\}$ a probablity measure $\mu$ on $\R$ and a constant $l$ such that
\[
\begin{aligned}
    2U^{\mu}+V &= l,\quad\text{on the support of }\mu\\
    2U^{\mu}+V&\geq l,\quad\text{on }\R
\end{aligned}
\]
Suppose that $\mu$ is a probability measure satisfying the equations above for some constant $l$. Let $x_0$ be a point where $V$ assumes its minimum on $\R$, prove that $x_0 \in \text{supp}(\mu)$.
\begin{proof}
   If $x\notin \text{supp}(\mu)$, we know there exists $\epsilon >0 $ such that $\mu((x_0-\epsilon,x_0+\epsilon)) = 0$ and we will know
   \[
   2U^{\mu}(x_0) + V(x_0)
   \] 
\end{proof}

\addappheadtotoc

\end{document}
