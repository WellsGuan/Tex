%!TEX program = xelatex
\documentclass[lang=en,11pt,a4paper,citestyle =authoryear]{elegantpaper}

% 标题
\title{Bonus 02 - MATH 722}
\author{Boren(Wells) Guan}

% 本文档命令
\usepackage{array,url,stix}
\usepackage{subfigure}
\newcommand{\ccr}[1]{\makecell{{\color{#1}\rule{1cm}{1cm}}}}
\newcommand{\code}[1]{\lstinline{#1}}
\newcommand{\prvd}{$\hfill \qedsymbol$}
\newcommand{\Z}{\mathbb{Z}}
\newcommand{\R}{\mathbb{R}}
\newcommand{\N}{\mathbb{N}}
\newcommand{\C}{\mathbb{C}}
\newcommand{\Q}{\mathbb{Q}}
\newcommand{\M}{\mathcal{M}}
\newcommand{\B}{\mathcal{B}}
\newcommand{\X}{\mathcal{X}}
\newcommand{\Hil}{\mathcal{H}}
\newcommand{\range}{\mathcal{R}}
\newcommand{\nul}{\mathcal{N}}

% 文档区
\begin{document}

% 标题
\maketitle

\subsection*{Problem} 
Let $f$ be holomorphic on a neighbourhood of $\overline{D}(0,1)$. Assume that the restriction of $f$ to $\overline{D}(0,1)$ is one-to-one and $f'$ is nowhere zero on $\overline{D}(0,1)$. Prove that in fact $f$ is one-to-one on a neigbourhood of $\overline{D}(0,1)$.
\vspace{0.5em}\\
\textbf{Sol.} \par
Let $D_n = D(0,1+n^{-1})$ and if the conclusion is not correct, then we can always find $x_n,y_n\in D(0,1+n^{-1})$ such that $f(x_n) = f(y_n)$, and since $x_n,y_n$ is in $\overline{D}(0,2)$. Then there exists $x,y$ such that $x_n\to x, y_n\to y$ with $|x|,|y|\leq 1$, and $f(x) - f(y) = \lim_{n\to\ infty} f(x_n) - f(y_n) = 0$, if $x\neq y$ there will be a contradiction and hence $x = y$, which is also impossible since we know $f'(x) \neq 0$ and we may assume there exists $\delta >0$ such that $Ref'(y) > \epsilon > 0$ for some $\epsilon$ whenever $|y-x| < \delta$, then for any $p,q\in D(x,\delta)$, we have
\[
Re(f(p)-f(q))  = Re (\int_{\overline{qp}} f'(\xi)d\xi) = \int_{\overline{qp}} Ref'(\xi)d\xi \geq \epsilon|p-q|
\]
which means $f$ should be one-to-one around $x$, the proof is the same when $Im(f'(x)) \neq 0$. Then we know $x_n,y_n \to x$ with $f(x_n) = f(y_n)$, which is a contradiction. Therefore, $f$ should be one-to-one on $D_n$ for some integer $n$.
\par 
\vspace{0.5em}

\addappheadtotoc


\end{document}
