%!TEX program = xelatex
\documentclass[lang=en,11pt,a4paper,citestyle =authoryear]{elegantpaper}

% 标题
\title{Homework01 - MATH 722}
\author{Boren(Wells) Guan}

% 本文档命令
\usepackage{array,url,stix}
\usepackage{subfigure}
\newcommand{\ccr}[1]{\makecell{{\color{#1}\rule{1cm}{1cm}}}}
\newcommand{\code}[1]{\lstinline{#1}}
\newcommand{\prvd}{$\hfill \qedsymbol$}
\newcommand{\Z}{\mathbb{Z}}
\newcommand{\R}{\mathbb{R}}
\newcommand{\N}{\mathbb{N}}
\newcommand{\C}{\mathbb{C}}
\newcommand{\Q}{\mathbb{Q}}
\newcommand{\M}{\mathcal{M}}
\newcommand{\B}{\mathcal{B}}
\newcommand{\X}{\mathcal{X}}
\newcommand{\Hil}{\mathcal{H}}
\newcommand{\range}{\mathcal{R}}
\newcommand{\nul}{\mathcal{N}}

% 文档区
\begin{document}

% 标题
\maketitle

\subsection*{Before Reading:}\par
To make the proof more readable, I will miss or gap some natural or not important facts or notations during my writing. If you feel it hard to see, you can refer the appendix after the proof, where I will try to explain some simple conclusions (will be marked) more clearly. In case that you misunderstand the mark, I will add the mark just after those formulas between \$ and before those between \$\$.\par
And I have to claim that the appendix is of course a part of my assignment, so the reference of it is required. Enjoy your grading!

\subsection*{Chapter.1 Ex.1} 
Write each of the following as complex numbers in the standard form $x+iy$.\par
(c) $\Big(\dfrac{i-1}{2i+6}\Big)^3$.\par
(e) $i^{4n+3}$.\par
(f) $\Big(\dfrac{1}{2}-\dfrac{\sqrt{3}}{2}i\Big)^6$
\vspace{0.5em}\\
\textbf{Sol.} \par
c. $\dfrac{11-2i}{1000}$.\par
e. $-i$.\par
f. $1$.
\par 
\vspace{0.5em}

\subsection*{Chapter.1 Ex.12} 
Compute all fifth roots of $1+i$, all cube roots of $-i$, all sixth roots of $-1$ and all square roots of $\dfrac{-\sqrt{3}}{2}+\dfrac{i}{2}$.
\vspace{0.5em}\\
\textbf{Sol.} \par
All fifth roots of $1+i$ are \[\{2^{10^{-1}} e^{i(\dfrac{\pi}{20}+\dfrac{2k\pi}{5})}\}_{1 \leq k \leq 5}\]
and all cube roots of $-i$ are
\[\{i e^{i\dfrac{2k\pi}{3}}\}_{1\leq k \leq 3}\]
and all sixth roots of $-1$ are
\[\{i e^{i\dfrac{2k\pi}{6}}\}_{1\leq k \leq 6}\]
and all square roots of $\dfrac{-\sqrt{3}}{2}+\dfrac{i}{2}$ are
\[\{e^{i(\dfrac{5\pi}{12})+k\pi}\}_{1\leq k \leq 2}\]
\prvd
\vspace{0.5em}

\subsection*{Chapter.2 Ex.4} 
Compute the following complex line integrals.\par
(b) $\int_{\gamma} \bar{z}+z^2{\bar{z}} dz$ where $\gamma$ is the unit square with a clockwise orientation.\par
(c) $\int_{\gamma} \dfrac{z}{8+z^2}dz$ where $\gamma$ is the triangle with vertices $1+0i,0+i,0-i$ and $\gamma$ is equipped with counterclockwise orienteation.\par
(d) $\int_{\gamma} \dfrac{\bar{z}}{8+z}dz$ where $\gamma$ is the rectangle with vertices $\pm 3\pm i$ with clockwise orientation.
\vspace{0.5em}\\
\textbf{Sol.} \par
b. We know here we can give an arbitrary parametrization of the curve and then
\[\int_{\gamma} \bar{z}+z^2{\bar{z}} dz = -\int_{-1}^1 4(i+t) dt = -8i \]\par
c. We know the integral is $0$ since $\dfrac{z}{8+z^2}$ is holomorphic on an open neighbourhood of the triangle domain.\par
d. Similar to (b), we have
\[
\begin{aligned}
\int_{\gamma} \dfrac{\bar{z}}{8+z} dz &= \int_{\gamma_1} \dfrac{2i-z}{8+z} dz + \int_{\gamma_2} \dfrac{6+z}{8+z} dz + \int_{\gamma_3} \dfrac{-z-2i}{8+z} dz + \int_{\gamma_4} \dfrac{z-6}{8+z} dz \\
&= [z+(-2i-8)Log(8+z)]|_{3+i}^{-3+i} + [z-2Log(8+z)]|_{-3-i}^{-3+i} \\ &+ [z+(2i-8)Log(8+z)]|_{-3-i}^{3-i} + [z-14Log(8+z)]|_{3+i}^{3-i} \\ &= 2(\ln 26+10\arctan15-\ln 122-22\arctan 111)i
\end{aligned}
\]

\vspace{0.5em}

\subsection*{Chapter.2 Ex.18} 
Compute each of the following complex line integrals.\par
(b) $\int_{\gamma} \dfrac{\xi}{(\xi+4)(\xi-1+i)} d\xi$ where $\gamma = \partial D(0,1)$ with counterclockwise orientation.\par
(f) $\int_{\gamma} \dfrac{\xi(\xi+3)}{(\xi+i)(\xi-8)}d\xi$ where $\gamma = \partial D(2+i,3)$ with clockwise orientation.
\vspace{0.5em}\\
\textbf{Sol.} \par
b. We know
\[\int_{\gamma} \dfrac{\xi}{(\xi+4)(\xi-1+i)} d\xi = 0\] since $\dfrac{\xi}{(\xi+4)(\xi-1+i)}$ is holomorphic on $D(0,1)$.\par
f. We know
\[\int_{\gamma} \dfrac{\xi(\xi+3)}{(\xi+i)(\xi-8)}d\xi = 2\pi i \dfrac{\xi(\xi+3)}{(\xi-8)}(-i) = \dfrac{-6\pi + 2\pi i}{8+i} \]
since $\tfrac{\xi(\xi+3)}{(\xi-8)}$ is holomorphic on $D(2+i,3)$.\par
\vspace{0.5em}

\subsection*{Chapter.2 Ex.21} 
Let $f$ be a continuous function on $\{z:|z|=1\}$. Define, with $\gamma$ the unite circle traversed counterclockwise,
\[F(z) = \begin{cases}f(z)\quad\text{if }|z|=1 \\ \dfrac{1}{2\pi i}\int_{\gamma}\dfrac{f(\xi)}{\xi-z}d\xi\quad\text{if }|z|<1\end{cases}\]
Is $F$ continuous on $\overline{D}(0,1)$?
\vspace{0.5em}\\
\textbf{Sol.} \par
Not really, consider $f(z) = \overline{z}$, then we know
\[
\dfrac{1}{2\pi i}\int_{\gamma}\dfrac{\bar{\xi}}{\xi-z}d\xi = \dfrac{1}{2\pi i}\int_{\gamma}\dfrac{1}{\xi(\xi-z)}d\xi = \dfrac{1}{z}\dfrac{1}{2\pi i}\int_{\gamma}(\dfrac{1}{\xi-z}-\dfrac{1}{\xi})d\xi = 0
\]
where $|z|<1$, and since $|f(z)| = 1$ on $\partial D(0,1)$ and hence the $F$ is not continuous on $\overline{D}(0,1)$ now.
\vspace{0.5em}


\subsection*{Chapter.2 Ex.38} 
Verify that
\[
\dfrac{1}{2\pi i}\int_{\gamma_1} \dfrac{d\xi}{(\xi-1)(\xi+1)} = \dfrac{1}{2\pi i}\int_{\gamma_2} \dfrac{d\xi}{(\xi-1)(\xi+1)} 
\]
where $\chi_1 = \partial D(1,1)$ equipped with counterwise orientation and $\gamma_2$ is $\partial D(-1,1)$ equipped with clockwise oritentation.
\vspace{0.5em}\\
\textbf{Sol.} \par
We have
\[
\dfrac{1}{2\pi i}\int_{\gamma_1} \dfrac{d\xi}{(\xi-1)(\xi+1)} = \dfrac{1}{\xi+1}|_{1} = \dfrac{1}{2}
\]
and
\[\dfrac{1}{2\pi i}\int_{\gamma_2} \dfrac{d\xi}{(\xi-1)(\xi+1)} = - \dfrac{1}{\xi-1}_{-1} = \dfrac{1}{2}\]
and hence the integrals equal.
\vspace{0.5em}


\addappheadtotoc

\end{document}
