\section{Preliminary}

\subsection{Manifolds}

\begin{definition}
    A topological space $M$ is locally Euclidean of dimension $n$ if for every point $p$ in $M$, there is a homeomorphism $\phi$ of a neighborhood $U$ of $p$ with an open subset of $\mathbb{R}^n$. Such a pair $(U, \phi: U \to \mathbb{R}^n)$ is called a coordinate chart or simply a chart. If $p \in U$, then we say that $(U,\phi)$ is a chart about $p$. A collection of charts $\{(U_{\alpha},\phi_{\alpha}: U_{\alpha} \to \mathbb{R}^n)\}$ is $C^{\infty}$ compatible if for every $\alpha$ and $\beta$, the transition function
    \[\phi_{\alpha}\circ \phi^{-1}_{\beta}: \phi_{\beta}(U_{\alpha}\cap U_{\beta}) \to \phi_{\alpha}(U_{\alpha}\cap U_{\beta})\]\
    is $C^{\infty}$. A collection of $C^{\infty}$ compatible charts $\{(U_{\alpha},\phi_{\alpha} : U_{\alpha} \to \mathbb{R}^n)\}$ that cover $M$ is
called a $C^{\infty}$ atlas. A $C^{\infty}$ atlas is said to be maximal if it contains every chart that is
$C^{\infty}$ compatible with all the charts in the atlas.
\end{definition}

\begin{definition}
    A topological manifold is a Hausdorff, second countable, locally Euclidean topological space. A smooth manifold is a pair consisting of a
topological manifold M and a maximal $C^{\infty}$ atlas $\{(U_{\infty},\phi_{\alpha})\}$ on $M$. 
\end{definition}

\begin{definition}
    A function $f:M\to\mathbb{R}^n$ on a manifold $M$ is said to be smooth if there is a chart $(U,\phi)$ about $p$ in the maximal atlas of $M$ such that the function \[
    f\circ \phi^{-1}: \mathbb{R}^m \supset \phi(U) \to \mathbb{R}^n
    \]
    is smooth. The function $f:M\to \mathbb{R}$ is said to be smooth on $M$ if it is smooth at every
point of $M$. Recall that an algebra over $\mathbb{R}$ is a vector space $A$ together with a bilinear
map $\mu: A\times A \to A$, called multiplication, such that under addition and multiplication, $A$ becomes a ring. Under addition, multiplication, and scalar multiplication, the set of all smooth functions $f : M \to R$ is an algebra over $\mathbb{R}$, denoted by $C^{\infty}(M)$.
\end{definition}

\begin{definition}
    A map $F: N\to M$ between two manifolds is smooth at $p\in N$ if there is a chart $(U,\phi)$ about $p$ in $N$ and a chart $(V,\psi)$ about $F(p)$ in $M$ with $V \supset F(U)$ such that the composite map 
    \[\psi\circ F \circ \phi^{-1}: \mathbb{R}^n \supset \phi(U) \to \psi(V) \subset\mathbb{R}^n\]
    is smooth at $\phi(p)$. It is smooth on $N$ if it is smooth at every point of $N$. A smooth map $F: N \to M$ is called a diffeomorphism if it has a smooth inverse, i.e., a smooth map $G: M \to N$ such that $F \circ G=\mathds{1}_M $ and $G\circ F = \mathds{1}_N$.
\end{definition}

\subsection{Tangent Vectors}

\begin{definition}
    For two $C^{\infty}$ functions $f: U \to \mathbb{R}$ and $g: V \to \mathbb{R}$ defined on neighborhoods $U$ and $V$ of $p$ to be equivalent if there is a neighborhood $W$ of $p$ contained in both $U$ and $V$ such that $f$ agrees with $g$ on $W$. The equivalence class of $f : U \to R$ is called the germ of $f$ at $p$. \par The set $C^{\infty}_p(M)$ of germs of $C^{\infty}$ real-valued functions at $p$ in $M$ is an algebra over $\mathbb{R}$.
\end{definition}

\begin{definition}
    A tangent vector (point-derivation) at a point $p$ of a manifold $M$ is a linear map $D:C^{\infty}_p(M) \to \mathbb{R}$ such that for any $f,g \in C^{\infty}_p(M)$
    \[ D(fg) = (Df)g(p) + f(p)Dg.\]
    The set of all tangent vectors at $p$ is a vector space $T_p(M)$ called the tangent space of $M$ at $p$. 
\end{definition}

\begin{definition}
    At a point $p$ in a coordinate chart $(U,\phi) = (U,x^1,\cdots,x^n)$ where $x^i = r^i \circ \phi$ is the $i$th component of $\phi$, we define the coordinate vectors $\partial/\partial x^i|_p \in T_p M$ by
    \[\left.\dfrac{\partial}{\partial x^i}\right|_p f = \left.\dfrac{\partial}{\partial r^i}\right|_{\phi(p)}f\circ \phi^{-1}\]
    for each $f\in C^{\infty}_p(M)$.
\end{definition}

\begin{proposition}
    The coordinate vectors $\partial/\partial x^i|_p$ form a basis of the tangent space $T_p M$.
\end{proposition}

\begin{definition}
    If $F:N\to\ M$ is a smooth map, then at each point $p\in N$ its differential
    \[
    F_{*,p}:T_pN\to T_{F(p)} M
    \]
    is the linear map defined by
    \[
    (F_(*,p) X_p)(h) = X_p(h\circ F)
    \]
    for $X_p \in T_pN$ and $h\in C^{\infty}_{F(p)}(M)$.
\end{definition}

\begin{proposition}
    If $F: N \to M$ abd $G: M \to P$ are $C^{\infty}$ maps, then for any $p\in N$,
    \[(G\circ F)_{*,p} = G_{*,F(p)} \circ F_{*,p}\]
\end{proposition}
\begin{proof}
    For any $X_p \in T_pN, h\in C^{\infty}_{G\circ F(p)}(M)$, we have
    \[
        \begin{aligned}
            (G\circ F)_{*,p}(X_p)(h) =& X_p(h\circ (G\circ F)) \\ =& X_p((h\circ G) \circ F) = F_{*,p}X_p(h\circ G) \\ =& (G_{*,p} \circ F_{*,p} )X_p(h)
        \end{aligned}
    \]
\end{proof}

\begin{definition}
    Let $\phi:M\to N$ be a smooth map from smooth manifold $M$ to $N$, then
    \begin{enumerate}
        \item[(a)] $\phi$ is an immersion if $d\phi_m$ is injective for each $m\in M$.
        \item[(b)] The pair $(M,\phi)$ is submanifold of $N$ if $\phi$ is an injective immersion.
        \item[(c)] $\phi$ is an imbedding if $\phi$ is an injective immesrsion which is also a homeomorphism into $\phi(M)$, that is $\phi$ is open with $\phi(M)$ equipped with the relative topology.
        \item[(d)] $\phi$ is a diffeomorphism if $\phi$ maps $M$ injectively onto $N$ and $\phi^{-1}$ is smooth.
    \end{enumerate}    
\end{definition}

\begin{definition}
    A set $f_1,\cdots,f_j$ of smooth functions defined on some neighborhood of $m$ in $M$ is called an independent set at $m$ if the differentials $df_1,\cdots,df_j$ form an independent set in $T_mM^*$.
\end{definition}

\begin{theorem}
    (Inverse Function Theorem) Let $U\subset \mathbb{R}^d$ be open, and let $f:U\to\mathbb{R}^d$ be smooth. If the Jacobian matrix is nonsingular at $p \in U$, then there exists an open set $V$ with $p\in V\subset U$ such that $f|V$ maps $V$ injectively onto the open set $f(V)$ and $(f|V)^{-1}$ is smooth.
\end{theorem}

\begin{corollary}
    Assume that $\phi: M\to N$ is smooth, that $m\in M$, and $d\phi:T_mM \to T_{\phi(m)}N$ is an isomorphism. Then there is a neighbourhood $U$ of $m$ such that $\phi: U \to \phi(U)$ is a diffeomorphism onto the open set $\phi(U)$ in $N$.
\end{corollary}
\begin{proof}
    Since $d\phi$ is an isomorphism, we know dim $M =$ dim $N$, Consider $(U,\psi)$ a chart containing $m$ and $(V,\tau)$ a chart containing $\phi(m)$, then we know $\psi:U\to\psi(U), \tau:V\to\tau(V)$ are both diffeomorphisms and hence $(\tau \circ \phi \circ \psi^{-1})_{*,m}:T_{\psi(m)} \psi(U) \to T_{\tau(\phi(m))}\tau(V)$ is an isomorphism and hence the Jacobian matrix is non-singular, so there is an open set $W\subset \psi(U)$ such that $\tau \circ \phi \circ \psi^{-1}: W \to \tau \circ \phi \circ \psi^{-1}(W)$ is a diffeomorphism and hence induce a map $\psi^{-1}(W)\to \tau^{-1}(\tau \circ \phi \circ \psi^{-1}(W)) = \phi(\psi^{-1}(W))$ is a diffeomorphism.
\end{proof}

\begin{corollary}
    Suppose that dim $M=d$ and that $f_1,\cdots,f_d$ is an independent set of functions at $m_0 \in M$. Then the functions $f_1,\cdots,f_d$ form a coordinate system on a neighborhood of $m_0$. 
\end{corollary}


\subsection{Vector Fields}

\begin{definition}
    A vector field $X$ on a manifold $M$ is the assignment of a tangent vector $X_p \in T_p M$ to each point $p$, then we can have
    \[
        X_p = a^i(p)\left.\dfrac{\partial}{\partial x^i}\right|_p\quad\text{with }a^i(p)\in\mathbb{R}
    \]
    and $X$ is said to be smooth if $M$ has a smooth atlas such that on each chart $(U,x^i)$ $a^i$ are smooth. We denote the set of all $C^{\infty}$ vector fields on $M$ by $\mathscr{X}(M)$.\par
    A frame of vector fields on an open set $U\subset M$ is a collection of vector fields $X_1,\cdots, X_n$ on $U$ such that at each point $p \in U$, the vectors $(X_i)_p$ form a basis for $T_p M$.
\end{definition}

\begin{proposition}
    For some $f\in C^{\infty}(M)$, we have the induced function on $M$ by
    \[
    (Xf)(p) = X_pf
    \]
    which is still in $C^{\infty}(M)$.
\end{proposition}
\begin{proof}
    For a chart $(U,x^i)$, we have
    \[
    (Xf)(p) = a^i(p) \partial f/\partial x_i|_p
    \]
    which is smooth on $U$.
\end{proof}
\
\begin{definition}
    The Lie bracket of two vector fields $X,Y \in \mathscr{X}(M)$ is the vector field $[X,Y]$ defined by
    \[[X,Y]_p f = X_p(Yf) - Y_p(Xf)\quad\text{for }p\in M\text{ and }f\in C^{\infty}_p(M)\]
    which is still in $\mathscr{X}(M)$.
\end{definition}

\subsection{Differential Forms}

\section{Vector Bundles}

\subsection{Definitions}

\begin{definition}[Vector Bundle]
A $C^{\infty}$ surjection $\pi : E \to M$ is a $C^{\infty}$ \emph{vector bundle of rank} $r$ if
\begin{enumerate}
    \item For every $p \in M$, the set $E_p := \pi^{-1}(p)$ is a real vector space of dimension $r$
    \item every point $p \in M$ has an open neighborhood $U$ such that there is a fiber-preserving diffeomorphism
    \[
    \varphi_U : \pi^{-1}(U) \to U \times \mathbb{R}^r
    \]
    that restricts to a linear isomorphism $E_p \to \{p\} \times \mathbb{R}^r$ on each fiber
\end{enumerate}
The space $E$ is called the \emph{total space}, the space $M$ the \emph{base space}, and the space $E_p$ the \emph{fiber above $p$} of the vector bundle. We often say that $E$ is a vector bundle over $M$. A vector bundle of rank 1 is also called a \emph{line bundle}.
\end{definition}

\begin{definition}[Trivialization] 
    We call the open set $U$ in (ii) a \emph{trivializing open subset} for the vector bundle, and $\varphi_U$ a \emph{trivialization} of $\pi^{-1}(U)$. A \emph{trivializing open cover} for the vector bundle is an open cover $\{U_\alpha\}$ of $M$ consisting of trivializing open sets $U_\alpha$ together with trivializations $\varphi_\alpha : \pi^{-1}(U_\alpha) \to U_\alpha \times \mathbb{R}^r$.
\end{definition}

\begin{example}[Product bundle]
If $V$ is a vector space of dimension $r$, then the projection $\pi : M \times V \to M$ is a vector bundle of rank $r$, called a \textbf{product bundle}. Via the projection $\pi : S^1 \times \mathbb{R} \to S^1$, the cylinder $S^1 \times \mathbb{R}$ is a product bundle over the circle $S^1$.
\end{example}

\begin{example}[M\"obius strip]
The open M\"obius strip is the quotient of $[0,1] \times \mathbb{R}$ by the identification
\[
(0,t) \sim (1,-t).
\]
It is a vector bundle of rank 1 over the circle (Figure 7.1).

\end{example}

\begin{example}[Restriction of a vector bundle]
Let $S$ be a submanifold of a manifold $M$, and $\pi : E \to M$ a $C^{\infty}$ vector bundle. Then $\pi_S : \pi^{-1}(S) \to S$ is also a vector bundle, called the \textbf{restriction} of $E$ to $S$, written $E|_S := \pi^{-1}(S)$ (Figure 7.2).


\begin{definition}
Let $\pi_E : E \to M$ and $\pi_F : F \to N$ be $C^{\infty}$ vector bundles. A $C^{\infty}$ \textbf{bundle map} from $E$ to $F$ is a pair of $C^{\infty}$ maps $(\varphi : E \to F, \underline{\varphi} : M \to N)$ such that
\begin{enumerate}[label=(\roman*)]
    \item the diagram
    \[
    \begin{tikzcd}
    E \arrow[r, "\varphi"] \arrow[d, "\pi_E"] & F \arrow[d, "\pi_F"] \\
    M \arrow[r, "\underline{\varphi}"] & N
    \end{tikzcd}
    \]
    commutes,
    \item $\varphi$ restricts to a linear map $\varphi_p : E_p \to F_{\underline{\varphi}(p)}$ of fibers for each $p \in M$.
\end{enumerate}
Abusing language, we often call the map $\varphi : E \to F$ alone the bundle map.
\end{definition}

An important special case of a bundle map occurs when $E$ and $F$ are vector bundles over the same manifold $M$ and the base map $\underline{\varphi}$ is the identity map $1_M$. In this case we call the bundle map $(\varphi : E \to F, 1_M)$ a \textbf{bundle map over $M$}. If there is a bundle map $\psi : F \to E$ over $M$ such that $\psi \circ \varphi = 1_E$ and $\varphi \circ \psi = 1_F$, then $\varphi$ is called a \textbf{bundle isomorphism over $M$}, and the vector bundles $E$ and $F$ are said to be \textbf{isomorphic over $M$}.

\begin{definition}
A vector bundle $\pi : E \to M$ is said to be \textbf{trivial} if it is isomorphic to a product bundle $M \times \mathbb{R}^r \to M$ over $M$.
\end{definition}

\begin{example}[Tangent bundle]
For any manifold $M$, define $TM$ to be the set of all tangent vectors of $M$:
\[
TM = \{(p,v) \mid p \in M, v \in T_p M\}.
\]
If $U$ is a coordinate open subset of $M$, then $TU$ is bijective with the product bundle $U \times \mathbb{R}^n$. We give $TM$ the topology generated by $TU$ as $U$ runs over all coordinate open subsets of $M$. In this way $TM$ can be given a manifold structure so that $TM \to M$ becomes a vector bundle. It is called the \textbf{tangent bundle} of $M$ (for details, see [21, Section 12]).
\end{example}

\begin{example}
If $f : M \to N$ is a $C^{\infty}$ map of manifolds, then its differential gives rise to a bundle map $f_* : TM \to TN$ defined by
\[
f_*(p,v) = (f(p), f_{*,p}(v)).
\]
\end{example}

\subsection{The Vector Space of Sections}

A \textbf{section} of a vector bundle $\pi : E \to M$ over an open set $U$ is a function $s : U \to E$ such that $\pi \circ s = 1_U$, the identity map on $U$. For each $p \in U$, the section $s$ picks out one element of the fiber $E_p$. The set of all $C^{\infty}$ sections of $E$ over $U$ is denoted by $\Gamma(U,E)$. If $U$ is the manifold $M$, we also write $\Gamma(E)$ instead of $\Gamma(M,E)$.

The set $\Gamma(U,E)$ of $C^{\infty}$ sections of $E$ over $U$ is clearly a vector space over $\mathbb{R}$. It is in fact a module over the ring $C^{\infty}(U)$ of $C^{\infty}$ functions, for if $f$ is a $C^{\infty}$ function over $U$ and $s$ is a $C^{\infty}$ section of $E$ over $U$, then the definition $(fs)(p) := f(p)s(p) \in E_p$, $p \in U$, makes $fs$ into a $C^{\infty}$ section of $E$ over $U$.

\begin{example}[Sections of a product line bundle]
A section $s$ of the product bundle $M \times \mathbb{R} \to M$ is a map $s(p) = (p, f(p))$. So there is a one-to-one correspondence
\[
\{\text{sections of } M \times \mathbb{R} \to M\} \longleftrightarrow \{\text{functions } f : M \to \mathbb{R}\}.
\]
In particular, the space of $C^{\infty}$ sections of the product line bundle $M \times \mathbb{R} \to M$ may be identified with $C^{\infty}(M)$.
\end{example}

\begin{example}[Sections of the tangent bundle]
A vector field on a manifold $M$ assigns to each point $p \in M$ a tangent vector $X_p \in T_p M$. Therefore, it is precisely a section of the tangent bundle $TM$. Thus, $\mathfrak{X}(M) = \Gamma(TM)$.
\end{example}

\begin{example}[Vector fields along a submanifold]
If $M$ is a regular submanifold of $\mathbb{R}^n$, then a $C^{\infty}$ vector field along $M$ is precisely a section of the restriction $T\mathbb{R}^n|_M$ of $T\mathbb{R}^n$ to $M$. This explains our earlier notation $\Gamma(T\mathbb{R}^3|_M)$ for the space of $C^{\infty}$ vector fields along $M$ in $\mathbb{R}^3$.
\end{example}

\begin{definition}
A bundle map $\varphi : E \to F$ over a manifold $M$ (meaning that the base map is the identity $1_M$) induces a map on the space of sections:
\[
\varphi_\# : \Gamma(E) \to \Gamma(F), \quad \varphi_\#(s) = \varphi \circ s.
\]
This induced map $\varphi_\#$ is $F$-linear because if $f \in F$, then
\begin{align*}
(\varphi_\#(fs))(p) &= (\varphi \circ (fs))(p) = \varphi(f(p)s(p)) \\
&= f(p)\varphi(s(p)) \quad (\text{because } \varphi \text{ is } \mathbb{R}\text{-linear on each fiber}) \\
&= f(\varphi_\#(s))(p).
\end{align*}
\end{definition}

Our goal in the rest of this chapter is to prove that conversely, every $F$-linear map $\alpha : \Gamma(E) \to \Gamma(F)$ comes from a bundle map $\varphi : E \to F$, i.e., $\alpha = \varphi_\#$.

\subsection{Extending a Local Section to a Global Section}

Consider the interval $(-\pi/2, \pi/2)$ as an open subset of the real line $\mathbb{R}$. The example of the tangent function $\tan : (-\pi/2, \pi/2) \to \mathbb{R}$ shows that it may not be possible to extend the domain of a $C^{\infty}$ function $f : U \to \mathbb{R}$ from an open subset $U \subset M$ to the manifold $M$. However, given a point $p \in U$, it is always possible to find a $C^{\infty}$ global function $\bar{f} : M \to \mathbb{R}$ that agrees with $f$ on some neighborhood of $p$. More generally, this is also true for sections of a vector bundle.

\begin{proposition}
Let $E \to M$ be a $C^{\infty}$ vector bundle, $s$ a $C^{\infty}$ section of $E$ over some open set $U$ in $M$, and $p$ a point in $U$. Then there exists a $C^{\infty}$ global section $\bar{s} \in \Gamma(M,E)$ that agrees with $s$ over some neighborhood of $p$.
\end{proposition}

\begin{proof}
Choose a $C^{\infty}$ bump function $f$ on $M$ such that $f \equiv 1$ on a neighborhood $W$ of $p$ contained in $U$ and $\operatorname{supp} f \subset U$ (Figure 7.3). Define $\bar{s} : M \to E$ by
\[
\bar{s}(q) = 
\begin{cases}
f(q)s(q) & \text{for } q \in U, \\
0 & \text{for } q \notin U.
\end{cases}
\]
On $U$ the section $\bar{s}$ is clearly $C^{\infty}$ for it is the product of a $C^{\infty}$ function $f$ and a $C^{\infty}$ section $s$. If $p \notin U$, then $p \notin \operatorname{supp} f$. Since $\operatorname{supp} f$ is a closed set, there is a neighborhood $V$ of $p$ contained in its complement $M \setminus \operatorname{supp} f$. On $V$ the section $\bar{s}$ is identically zero. Hence, $\bar{s}$ is $C^{\infty}$ at $p$. This proves that $\bar{s}$ is $C^{\infty}$ on $M$. On $W$, since $f \equiv 1$, the section $\bar{s}$ agrees with $s$.
\end{proof}


\subsection{Local Operators}

In this section, $E$ and $F$ are $C^{\infty}$ vector bundles over a manifold $M$, and $F$ is the ring $C^{\infty}(M)$ of $C^{\infty}$ functions on $M$.

\begin{definition}
Let $E$ and $F$ be vector bundles over a manifold $M$. An $\mathbb{R}$-linear map $\alpha : \Gamma(E) \to \Gamma(F)$ is a \textbf{local operator} if whenever a section $s \in \Gamma(E)$ vanishes on an open set $U$ in $M$, then $\alpha(s) \in \Gamma(F)$ also vanishes on $U$. It is a \textbf{point operator} if whenever a section $s \in \Gamma(E)$ vanishes at a point $p$ in $M$, then $\alpha(s) \in \Gamma(F)$ also vanishes at $p$.
\end{definition}

\begin{example}
By Example 7.9, the vector space $C^{\infty}(\mathbb{R})$ of $C^{\infty}$ functions on $\mathbb{R}$ may be identified with the vector space $\Gamma(\mathbb{R} \times \mathbb{R})$ of $C^{\infty}$ sections of the product line bundle over $\mathbb{R}$. The derivative $\frac{d}{dt} : C^{\infty}(\mathbb{R}) \to C^{\infty}(\mathbb{R})$ is a local operator since if $f(t) \equiv 0$ on $U$, then $f'(t) \equiv 0$ on $U$. However, $d/dt$ is not a point operator.
\end{example}

\begin{example}
Let $\Omega^k(M)$ denote the vector space of $C^{\infty}$ $k$-forms on a manifold $M$. Then the exterior derivative $d : \Omega^k(M) \to \Omega^{k+1}(M)$ is a local operator.
\end{example}

\begin{proposition}
Let $E$ and $F$ be $C^{\infty}$ vector bundles over a manifold $M$, and $F = C^{\infty}(M)$. If a map $\alpha : \Gamma(E) \to \Gamma(F)$ is $F$-linear, then it is a local operator.
\end{proposition}

\begin{proof}
Suppose the section $s \in \Gamma(E)$ vanishes on the open set $U$. Let $p \in U$ and let $f$ be a $C^{\infty}$ bump function such that $f(p) = 1$ and $\operatorname{supp} f \subset U$ (Figure 7.3). Then $fs \in \Gamma(E)$ and $fs \equiv 0$ on $M$ (Figure 7.4). So $\alpha(fs) \equiv 0$. By $F$-linearity,
\[
0 = \alpha(fs) = f \alpha(s).
\]
Evaluating at $p$ gives $\alpha(s)(p) = 0$. Since $p$ is an arbitrary point of $U$, $\alpha(s) \equiv 0$ on $U$.
\end{proof}


\begin{example}
On a $C^{\infty}$ manifold $M$, a derivation $D : C^{\infty}(M) \to C^{\infty}(M)$ is $\mathbb{R}$-linear, but not $F$-linear since by the Leibniz rule,
\[
D(fg) = (Df)g + fDg, \quad \text{for } f,g \in F.
\]
However, by Problem 7.1, $D$ is a local operator.
\end{example}

\begin{example}
Fix a $C^{\infty}$ vector field $X \in \mathfrak{X}(M)$. Then a connection $\nabla$ on $M$ induces a map
\[
\nabla_X : \mathfrak{X}(M) \to \mathfrak{X}(M)
\]
that satisfies the Leibniz rule. By Problem 7.2, $\nabla_X$ is a local operator.
\end{example}

\subsection{Restriction of a Local Operator to an Open Subset}

A continuous global section of a vector bundle can always be restricted to an open subset, but in general a section over an open subset cannot be extended to a continuous global section. For example, the tangent function defined on the open interval $(-\pi/2, \pi/2)$ cannot be extended to a continuous function on the real line. Nonetheless, a local operator, which is defined on global sections of a vector bundle, can always be restricted to an open subset.

\begin{theorem}
Let $E$ and $F$ be vector bundles over a manifold $M$. If $\alpha : \Gamma(E) \to \Gamma(F)$ is a local operator, then for each open subset $U$ of $M$ there is a unique linear map, called the \textbf{restriction of $\alpha$ to $U$},
\[
\alpha_U : \Gamma(U,E) \to \Gamma(U,F)
\]
such that for any global section $t \in \Gamma(E)$,
\[
\alpha_U(t|_U) = \alpha(t)|_U.
\]
\end{theorem}

\begin{proof}
Let $s \in \Gamma(U,E)$ and $p \in U$. By Proposition 7.13, there exists a global section $\bar{s}$ of $E$ that agrees with $s$ in some neighborhood $W$ of $p$ in $U$. We define $\alpha_U(s)(p)$ using (7.1):
\[
\alpha_U(s)(p) = \alpha(\bar{s})(p).
\]
Suppose $\tilde{s} \in \Gamma(E)$ is another global section that agrees with $s$ in the neighborhood $W$ of $p$. Then $\bar{s} = \tilde{s}$ in $W$. Since $\alpha$ is a local operator, $\alpha(\bar{s}) = \alpha(\tilde{s})$ on $W$. Hence, $\alpha(\bar{s})(p) = \alpha(\tilde{s})(p)$. This shows that $\alpha_U(s)(p)$ is independent of the choice of $\bar{s}$, so $\alpha_U$ is well defined and unique. Fix $p \in U$. If $s \in \Gamma(U,E)$ and $\bar{s} \in \Gamma(M,E)$ agree on a neighborhood $W$ of $p$, then $\alpha_U(s) = \alpha(\bar{s})$ on $W$. Hence, $\alpha_U(s)$ is $C^{\infty}$ as a section of $F$.

If $t \in \Gamma(M,E)$ is a global section, then it is a global extension of its restriction $t|_U$. Hence,
\[
\alpha_U(t|_U)(p) = \alpha(t)(p) \quad \text{for all } p \in U.
\]
This proves that $\alpha_U(t|_U) = \alpha(t)|_U$.
\end{proof}

\begin{proposition}
Let $E$ and $F$ be $C^{\infty}$ vector bundles over a manifold $M$, let $U$ be an open subset of $M$, and let $F(U) = C^{\infty}(U)$, the ring of $C^{\infty}$ functions on $U$. If $\alpha : \Gamma(E) \to \Gamma(F)$ is $F$-linear, then the restriction $\alpha_U : \Gamma(U,E) \to \Gamma(U,F)$ is $F(U)$-linear.
\end{proposition}

\begin{proof}
Let $s \in \Gamma(U,E)$ and $f \in F(U)$. Fix $p \in U$ and let $\bar{s}$ and $\bar{f}$ be global extensions of $s$ and $f$ that agree with $s$ and $f$, respectively, on a neighborhood of $p$ (Proposition 7.13). Then
\begin{align*}
\alpha_U(fs)(p) &= \alpha(\bar{f}\bar{s})(p) \quad (\text{definition of } \alpha_U) \\
&= \bar{f}(p)\alpha(\bar{s})(p) \quad (F\text{-linearity of } \alpha) \\
&= f(p)\alpha_U(s)(p).
\end{align*}
Since $p$ is an arbitrary point of $U$,
\[
\alpha_U(fs) = f\alpha_U(s),
\]
proving that $\alpha_U$ is $F(U)$-linear.
\end{proof}

\subsection{Frames}

A \textbf{frame} for a vector bundle $E$ of rank $r$ over an open set $U$ is a collection of sections $e_1, \dots, e_r$ of $E$ over $U$ such that at each point $p \in U$, the elements $e_1(p), \dots, e_r(p)$ form a basis for the fiber $E_p$.

\begin{proposition}
A $C^{\infty}$ vector bundle $\pi : E \to M$ is trivial if and only if it has a $C^{\infty}$ frame.
\end{proposition}

\begin{proof}
Suppose $E$ is trivial, with $C^{\infty}$ trivialization $\varphi : E \to M \times \mathbb{R}^r$. Let $v_1, \dots, v_r$ be the standard basis for $\mathbb{R}^r$. Then the elements $(p, v_i)$, $i = 1, \dots, r$, form a basis for $\{p\} \times \mathbb{R}^r$ for each $p \in M$, and so the sections of $E$
\[
e_i(p) = \varphi^{-1}(p, v_i), \quad i = 1, \dots, r,
\]
provide a basis for $E_p$ at each point $p \in M.

Conversely, suppose $e_1, \dots, e_r$ is a frame for $E \to M$. Then every point $e \in E$ is a linear combination $e = \sum a_i e_i$. The map
\[
\varphi(e) = (\pi(e), a_1, \dots, a_r) : E \to M \times \mathbb{R}^r
\]
is a bundle map with inverse
\[
\psi : M \times \mathbb{R}^r \to E, \quad \psi(p, a_1, \dots, a_r) = \sum a_i(p) e_i(p).
\]
\end{proof}

It follows from this proposition that over any trivializing open set $U$ of a vector bundle $E$, there is always a frame.

\subsection{$F$-Linearity and Bundle Maps}

Throughout this subsection, $E$ and $F$ are $C^{\infty}$ vector bundles over a manifold $M$, and $F = C^{\infty}(M)$ is the ring of $C^{\infty}$ real-valued functions on $M$. We will show that an $F$-linear map $\alpha : \Gamma(E) \to \Gamma(F)$ can be defined pointwise and therefore corresponds uniquely to a bundle map $E \to F$.

\begin{lemma}
An $F$-linear map $\alpha : \Gamma(E) \to \Gamma(F)$ is a point operator.
\end{lemma}

\begin{proof}
We need to show that if $s \in \Gamma(E)$ vanishes at a point $p$ in $M$, then $\alpha(s) \in \Gamma(F)$ also vanishes at $p$. Let $U$ be an open neighborhood of $p$ over which $E$ is trivial. Thus, over $U$ it is possible to find a frame $e_1, \dots, e_r$ for $E$. We write
\[
s|_U = \sum a_i e_i, \quad a_i \in C^{\infty}(U) = F(U).
\]
Because $s$ vanishes at $p$, all $a_i(p) = 0$. Since $\alpha$ is $F$-linear, it is a local operator (Proposition 7.17) and by Theorem 7.20 its restriction $\alpha_U : \Gamma(U,E) \to \Gamma(U,F)$ is defined. Then
\begin{align*}
\alpha(s)(p) &= \alpha_U(s|_U)(p) \quad (\text{Theorem 7.20}) \\
&= \alpha_U\left(\sum a_i e_i\right)(p) \\
&= \sum a_i \alpha_U(e_i)(p) \quad (\alpha_U \text{ is } F(U)\text{-linear (Proposition 7.21)}) \\
&= \sum a_i(p) \alpha_U(e_i)(p) = 0.
\end{align*}
\end{proof}

\begin{lemma}
Let $E$ and $F$ be $C^{\infty}$ vector bundles over a manifold $M$. A fiber-preserving map $\varphi : E \to F$ that is linear on each fiber is $C^{\infty}$ if and only if $\varphi_\#$ takes $C^{\infty}$ sections of $E$ to $C^{\infty}$ sections of $F$.
\end{lemma}

\begin{proof}
($\Rightarrow$) If $\varphi$ is $C^{\infty}$, then $\varphi_\#(s) = \varphi \circ s$ clearly takes a $C^{\infty}$ section $s$ of $E$ to a $C^{\infty}$ section of $F$.

($\Leftarrow$) Fix $p \in M$ and let $(U, x^1, \dots, x^n)$ be a chart about $p$ over which $E$ and $F$ are both trivial. Let $e_1, \dots, e_r \in \Gamma(E)$ be a frame for $E$ over $U$. Likewise, let $f_1, \dots, f_m \in \Gamma(F)$ be a frame for $F$ over $U$. A point of $E|_U$ can be written as a unique linear combination $\sum a_j e_j$. Suppose
\[
\varphi \circ e_j = \sum_i b_{ij} f_i.
\]
In this expression the $b_{ij}$'s are $C^{\infty}$ functions on $U$, because by hypothesis $\varphi \circ e_j = \varphi_\#(e_j)$ is a $C^{\infty}$ section of $F$. Then
\[
\varphi \circ \left(\sum_j a_j e_j\right) = \sum_{i,j} a_j b_{ij} f_i.
\]
One can take local coordinates on $E|_U$ to be $(x^1, \dots, x^n, a_1, \dots, a_r)$. In terms of these local coordinates,
\[
\varphi(x^1, \dots, x^n, a_1, \dots, a_r) = \left(x^1, \dots, x^n, \sum_j a_j b_{1j}, \dots, \sum_j a_j b_{mj}\right)
\]
which is a $C^{\infty}$ map.
\end{proof}

\begin{proposition}
If $\alpha : \Gamma(E) \to \Gamma(F)$ is $F$-linear, then for each $p \in M$, there is a unique linear map $\varphi_p : E_p \to F_p$ such that for all $s \in \Gamma(E)$,
\[
\varphi_p(s(p)) = \alpha(s)(p).
\]
\end{proposition}

\begin{proof}
Given $e \in E_p$, to define $\varphi_p(e)$, choose any section $s \in \Gamma(E)$ such that $s(p) = e$ (Problem 7.4) and define
\[
\varphi_p(e) = \alpha(s)(p) \in F_p.
\]
This definition is independent of the choice of the section $s$, because if $s'$ is another section of $E$ with $s'(p) = e$, then $(s - s')(p) = 0$ and so by Lemma 7.23, we have $\alpha(s - s')(p) = 0$, i.e.,
\[
\alpha(s)(p) = \alpha(s')(p).
\]
Let us show that $\varphi_p : E_p \to F_p$ is linear. Suppose $e_1, e_2 \in E_p$ and $a_1, a_2 \in \mathbb{R}$. Let $s_1, s_2$ be global sections of $E$ such that $s_i(p) = e_i$. Then $(a_1 s_1 + a_2 s_2)(p) = a_1 e_1 + a_2 e_2$, so
\begin{align*}
\varphi_p(a_1 e_1 + a_2 e_2) &= \alpha(a_1 s_1 + a_2 s_2)(p) \\
&= a_1 \alpha(s_1)(p) + a_2 \alpha(s_2)(p) \\
&= a_1 \varphi_p(e_1) + a_2 \varphi_p(e_2).
\end{align*}
\end{proof}

\begin{theorem}
There is a one-to-one correspondence
\[
\{\text{bundle maps } \varphi : E \to F\} \longleftrightarrow \{\text{$F$-linear maps } \alpha : \Gamma(E) \to \Gamma(F)\},
\]
given by $\varphi \mapsto \varphi_\#$.
\end{theorem}

\begin{proof}
We first show surjectivity. Suppose $\alpha : \Gamma(E) \to \Gamma(F)$ is $F$-linear. By the preceding proposition, for each $p \in M$ there is a linear map $\varphi_p : E_p \to F_p$ such that for any $s \in \Gamma(E)$,
\[
\varphi_p(s(p)) = \alpha(s)(p).
\]
Define $\varphi : E \to F$ by $\varphi(e) = \varphi_p(e)$ if $e \in E_p$. For any $s \in \Gamma(E)$ and for every $p \in M$,
\[
(\varphi_\#(s))(p) = \varphi(s(p)) = \alpha(s)(p),
\]
which shows that $\alpha = \varphi_\#$. Since $\varphi_\#$ takes $C^{\infty}$ sections of $E$ to $C^{\infty}$ sections of $F$, by Lemma 7.24 the map $\varphi : E \to F$ is $C^{\infty}$. Thus, $\varphi$ is a bundle map.

Next we prove the injectivity of the correspondence. Suppose $\varphi, \psi : E \to F$ are two bundle maps such that $\varphi_\# = \psi_\# : \Gamma(E) \to \Gamma(F)$. For any $e \in E_p$, choose a section $s \in \Gamma(E)$ such that $s(p) = e$. Then
\[
\varphi(e) = \varphi(s(p)) = (\varphi_\#(s))(p) = (\psi_\#(s))(p) = (\psi \circ s)(p) = \psi(e).
\]
Hence, $\varphi = \psi$.
\end{proof}

\begin{corollary}
An $F$-linear map $\omega : \mathfrak{X}(M) \to C^{\infty}(M)$ is a $C^{\infty}$ 1-form on $M$.
\end{corollary}

\begin{proof}
By Proposition 7.25, one can define for each $p \in M$ a linear map $\omega_p : T_p M \to \mathbb{R}$ such that for all $X \in \mathfrak{X}(M)$,
\[
\omega_p(X_p) = \omega(X)(p).
\]
This shows that $\omega$ is a 1-form on $M$. For every $C^{\infty}$ vector field $X$ on $M$, $\omega(X)$ is a $C^{\infty}$ function on $M$. This shows that as a 1-form, $\omega$ is $C^{\infty}$.
\end{proof}

\subsection{Multilinear Maps over Smooth Functions}

By Proposition 7.25, if $\alpha : \Gamma(E) \to \Gamma(F)$ is an $F$-linear map of sections of vector bundles over $M$, then at each $p \in M$, it is possible to define a linear map $\varphi_p : E_p \to F_p$ such that for any $s \in \Gamma(E)$,
\[
\varphi_p(s(p)) = \alpha(s)(p).
\]
This can be generalized to $F$-multilinear maps.

\begin{proposition}
Let $E, E', F$ be vector bundles over a manifold $M$. If $\alpha : \Gamma(E) \times \Gamma(E') \to \Gamma(F)$ is $F$-bilinear, then for each $p \in M$ there is a unique $\mathbb{R}$-bilinear map $\varphi_p : E_p \times E'_p \to F_p$ such that for all $s \in \Gamma(E)$ and $s' \in \Gamma(E')$,
\[
\varphi_p(s(p), s'(p)) = (\alpha(s, s'))(p).
\]
\end{proposition}

Since the proof is similar to that of Proposition 7.25, we leave it as an exercise.

Of course, Proposition 7.28 generalizes to $F$-linear maps with $k$ arguments. Just as in Corollary 7.27, we conclude that if an alternating map
\[
\omega : \mathfrak{X}(M) \times \cdots \times \mathfrak{X}(M) \ (k \text{ times}) \to C^{\infty}(M)
\]
is $F$-linear in each argument, then $\omega$ induces a $k$-form $\tilde{\omega}$ on $M$ such that for $X_1, \dots, X_k \in \mathfrak{X}(M)$,
\[
\tilde{\omega}_p(X_{1,p}, \dots, X_{k,p}) = (\omega(X_1, \dots, X_k))(p).
\]
It is customary to write the $k$-form $\tilde{\omega}$ as $\omega$.