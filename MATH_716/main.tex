\documentclass{article}

\usepackage{amsmath, amsthm, amssymb, amsfonts}
\usepackage{thmtools}
\usepackage{graphicx}
\usepackage{setspace}
\usepackage{geometry}
\usepackage{float}
\usepackage{hyperref}
\usepackage[utf8]{inputenc}
\usepackage[english]{babel}
\usepackage{framed}
\usepackage[dvipsnames]{xcolor}
\usepackage{tcolorbox}
\usepackage{tikz}
\usepackage{tikz-cd}

\colorlet{LightGray}{White!90!Periwinkle}
\colorlet{LightOrange}{Orange!15}
\colorlet{LightGreen}{Green!15}

\newcommand{\HRule}[1]{\rule{\linewidth}{#1}}
\newcommand{\Pf}[1]{$Proof.$\par}

\declaretheoremstyle[name=Definiton,]{thmsty}
\declaretheorem[style=thmsty,numberwithin=subsection]{definition}

\declaretheoremstyle[name=Theorem,]{thmsty}
\declaretheorem[style=thmsty,numberwithin=subsection]{theorem}


\declaretheoremstyle[name=Lemma,]{thmsty}
\declaretheorem[style=thmsty,numberlike=theorem]{lemma}

\declaretheoremstyle[name=Corollary,]{thmsty}
\declaretheorem[style=thmsty,numberlike=theorem]{corollary}

\declaretheoremstyle[name=Proposition,]{prosty}
\declaretheorem[style=prosty,numberlike=theorem]{proposition}

\declaretheoremstyle[name=Principle,]{prcpsty}
\declaretheorem[style=prcpsty,numberlike=theorem]{principle}

\declaretheoremstyle[name=Example,]{prcpsty}
\declaretheorem[style=prcpsty,numberwithin=subsection]{example}

\declaretheoremstyle[name=Ex,]{prcpsty}
\declaretheorem[style=prcpsty,numberwithin=section]{exercise}


\setstretch{1.2}
\geometry{
    textheight=9in,
    textwidth=5.5in,
    top=1in,
    headheight=12pt,
    headsep=25pt,
    footskip=30pt
}

% ------------------------------------------------------------------------------

\begin{document}

% ------------------------------------------------------------------------------
% Cover Page and ToC
% ------------------------------------------------------------------------------

\title{ \normalsize \textsc{}
		\\ [2.0cm]
		\HRule{1.5pt} \\
		\LARGE \textbf{\uppercase{Notes for ODE}
		\HRule{2.0pt} \\ [0.6cm] \LARGE{Based on lectures provided by Chanwoo Kim on MATH 716 2025 SPRING} \vspace*{10\baselineskip}}
		}
\date{}
\author{\textbf{Author} \\ 
		Wells Guan \\
		 \\
		}

\maketitle
\newpage

\tableofcontents
\newpage

% ------------------------------------------------------------------------------

\section{Introduction}

\subsection{Differential equations}

\begin{definition}(General ODE)\par
    Generally, we consider
    \[y:\mathbb{R}\to N\]
    where $t\mapsto y(t)$, $N$ is a vector space such as $\mathbb{R}^d$. Then the general form ODE is
    \[F(t,y, \dfrac{dy}{dt},\cdots, \dfrac{d^n y}{dt^n}) = 0\]
    Once we can solve it explicitly by
    \[
    \dfrac{d^n y}{dt^n} = G(t,y,y',\cdots,y^{(n-1)})
    \]
    and we may convert it into a system of first-order equation by
    \[
    \begin{cases}
        \dfrac{dx_i}{dt} = x_{i+1}\quad i = 1,\cdots,n-1 \\

        \dfrac{dx_n}{dt} = G(t,x_1,\cdots,x_{n-1})
    \end{cases}
    \]
\end{definition}

\begin{definition}(General ODE systems)\par
    \[\dfrac{dx_i}{dt} = f_i(t,x_1,\cdots,x_n) i = 1, \cdots,n\]
    or
    \[\nabla x = f(t,x)\]
    where $x:\mathbb{R} \to\mathbb{R}^n$ and $f:\mathbb{R}\times\mathbb{R}^n \to \mathbb{R}^n$
    if $f(t,x) = f(x)$, we call it to be a \textbf{autonomous system}, else a \textbf{nonautonomous system}.\par
    An \textbf{initial problem} is
    \[
    \begin{cases}
        \nabla x = f(t,x) \\
        x(t_0) = x_0
    \end{cases}
    \]
    and the \text{general solution}
    \[
    \begin{cases}
        \nabla x(t;c) = f(t,x(t;c)) \\
        x(0;c) = c
    \end{cases}
    \]
\end{definition}

\begin{example}(Hamiltonian system)\par
    Consider a $d$-bodies problem in $\mathbb{R}^3$, where denote $q_i$ the locations $i=1,\cdots, d$ and the potential energy to be $V(q) = V(q_1,\cdots,q_n)$ and we will have
    \[m_iq_i'' = -\dfrac{\partial}{\partial q_i}V(q)\]
    and for the momentum $p_i = m_iq_i'$ and we will have
    \[
    \begin{cases}
        q_i' = \dfrac{p_i}{m_i} \\
        p_i' = -\dfrac{\partial}{\partial q_i}V(q_i)
    \end{cases}
    \]
    where ingeneral we define the Hamiltonian function
    \[
    H(q,p) = T(p) + V(q)
    \]
    where $T$ for kinetic energy and $V$ for potential energy, then
    \[
    \begin{cases}
        q_i' = \dfrac{\partial H}{\partial p_i} \\
        p_i' = -\dfrac{\partial H}{\partial q_i}
    \end{cases}
    \]
    and hence $H' = 0$.
\end{example}

\subsection{One-dimensional Dynamics}

Consider Autonomous initial value problem for $x(t)\in \mathbb{R}$
\[
\begin{cases}
    x' = f(x) \\
    x(0) = x_0
\end{cases}
\]
then
\[
\int_0^t \dfrac{x'(s)}{f(x)} ds = \int_0^t ds \implies \int_{x_0}^x \dfrac{ds}{f(u)} du = t
\]

\begin{example}
    The logistic equation
    \[x' = rx(1-x)\]
    has equilibrium at $x \equiv 0$ and $x \equiv 1$, for the rest
    \[
    x(t) = \dfrac{x_0}{x_0 + (1-x_0)e^{-rt}}
    \]
    which implies $x\equiv 1$ to be an attractor.
\end{example}

\subsection{Two-dimensional Dynamics}

Consider $z = (x,y) \in \mathbb{R}^2, z' = [P(x,y),Q(x,y)]^T$ with equilibrium
\[S = \{(x,y), P(x,y) = Q(x,y) = 0\}\]
The nullclines, i.e. curves on which a single component vanishes
\[
N_x = \{P = 0\},\quad N_y = \{Q= 0\}
\]
and hence $S = N_x\cap N_y$.

\begin{definition}
    Phase curves: determine solutions to
    \[(x',y') = (P(x,y),Q(x,y))\]
    as curves in the phase space.
\end{definition}

Suppose $y = Y(x)$, an orbit is locally a graph, $y' = dY/dx x'$ and hence $dY/dx = y'/x' = Q/P = F(x,Y)$ which is a single, first order nonautonomous ODE.\par
Also the equation in the defintion can be viewed as
\[
\begin{cases}
    dx = Pdt \\
    dy = Qdt
\end{cases} \implies \dfrac{dx}{ P} = \dfrac{dy}{Q} = dt \implies -Q dx + Pdy = \alpha
\]
Suppose $\alpha = F(x,y)dH = F H_x dx + FH_y dy$ and hence
\[
\begin{cases}
    x' = F\tfrac{\partial H}{\partial x} \\
    y' = -F\tfrac{\partial H}{\partial y} 
\end{cases}
\]

\section{Matrix ODEs}

Consider $x' = f(x) , x(t) \in \mathbb{R}^n$, if $f(x) = Ax$ where $A \in \mathbb{R}^{n\times n}$, then
\[
\dfrac{dx}{dt} = Ax
\]
% ------------------------------------------------------------------------------
% Reference and Cited Works
% ------------------------------------------------------------------------------

\bibliographystyle{IEEEtran}
\bibliography{References.bib}

% ------------------------------------------------------------------------------

\end{document}
