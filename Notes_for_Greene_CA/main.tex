
%%%%%%%%%%%%%%%%中文%%%%%%蓝色标题%%%    
\documentclass[lang=en, color=blue, ]{elegantbook}
%%%使用包
\usepackage{amsmath, amssymb, amstext,mathrsfs}

%%%标题
\title{Notes for Greene CA}
%%%作者
\author{Wells Guan}
%%%封面中间色块
\definecolor{customcolor}{RGB}{102,102,255}
\colorlet{coverlinecolor}{customcolor}
%%%封面图

%%%自定义符号区
    %%% 组合数, 在数学环境中使用
\newcommand{\per}[2]{\left(\begin{array}{c} #1 \\ #2 \end{array}\right)}
\newcommand{\proba}[1]{\mathsf{P}(#1)}
%%%文档
\newcommand{\cov}{\text{cov}}
\newcommand{\var}{\text{var}}
\newcommand{\E}{\mathbb{E}}
\newcommand{\WN}{\varepsilon}
\newcommand{\pushop}{\mathscr{B}}
\newcommand{\F}{\mathcal{F}}
\newcommand{\R}{\mathbb{R}}
\newcommand{\Q}{\mathbb{Q}}
\newcommand{\N}{\mathbb{N}}
\newcommand{\C}{\mathbb{C}}
\newcommand{\B}{\mathcal{B}}
\newcommand{\ParZ}{\dfrac{\partial}{\partial z}}
\newcommand{\ParbZ}{\dfrac{\partial}{\partial \bar{z}}}
\newcommand{\ParX}{\dfrac{\partial}{\partial x}}
\newcommand{\ParY}{\dfrac{\partial}{\partial y}}
\begin{document}

%%%封面页

%%%正文

%%% Stochastic Processes
\chapter{}
\section*{Fundamental Concepts}

\begin{definition}
If $U\subset \R^2$ is open and $f:U\to\R$ is a continuous function, then $f$ is called $C^1$ on $U$ if $\partial f/\partial x, \partial f/\partial y$ exist and are continous on $U$.
\end{definition}

\begin{definition}
    We define for $f = u+iv:U\to \C$ a $C_1$ function
    \[
    \begin{aligned}
    \dfrac{\partial}{\partial z}f &:=\dfrac{1}{2}(\dfrac{\partial}{\partial x} - i\dfrac{\partial}{\partial y})f \\
    \dfrac{\partial}{\partial \bar{z}}f &:=\dfrac{1}{2}(\dfrac{\partial}{\partial x} + i\dfrac{\partial}{\partial y})f
    \end{aligned}
    \]
    which is easy to be checked linear and the chain rules.
\end{definition}
where we may check let $z = x+iy, \bar{z} = x-iy$, we have
\[
\begin{aligned}
    &\dfrac{\partial}{\partial z} z = 1,\quad \dfrac{\partial}{\partial z}\bar{z} = 0 \\
    &\dfrac{\partial}{\partial \bar{z}} z = 0,\quad \dfrac{\partial}{\partial \bar{z}}\bar{z} = 1
\end{aligned}
\]

\begin{proposition}
(The Leibniz Rules) We have for any $F,G \in C^1$
\[
\begin{aligned}
\dfrac{\partial}{\partial z}(F\cdot G) &= \dfrac{\partial F}{\partial z}\cdot G + F\cdot \dfrac{\partial G}{\partial z} \\ 
\dfrac{\partial}{\partial \bar{z}}(F\cdot G) &= \dfrac{\partial F}{\partial \bar{z}}\cdot G + F\cdot \dfrac{\partial G}{\partial \bar{z}} \\ 
\end{aligned}
\]
\end{proposition}

\begin{proposition}
    We have for $l\leq j, m\leq k$ nonnegative integers and then
    \[
    \begin{aligned}
    (\dfrac{\partial^l}{\partial z^l})(\dfrac{\partial^m}{\partial \bar{z}^m})(z^j\bar{z}^k) = \dfrac{j!}{l!}\dfrac{k!}{m!}z^{j-l}\bar{z}^{k-m}
    \end{aligned}
    \]
\end{proposition}

\begin{proposition}
    If $p(z,\bar{z}) = \sum a_{lm} z^l\bar{z}^m$ is a polynomaial, then $p$ contains no term with $m>0$ iff $\dfrac{\partial p}{\partial \bar{z}} \equiv 0$. 
\end{proposition}

\begin{corollary}
    If $p(z,\bar{z}) = q{z,\bar{z}}$ are polynomials, then they have same coefficients.
\end{corollary}

\begin{definition}
    A $C_1$ function $f:U\mapsto \C$ is said to be $holomorphic$ if
    \[\dfrac{\partial f}{\partial \bar{z}} = 0\]
    at every point of $U$.
\end{definition}

\begin{definition}
    A $C^1$ function $f = u(x,y)+iv(x,y): U \to \C$ is holomorphic if
    \[
    \begin{cases}\dfrac{\partial u}{\partial x} = \dfrac{\partial v}{\partial y} \\
    \dfrac{\partial u}{\partial y} = -\dfrac{\partial v}{\partial x}
    \end{cases}\]
    at every point of $U$, which is called the $Cauchy$-$Riemann$ equations.
\end{definition}

\begin{proposition}
    If $f:U\to\C$ is $C^1$ and if $f$ satisfies the C-R equations, then
    \[\ParZ f = \ParX f = -i\ParY f \]
    on $U$.
\end{proposition}
\begin{proof}\par
    We have
    \[
    \begin{aligned}
    \ParX f &= \ParX u + i\ParX v = (\ParX -i\ParY) u = 2\ParZ u \\
    \ParX f &= \ParX u + i\ParX v = i(\ParX - i\ParY) v = 2\ParZ iv \\
    -i\ParY f &= -i\ParY u + \ParY v = (\ParX -i\ParY) u = 2\ParZ u \\
    -i\ParY f &= -i\ParY u + \ParY v = i(\ParX - i\ParY) v = 2\ParZ iv \\
    \end{aligned}
    \]
    on $U$.
\end{proof}

\begin{definition}
If $U\subset \C$ is open and $u\in C^2(U)$, then $u$ is called $harmonic$ if
\[\dfrac{\partial^2 u}{\partial x^2}+\dfrac{\partial^2 u}{\partial y^2} = 0\]
where we also denote it as
\[\Delta u = \dfrac{\partial^2 u}{\partial x^2}+\dfrac{\partial^2 u}{\partial y^2}\]
where the operator is called the $Laplace\ operator$.
\end{definition}
Here we have
\[4\ParbZ\ParZ u = 4 \ParZ\ParbZ = \Delta u\]

\begin{proposition}
    The real and imaginary parts of a holomorphic $C^2$ function are harmonic.
\end{proposition}
\begin{proof}\par
    Assume $f = u + iv$ and then according to C-R equations, we have
    \[\dfrac{\partial^2}{\partial x^2} u = \dfrac{\partial^2}{\partial x\partial y} v = \dfrac{\partial^2}{\partial y\partial x} v = -\dfrac{\partial^2}{\partial y^2} u
    \]
    and
    \[\dfrac{\partial^2}{\partial x^2} v = -\dfrac{\partial^2}{\partial x\partial y} u = -\dfrac{\partial^2}{\partial y\partial x} u = -\dfrac{\partial^2}{\partial y^2} v
    \]
\end{proof}

\begin{lemma}
It $u(x,y)$ is a real-valued polynomial with $\Delta u = 0$, then there exists a (holomorphic) $Q(z)$ such that $Re Q= u$.
\end{lemma}
\begin{proof}\par
    Consider $u(x,y) = u(\dfrac{z+\bar{z}}{2},\dfrac{z-\bar{z}}{2}) = P(z,\bar{z}) = \sum a_{lm}z^l\bar{z}^m$, we know $\Delta u = 0$ and hence
    \[P(z,\bar{z}) = a_00 + \sum^m a_k z^k + \sum^n b_k \bar{z}^k\]
    $P$ is real-valued and we know
    \[a_00 + \sum^m a_k z^k + \sum^n b_k \bar{z}^k = \bar{a_00} + \sum^m \bar{a_k} \bar{z}^k + \sum^n \bar{b_k} z^k\]
    and hence $a_00\in \R, a_k = \bar{b_k}$ and hence
    \[
    u(z) = c + \sum^n a_{k}z^k + \sum^n \bar{a_{k}}\bar{z}^k = Re(c+2\sum^n a_k z^k) = Re(Q)
    \]
    where $Q$ is obviously holomorphic.
\end{proof}

\begin{theorem}
    If $f,g$ are $C^1$ functions on the rectangle
    \[\mathcal{R} = \{(x,y)\in \R^2: |x-a|<\delta, |y-b|<\epsilon\}\]
    and if 
    \[
    \dfrac{\partial f}{\partial y} = \dfrac{\partial g}{\partial x}\text{ on }\mathcal{R}
    \]
    then there is a function $h\in C^(\mathcal{R})$ such that
    \[\ParX h =f, \ParY h = g\]
    on $\mathcal{R}$. If $f,g$ are real-valuedd, the nwe may take $h$ to be real-valued also.
\end{theorem}
\begin{proof}\par
    For $(x,y) \in \mathcal{R}$, define
    \[h(x,y) = \int_a^x f(t,b) dt + \int_b^y g(x,s)ds\]
    and we know
    \[
    \ParY h(x,y) = g(x,y)
    \]
    and 
    \[\ParX h(x,y) = f(x,b) + \ParX \int_b^y g(x,s) ds = f(x,b) + \int_b^y \ParY f(x,s) = f(x,b) + f(x,y)-f(x,b) = f(x,y)\]
    and hence $h \in C^2(\mathcal{R})$ and real-valued if $f,g$ are.
\end{proof}

\begin{corollary}
    If $\mathcal{R}$ is an open rectangle (or open disc) and if $u$ is a real-valued harmonic function on $\R$, then there is a holomorphic function $F$ on $\R$ such that $Re F = u$.
\end{corollary}
\begin{proof}\par
    We know 
    \[\dfrac{\partial^2}{\partial x^2} u + \dfrac{\partial^2}{\partial y^2} u =0\]
    and hence there exists $v$ real-valued such taht
    \[\ParX v = - \ParY u, \ParY v = \ParX u\]
    and hence $F = u+iv$ is a holomorphic function with $Re(F) = u$.
\end{proof}

\begin{theorem}
    If $U\subset \C$ is either an open rectangle or an open disc and if $F$ is holomorphic on $U$, then there is a holomorphic function $H$ on $U$ such that $\partial H/\partial z = F$ on $U$. 
\end{theorem}
\begin{proof}\par
    Consider $H = h_1 + ih_2$ and 
    we have $F = u(z) + iv(z)$, then we let $f= u ,g = -v$ and we will have
    \[\ParY f = \ParX g\]
    and hence we have a real $C^2$ function $h_1$ such that
    \[\ParX h_1 = u, \ParY h_1 = -v\]
    and $h_2 \in C^2$ with
    \[\ParX h_2 = v, \ParY h_2 = u\]
    Then
    \[\ParZ H = \dfrac{1}{2}(\ParX h_1 + \ParY h_2)+\dfrac{i}{2}(\ParX h_2 - \ParY h_1) = u+iv = F\]
\end{proof}

\begin{definition}
    A function $\phi:[a,b]\to\R$ is called $continuously\ differentiable$ and we write $\phi\in C^1([]a,b)$ if\par
    (a) $\phi$ is continous on $[a,b]$\par
    (b) $\phi'$ exists on $(a,b)$\par
    (c) $\phi'$ has a continuous extension to $[a,b]$, i.e.
    \[\lim_{t\to a^+} \phi'(t)\text{ and }\lim_{t\to b^-} \phi'(t)\]
    both exists. Then $\phi(b)-\phi(a) = \int_a^b \phi'(t)dt$.
\end{definition}
\begin{proof}\par
    Here notice that $\phi$ is absolutely continuous on $[a,b]$ respect to $m$, then we know $\phi(b-\epsilon) - \phi(a+\epsilon) = \int_{a+\epsilon}^{b-\epsilon} \phi'(t)dt$ for any $epsilon > 0$, and hence
    \[\phi(b)-\phi(a) = \int_a^b \phi'(t) dt\]    
\end{proof}

\begin{definition}
    A curve $\gamma:[a,b]\to \C$ is said to be $continuous$ on $[a,b]$ if both $\gamma_1$ and $\gamma_2$ are, $\gamma = \gamma_1 + i\gamma_2$. The curve is $C_1$ on $[a,b]$ if $\gamma_1,\gamma_2$ are $C_1$ on $[a,b]$ and then we may denote
    \[\dfrac{d\gamma}{dt} = \dfrac{d\gamma_1}{dt} + i\dfrac{d\gamma_2}{dt}\]
\end{definition}

\begin{definition}
    Let $\varphi:[a,b] \to \C$ be continuous on $[a,b]$. Write $\varphi(t) = \varphi_1(t) + i\varphi_2(t)$. Then we define
    \[\int_a^b \varphi(t) dt = \int_a^b \varphi_1(t)dt + i\int_a^b \varphi_2(t) dt\]
\end{definition}

\begin{proposition}
    Let $U\subset \C$ be open and let $\gamma:[a,b]\to U$ be a $C_1$ curve. If $f:U\to\R$ and $f\in C^1(U)$, then
    \[f(\gamma(b))-f(\gamma(a)) = \int_a^b\Big(\ParX f(\gamma(t))\dfrac{d\gamma_1}{dt}+\ParY f(\gamma(t))\dfrac{d\gamma_2}{dt}\Big)dt\]
\end{proposition}
This is due to the chain rule.

\begin{proposition}
    Repalce $f$ above as complex-valued and holomorphic, then we have
    \[f(\gamma(b))-f(\gamma(a)) = \int_a^b \ParZ f(\gamma(t))\cdot \dfrac{d\gamma}{dt}(t)dt\]
\end{proposition}
\begin{proof}\par
    Notice
    \[
    \begin{aligned}
    f(\gamma(b)) - f(\gamma(a))
    &= \int_a^b \Big(\ParX u(\gamma(t))\dfrac{d\gamma_1}{dt}(t)+ \ParY u(\gamma(t))\dfrac{d\gamma_2}{dt}(t)\Big) +i \Big(\ParX v(\gamma(t))\dfrac{d\gamma_1}{dt}(t)+ \ParY v(\gamma(t))\dfrac{d\gamma_2}{dt}(t)\Big) dt \\ &= \ParX f(\gamma(t))\dfrac{d\gamma}{dt}(t) = \int_a^b\ParZ f(\gamma(t)) \dfrac{d\gamma}{dt}(t) dt
    \end{aligned}
    \]
\end{proof}

\begin{definition}
    If $U\subset \C$ open and $F:U\to\C$ is continuous on $U$ and $\gamma:[a,b]\to U$ is a $C_1$ curve, then we define the $complex\ line\ integral$
    \[\int_{\gamma} F(z)dz = \int_a^b F(\gamma(t))\dfrac{d\gamma}{dt}dt\]
\end{definition}

\begin{proposition}
    Let $U\subset \C$ be open and let $\gamma:[a,b]\to U$ be a $C^1$ curve. If $f$ is a holomorphic function on $U$, then
    \[f(\gamma(b)) - f(\gamma(a)) = \int_{\gamma} \ParZ f(z)dz\]
\end{proposition}

\begin{proposition}
If $\phi:[a,b] \to \C$ is continuous, then
\[|\int_a^b \phi(t)dt|\leq \int_a^b |\phi(t)|dt\]
\end{proposition}

\begin{proposition}
Let $U \subset \C$ be open and $f\in C^0(U)$. If $\gamma:[a,b]\to U$ is a $C^1$ curve, then
\[|\int_{\gamma} f(z)dz| \leq (\sup_{t\in[a,b]} |f(\gamma(t))|)\cdot l(\gamma)\]
where
\[l(\gamma) = \int_a^b |\dfrac{d\gamma}{dt}(t)|dt\]
\end{proposition}

\begin{proposition}
    Let $U\subset \C$ be an open set and $F:U\to\C$ a continuous function. Let $\gamma:[a,b]\to U$ be a $C^1$ curve. Suppose that $\theta: [c,d]\to[a,b]$ is a one-to-one, onto, increasing $C^1$ function with a $C^1$ inverse. Let $\tilde{\gamma} = \gamma \circ \phi$. Then
    \[\int_{\tilde{\gamma}} fdz = \int_{\gamma} fdz\]
\end{proposition}
\begin{proof}\par
    We have
    \[
    \begin{aligned}
        \int_{\tilde{\gamma}} fdz = \int_c^d f(\gamma(\phi(t))) \dfrac{d\gamma(\phi(t))}{dt} dt = \int_a^b f(\gamma(s)) \dfrac{\gamma(s)}{ds} \phi'(\phi^{-1}(s)) (\phi^{-1})'(s)ds = \int_{\gamma} fdz
    \end{aligned}
    \]
    since $\phi'(\phi^{-1}(s))(\phi^{-1})' = 1$.
\end{proof}

\begin{definition}
    Let $f$ be a function on the open set $U$ in $\C$ and consider if 
    \[\lim_{z\to z_0} \dfrac{f(z)-f(z_0)}{z-z_0}\]
    exists then we say that $f$ has a $complex\ derivative$ at $z_0$. We denote the complex derivative by $f'(z_0)$. 
\end{definition}

\begin{theorem}
    Let $U\subset \C$ be an open set and let $f$ be holomorphic on $U$. Then $f'$ exists at each point of $U$ and
    \[f'(z) = \ParZ f\]
    for all $z\in U$.
\end{theorem}
\begin{proof}\par
    Consider
    \[\gamma(t) = (1-t)z_0 + tz\]
    and then we know
    \[f(z)-f(z_0) = f(\gamma(1)) - f(\gamma_0) = \int_{\gamma} \ParZ fdz = (z-z_0) \int_0^1 \ParZ f(\gamma(t)) dt = \ParZ f(z_0) + \int_0^1(\ParZ f (\gamma(t))- \ParZ f(z_0))dt\]
    and hence
    \[|\dfrac{f(z)-f(z_0)}{z-z_0}- \ParZ f(z_0)| \leq \int |\ParZ f(\gamma(t))- \ParZ f(z_0)|dt \to 0\]
    when $z\to z_0$.
\end{proof}

\begin{theorem}
    If $f\in C^1(U)$ and $f$ has a complex derivative at each point of $U$, then $f$ is holomorphic on $U$. In particular, if a continuous, complex-valued function $f$ on $U$ has a complex derivative at each point and if $f'$ is continuous on $U$, then $f$ is holomorphic on $U$.
\end{theorem}
\begin{proof}\par
    It is easy to check
    \[\lim_{h\to 0, h\in\R} \dfrac{f(z_0+h)-f(z_0)}{h} = \ParX u (x_0, y_0) + i\ParX v(x_0,y_0)\]
    and
    \[
    \lim_{h\to 0, h\in \R}\dfrac{f(z_0+h)-f(z_0)}{h} =  - i\ParY u(x_0,y_0) + \ParY v(x_0,y_0)
    \]
    and hence $f$ satisfies the C-R equations so holomorphic.\par
    Notice the continuity of $f'$ may implies that $f\in C^1(U)$ and hence the problem goes.
\end{proof}

\begin{theorem}
    Let $f$ be holomorphic in a neighborhood of $P\in\C$. Let $\omega_1,\omega_2$ be complex numbers of unit modulus. Consider the directional derivatives
    \[D_{\omega_1}f(P) = \lim_{t\to 0}\dfrac{f(P+t\omega_1)-f(P)}{t}\]
    and
    \[D_{\omega_2}f(P) = \lim_{t\to 0}\dfrac{f(P+t\omega_2)-f(P)}{t}\]
    then\par
    a. $|D_{\omega_1} f(P)| = |D_{\omega_2}f(P)|$\par
    b. If $f'(P) \neq 0$, then the directed angle from $\omega_1$ to $\omega_2$ equals the directed angle from $D_{\omega_1}f(P)$ to $D_{\omega_2} f(P)$. 
\end{theorem}
\begin{proof}\par
    Notice that
    \[D_{\omega_j} = f'(P)\omega_j, j= 1,2\]
    and then the conclusions go.
\end{proof}

\begin{lemma}
    Let $(\alpha,\beta)\subset\R$ be an open interval and let $H:(\alpha,\beta)\to\R, F:(\alpha,\beta)\to\R$ be continuous functions. Let $p\in(\alpha,\beta)$ and suppose that $dH/dx$ exists and equals $F(x)$ for all $x\in(\alpha,\beta)-\{p\}$. Then $(dH/dx)(p)$ exists and $(dH/dx)(x) = F(x)$ for all $x\in(\alpha,\beta)$.
\end{lemma}
\begin{proof}\par
    Assume $[a,b]\subset (\alpha,\beta)$ and then $K(x) = H(a) + \int_a^x F(t)dt$ on $[a,b]$, so we know $K-H$ is continuous on $[a,b]$ and constant on $[a,p)\cup(p,b]$, which means $K=H$ on $[a,b]$.
\end{proof}

\begin{theorem}
    Let $U\subset \C$ be either an open rectangle or an open disc and let $P\in U$. Let $f$ and $g$ be continuous, real-valued functions on $U$ which are continuously differentiable on $U-\{P\}$. Suppose further that
    \[\ParY f = \ParX g\text{ on }U-\{P\}\]
    Then there exists a $C^1$ function $h:U\to\R$ such that
    \[\ParX = f, \ParY = g\]
    at every point of $U$.
\end{theorem}
\begin{proof}\par
    Consider a closed rectangle containing $p$ inside in $U$ and define $h(x,y) = \int_a^x f(t,b)dt + \int_b^y g(x,s)ds$ and we know that $\ParY h = g(x,y)$ and $\ParX h = f(x,y)$ for any $x\neq P_x$, then for a fixed $y$, we know $dh(x,y)/dx= f(x,y)$ exists for all points in $U$ except for $(p_x,y)$ and hence $dh(x,y)/dx = f(x,y)$ at $(p_x,y)$. Then we know $\ParX h = f, \ParY h = g$ on $U$.
\end{proof}

\begin{theorem}
    Let $U\subset\C$ be either an open rectangle or an open disc. Let $P\in U$ be fixed. Suppose that $F$ is continuous on $U$ and holomorphic on $U-\{P\}$. Then there is a holomorphic $H$ on $U$ such that $U$ such that $\ParZ H = F$.    
\end{theorem}
\begin{proof}\par
    Consider $F = u+iv$, then we have
    \[\ParY v = \ParX u\text{ and } \ParY u = \ParX (-v)\]
    on $U-\{P\}$, then we know there exists $h_1,h_2$ on $U$ such that $\ParX h_1 = u, \ParY h_1 = (-v), \ParX h_2 = v, \ParY h_2 = u$
    and let $H = h_1 + ih_2$, we have
    \[
    \ParZ H = \dfrac{1}{2}(\ParX - i\ParY)(h_1+ih_2) = (u+u) + i(v+v) = F
    \]
\end{proof}

\begin{definition}
    The boundary $\partial D(P,r)$ of the disc $D(P,r)$ can be parametrized as a simple closed curve $\gamma:[0,1]\to \C$ by setting
    \[\gamma(t) = P+re^{2\pi it}\]
    we call it $counterclockwise$ orientation.
\end{definition}

\begin{lemma}
    Let $\gamma$ be the boundary of a disc $D(z_0,r)$ in the complex plane, equipped with counterclockwise orientation. Let $z$ be a point inside the circle $\partial D(z_0,\gamma)$ . Then
    \[\dfrac{1}{2\pi i} \int_{\gamma} \dfrac{1}{\xi-z}d\xi = 1\]
\end{lemma}
\begin{proof}\par
    Consider $I(z) = \int_{\gamma} \dfrac{1}{\xi-z}d\xi = \int_0^1 \dfrac{1}{(z_0+e^{2\pi it})-z}(2\pi i)e^{2\pi i t}dt$ and since
    \[\ParX \dfrac{1}{\xi-z} = \dfrac{1}{(\xi-z)^2},\quad \ParY \dfrac{1}{\xi -z} = i\dfrac{1}{(\xi-z)^2}\]
    and hence we have
    \[\ParbZ I(z) = \int_{\gamma} \ParbZ(\dfrac{1}{\xi-z})d\xi = 0\quad \ParZ I(z) = \int_{\gamma} \ParZ (\dfrac{1}{\xi-z})d\xi = \int_{\gamma}\dfrac{1}{(\xi-z)^2} d\xi\]
    where $\dfrac{1}{(\xi-z)^2}$ is the complex derivative of the holomorphic function $\dfrac{-1}{\xi-z}$ and hence 
    \[\ParZ I(z) = \int_{\gamma} \dfrac{1}{(\xi-z)^2}d\xi = 0\]
    Therefore, $I(z)$ is holomorphic on $D(z_0,r)$ and $\ParZ I = 0$ which means $I$ is constant on $D(z_0,r)$ and notice
    \[I(z_0) = 2\pi i\]
    and hence the equation holds.
\end{proof}

\begin{theorem}
    (The Cauchy integral fomula) Suppose that $U$ is an open set in $\C$ and that $f$ is a holomorphic function on $U$. Let $z_0\in U$ and let $r>0$ be such that $\overline{D}(z_0,r) \subset U$. Let $\gamma:[0,1] \to \C$ be the $C^1$ curve $\gamma(t)= z_0+r\cos(2\pi t) + ir \sin(2\pi t)$. Then for each $z\in D(z_0,r)$,
    \[f(z) = \dfrac{1}{2\pi i}\int_{\gamma} \dfrac{f(\xi)}{\xi-z}d\xi\] 
\end{theorem}
\begin{proof}
    By theorem 1.7, there is $H$ such that
    \[\ParZ H = \dfrac{f(\xi)-f(z)}{\xi-z}\]
    if $\xi \neq z$ and $\ParZ H(z) = f'(z)$ holomorphic on $D(z_0,r+\epsilon)$ and hence
    \[\int_{\gamma} \dfrac{f(\xi)-f(z)}{\xi-z}d\xi = 0\]
    and the equation holds by the lemma 1.3.
\end{proof}

\begin{theorem}
    (The Cauchy integral theorem) If $f$ is a holomorphic function on an open disc $U$ in the complex plane, and if $\gamma:[a,b] \to U$ is a $C^1$ curve in $U$ with $\gamma(a) = \gamma(b)$, then
    \[\int_{\gamma} f(z)dz = 0\]
\end{theorem}
\begin{proof}
Only need to pick $G$ such that $\ParZ G = f$ on $U$ is fine.
\end{proof}

\begin{definition}
    A $piecewise$ $C^1$ curve $\gamma:[a,b]\to\C, a<b,a,b\in\R$ is a continuous function such that there exists a finite set of numbers $a_1\leq a_2 \leq \cdots \leq a_k$ satisfying $a_1=a$ and $a_k=b$ and with the property that for every $1\leq j \leq k-1$, $\gamma|_{[a_j,a_{j+1}]}$ is a $C^1$ curve. As before, $\gamma$ is a piecewise $C^1$ curve in an open set $U$ if $\gamma_{[a,b]}\subset U$.
\end{definition}

\begin{definition}
    If $U\subset \C$ is open and $\gamma:[a,b]\to U$ is a piecewise $C^1$ curve in $U$ and if $f:U\to\C$ is a continuous, complex-valued function on $U$, then
    \[
    \int_{\gamma} f(z)dz = \sum\limits_{j=1}^k \int_{\gamma|_{[a_j,a_{j+1}]}} f(z)dz
    \]
    and the definition is well-defined.
\end{definition}
\begin{proof}\par
    We need to show for any $\{a_j\}_1^k,\{b_i\}_1^m$, the RHS determined by the chosen sequence is the same. Assume $a_{j_t} = b_{i_t}, 1\leq t \leq q$, with $\{a_j\}_{j_t+1}^{j_{t+1}-1} \cap \{b_i\}_{i_t+1}^{j_{i+1}-1} = \emptyset$, then we know $\gamma|_{a_{j_t},a_{j_{t+1}}}$ is a $C_1$ curve and hence the integral over the curve is the same.
\end{proof}

\begin{lemma}
    Let $\gamma:[a,b]\to U$ open in $\C$ to be a piece wise $C^1$ curve. Let $\phi:[c,d]\to [a.b]$ be a piecewise $C^1$ strictly monotone increasing function with $\phi(c) = a, \phi(d) = b$. Let $f:U\to \C$ be a continuous function on $U$. Then the function $\gamma\circ \phi:[c,d]\to U$ is a piecewise $C^1$ curve and
    \[
        \int_{\gamma} f(z)dz = \int_{\gamma\circ\phi} f(z)dz
    \]
\end{lemma}
\begin{proof}
    Use the proposition 1.11.
\end{proof}

\begin{lemma}
If $f:U\to \C$ is a holomorphic function and if $\gamma:[a,b]\to U$ is a piecewise $C^1$ curve, then
\[f(\gamma(b))-f(\gamma(a)) = \int_{\gamma} f'dz\]
\end{lemma}
\begin{proof}
    Use the proposition 1.7.
\end{proof}

\begin{proposition}
If $f:\C-\{0\} \to \C$ is a holomorphic function, and if $\gamma_r$ describes the circle of radius $r$ around $0$, traversed once around counter-clockwise, then, for any two positive numbers $r_1<r_2$,
\[\int_{\gamma_{r_1}} f(z)dz = \int_{\gamma_{r_2}} f(z)dz\]
\end{proposition}

\begin{proposition}
    Let $0<r<R<\infty$ and define the annulus $\mathcal{A} = \{z\in\C: r < |z| < R\}$. Let $f;\mathcal{A}\to\C$ be a holomorphic function. If $r<r_1<r_2<R$ and if for each $j$ the curve $\gamma_{r_j}$ describes the circle pf radius $r_j$ around $0$, traversed once counter clockwise, then we have 
    \[
    \int_{\gamma_{r_1}}fdz = \int_{\gamma_{r_2}}fdz
    \]
\end{proposition}

\newpage

\subsection*{Applications of the Cauchy integral}
\begin{theorem}
    Let $U\subset \C$ be an open set and let $f$ be holomorphic on $U$. Then $f\in C^{\infty}(U)$. Moreover, if $\overline{D}(P,r)\subset U$ and $z\in D(P,r)$, then
    \[
    \Big(\ParZ \big)^k f(z) = \dfrac{k!}{2\pi i}\int_{|\xi-P|=r}\dfrac{f(\xi)}{(\xi-z)^{k+1}}d\xi
    \]
    for any integer $k$.
\end{theorem}
\begin{proof}\par
    Use the induction to $f$, assume
    \[
    (\ParZ \big)^k f(z) = \dfrac{k!}{2\pi i}\int_{|\xi-P|=r}\dfrac{f(\xi)}{(\xi-z)^{k+1}}d\xi
    \]
    and $(\ParZ \big)^k f(z)$ is holomorphic, then we gonna prove that
    \[
    (\ParZ \big)^{k+1} f(z) = \dfrac{(k+1)!}{2\pi i}\int_{|\xi-P|=r}\dfrac{f(\xi)}{(\xi-z)^{k+2}}d\xi
    \]
    and $(\ParZ \big)^{k+1} f(z)$ is holomorphic. Consider
    \[
    \begin{aligned}
    \Big|\dfrac{f(\xi)}{(\xi-\omega)^{k+1}}-
    \dfrac{f(\xi)}{(\xi-z)^{k+1}}\Big| &\leq \sup_{\xi\in \partial D(P,r)}|f(\xi)| \epsilon^{-2k-2}\Big|\sum\limits_{i=1}^{k+1} C_{k+1}^{i}(2r)^{k+1-i}(\omega-z)^i\Big| \\
    & \leq |\omega-z|(k+1)\Big(\sup_{\xi\in \partial D(P,r)}|f(\xi)| \epsilon^{-2k-2}\Big|\sum\limits_{i=0}^{k} C_{k}^{i}(2r)^{k-i}(\omega-z)^i\Big|\Big)\\
    & \leq |\omega-z|(k+1)\Big(\sup_{\xi\in \partial D(P,r)}|f(\xi)| \epsilon^{-2k-2}(2r+1)^k\Big)
    \end{aligned}
    \]
    for all $|\omega-z|$ small enough and hence
    \[
    \dfrac{f(\xi)}{(\xi-\omega)^{k+1}} \to 
    \dfrac{f(\xi)}{(\xi-z)^{k+1}}
    \]
    uniformly when $\omega \to z$, so may know
    \[
    \lim_{\omega \to z} \dfrac{(\ParZ \big)^{k+1} f(\omega) - (\ParZ \big)^{k+1} f(z)}{\omega - z} = \lim_{\omega \to z} \dfrac{k!}{2\pi i}\int_{|\xi-P|=r}\dfrac{\dfrac{f(\xi)}{(\xi-\omega)^{k+1}}-\dfrac{f(\xi)}{(\xi-z)^{k+1}}}{\omega - z}d\xi
    \]
    and we know that
    \[
    \lim_{\omega \to z} \dfrac{k!}{2\pi i}\int_{|\xi-P|=r}\dfrac{\dfrac{f(\xi)}{(\xi-z)^{k+1}}-\dfrac{f(\xi)}{(\xi-z)^{k+1}}}{\omega - z}d\xi =  \dfrac{k!}{2\pi i}\int_{|\xi-P|=r}\lim_{\omega \to z}\dfrac{\dfrac{f(\xi)}{(\xi-\omega)^{k+1}}-\dfrac{f(\xi)}{(\xi-z)^{k+1}}}{\omega - z}d\xi
    \]
    by the DCT and hence
    \[
    \lim_{\omega \to z} \dfrac{(\ParZ \big)^{k+1} f(\omega) - (\ParZ \big)^{k+1} f(z)}{\omega - z}
    = \dfrac{(k+1)!}{2\pi i}\int_{|\xi-P|=r}\dfrac{f(\xi)}{(\xi-z)^{k+2}}d\xi
    \]
    which means $(\ParZ \big)^k f(z)$ is holomorphic and the equality holds. Then we use the induction, and the conclusion goes.
\end{proof}

\begin{corollary}
    If $f:U\to\C$ is holomorphic, then $f':U\to \C$ is holomorphic.
\end{corollary}

\begin{theorem}
    If $\phi$ is a continuous function on $\{\xi:|\xi-P| = r\}$, then the function $f$ given by
    \[
    f(z) = \dfrac{1}{2\pi i}\int_{|\xi-P|=r}\dfrac{\phi(\xi)}{\xi-z}d\xi
    \]
    is defined and holomorphic on $D(P,r)$.
\end{theorem}

\begin{theorem}
    (Morera) Suppose that $f:U\to \C$ is a continuous function on a connected open subset $U$ of $\C$. Assume that for every closed, piecewise $C^1$ curve $\gamma:[0,1]\to U$, $\gamma(0) = \gamma(1)$, it holds that
    \[
    \int_{\gamma} f(\xi) d\xi = 0
    \]
    Then $f$ is holomorphic on $U$.
\end{theorem}
\begin{proof}
    Consider $x\in U$ and define $F(y) = \int_{\phi} fdz$ for any $y\in U$ where $\phi$ is a picewise $C^1$ curve from $x$ to $y$, where we know the integral is well-defined since any integral of $f$ on a closed, piece wise $C^1$ curve is $0$. Then for any $y\in U$, consider a segment from $y+h$ where $|h|$ is small enough and we know
    \[
    \lim_{|h|\to 0}\dfrac{F(y+h)-F(y)}{h} = \lim_{|h|\to 0}\dfrac{1}{h}\int_0^h f(y+z) dz = f(y)
    \]
    which means $F$ is holomorphic on $U$ and $F' = f$ on $U$, and hence $f$ is holomorphic on $U$.
\end{proof}

\begin{definition}
    let $P\in\C$ be fixed. A $complex\ power\ series$ centered at $P$ is an expression of the form
    \[
    \sum a_k(z-P)^k
    \]
    where $a_k$ is complex valued.
\end{definition}

\begin{lemma}
    (Abel) If $\sum a_k(z-P)^k$ converges at some $z$, then the series converges at each $\omega \in D(P,r)$, where $r = |z-P|$.
\end{lemma}
\begin{proof}\par
    Since $\sum a_k(z-P)^k$ converges, we know $a_k(z-P)^k \to 0$ and hence bounded, then we know 
    \[
    |a_k| \leq Mr^{-k}
    \]
    for some $M>0$ and then for any $\omega \in D(P,r)$, assume $|\omega - P| = \delta < r$, then we know
    \[
    |a_k(\omega-P)^k| \leq |a_k|\delta^k \leq M(\delta/r)^{-k}
    \]
    and hence
    \[
    \sum |a_k(\omega-P)^k| \leq M\sum (\delta/r)^{-k} < \infty
    \]
    which means $\sum a_k(\omega-P)^k$ converges.
\end{proof}

\begin{definition}
    Let $\sum a_k(z-P)^k$ be a power series. Then
    \[r = \sup\{|\omega-P|:\sum a_k(\omega-P)^k\text{ converges}\}\]
    is called the $radius\ of\ convergence$ of the power series.
\end{definition}

\begin{lemma}
    If $\sum a_k(z-P)^k$ is a power series with radius of convergence $r$, then the series converges for each $\omega \in D({P,r})$ and diverges for each $\omega$ such that $|\omega-P|>r$.
\end{lemma}

\begin{lemma}
    (The root test) The radius of convergence of the power series $\sum a_k(z-P)^k$ is
    \[\dfrac{1}{\limsup |a_k|^{1/k}}\]
    if $\limsup |a_k|^{1/k} > 0$ or
    \[\infty\]
    if $\limsup |a_k|^{1/k} = 0$.
\end{lemma}
\begin{proof}\par
    Assume $\alpha = \limsup |a_k|^{1/k}$, if $|\omega-P|>1/\alpha$, then denote $|\omega-P| = c/\alpha,c>1$ and we know
    \[
    |a_k(z-P)^k| = (c|a_k|^{1/k}/\alpha)^k
    \]    
    and we know there are infinitly many $a_k$ such that $|a_k|^{1/k}/\alpha > 1/c$ and hence the series diverge.\par
    For $|\omega-P|<1/\alpha$, we denote $|\omega-P| = d/\alpha, d<1-\epsilon$ for some $\epsilon >0$ and we have
    \[
    |a_k(\omega-P)^k| \leq (|a_k|^{1/k}d/\alpha)^k \leq (1-\epsilon)^k 
    \]
    when $k$ is sufficiently large and hence the series is absolutely convergent and the condition for $\alpha = 0$ is similar.
\end{proof}

\begin{definition}
    Let $\sum f_k(z)$ be a series of functions on a set $E$. The series is said to be uniformly Cauchy if for any $\epsilon > 0$, these is an integer $N$ such that
    \[
    |\sum\limits_{k=m}^{n} f_k(z)| <\epsilon
    \]
    on $E$ for any $n\geq m \geq N$.
\end{definition}

\begin{proposition}
    Let $\sum a_k(z-P)^k$ be a power series with radius of convergence $r$. Then, for any number $R$ with $0 \leq R < r$ ,the series $\sum |a_k(z-P)|^k$ converges uniformly on $\overline{D}(P,R)$ and hence $\sum a_k(z-P)^k$ converges uniformly and absolutely on $\overline{D}(P,R)$.
\end{proposition}
\begin{proof}

\end{proof}

\end{document}