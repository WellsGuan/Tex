\section{Gaps in Critical Temperatures}

Thoerem B d=3.

\subsection{Preliminaries}

\begin{theorem}(From F.Comets and N.Yoshida 2006)\par
    \begin{itemize}
        \item There exists $\beta_c \in [0,\infty]$ such that weak disorder, i.e. $W_{\infty}^{\beta} > 0$ a.s., holds for $\beta < \beta_c$ and strong disorder,i.e. $W_{\infty}^{\beta} = 0$ a.s., holds for $\beta > \beta_c$.
        \item The function
        \[\beta\mapsto f(\beta):= \lim_{n\to\infty} \dfrac{1}{n}\mathbb{E}(\ln W_n^{\beta}) = p(\beta) - \lambda(\beta)\]
        is non-increasing. There exists $\bar{\beta}_c \in [0,\infty]$ such that very strong disorder, i.e. $f(\beta) < 0$, holds iff $\beta > \bar{\beta}_c$.
    \end{itemize}
\end{theorem}
\begin{definition}
    (The P2P partition function)\par
    \[
    \widehat{W}^{\beta}_n(x):=E(\exp(\beta H_n(X))\chi({X_n = x}))
    \]
    and then endpoint distribution $\mu_n^{\beta}$ is defined by
    \[
    \mu_n^{\beta}(x):= P^{\beta,\omega}_n(X_n = x) = \dfrac{\widehat{W}^{\beta}_n(x)}{Z^{\beta}_n}
    \]
\end{definition} 

\begin{proposition}
    \[
    P^{\beta,\omega}_{n-1}(X_n = x) = \sum\limits_{y} \dfrac{\chi(|x-y|=1)}{2d}\mu_{n-1}^{\beta}(y) = D\mu_{n-1}^{\beta}(x)
    \]
    where $D$ is the transition matrix of SRW on $\mathbb{Z}^d$.
\end{proposition}

\begin{definition}
    (The end-point replica overlap)\par
    Define \[I_n : = \sum\limits_x D\mu_{n-1}^{\beta}(x)^2\] and we will have
    \[
    \mathbb{E}[(W_n^{\beta} - W_{n-1}^{\beta})^2|\mathcal{F}_n] = \xi(\beta)(W_{n-1}^{\beta})^2 I_n
    \]
    where $\xi(\beta) := \exp(\lambda_2(\beta)) - 1$.
\end{definition}
\begin{proof}
    We have
    \[\begin{aligned}
        \mathbb{E}[(W_n^{\beta} - W_{n-1}^{\beta})^2|\mathcal{F}_n] &= (W_{n-1}^{\beta})^2 \mathbb{E}\left[\left(\dfrac{W_n}{W_{n-1}}\right)^2 - 1| \mathcal{F}_n\right] \\
        &= (W_{n-1}^{\beta})^2 \mathbb{E}\left[\left(\dfrac{1}{W_{n-1}}\sum\limits_x \dfrac{1}{2d} \exp(\beta\omega H_n(x) - n\lambda(\beta))\right)^2 - 1| \mathcal{F}_n\right] \\
        &= (W_{n-1}^{\beta})^2 \mathbb{E}\left[(E^{\beta,\omega}_{n-1})^{\otimes 2}(e^{\beta(\omega(n,S_n)+\omega(n,\tilde{S}_n)) - 2\lambda(\beta)}-1)| \mathcal{F}_n\right] \\
        &= \xi(\beta)(W_{n-1}^{\beta})^2I_n
    \end{aligned}
    \]
\end{proof}

\begin{definition}
    We set $L:=\text{ess}\sup(\exp(\beta\omega -\lambda(\beta)))$ and we will have a.s.
    \[
    W_{n+1}^{\beta} \leq L W_n^{\beta}
    \]
\end{definition}

\begin{definition}
    (The $L^2$-regime)\par
    We know
    \[
    \lim_{n\to\infty} E (W_n^{\beta})^2 = E^{\otimes 2}\exp(\lambda_2(\beta) N(S,\tilde{S}))
    \]
    where $N$ is the intersection times between $S,\tilde{S}$.
\end{definition}

\begin{proposition}
    The limit above exists iff
    \[
    \exp(\lambda(2\beta) - 2\lambda(\beta)) < \dfrac{1}{E^{\otimes 2} N(S,\tilde{S})} + 1
    \]
\end{proposition}
\begin{proof}
    Notice $N$ is geometrically distributed and we may assume
    \[
    P^{\otimes 2}(N = k) = p^k(1-p)
    \]
    and then we have $E^{\otimes 2} N = p/(1-p)$. And 
    \[
    E^{\otimes 2}(\exp(\lambda_2 N)) = \sum\limits_{k=0}^{\infty}(\exp{\lambda_2})^kp^k(1-p)
    \]
    is finite iff $\exp(\lambda_2(\beta)) < 1/p = \dfrac{1}{E^{\otimes 2}N(S,\tilde{S})} + 1$.
\end{proof}

\begin{definition}
    We know $(W_n^{\beta})$ is bounded in $L^2$ iff $\beta < \beta_2$ where
    \[
    \beta_2 := \sup\{\beta: \xi(\beta)E^{\otimes 2}(N(S,\tilde{S})) < 1\} \in (0,\infty]
    \]
    so for any $\beta < \beta_2$, we know $W_n^{\beta}$ converges in $L^2$ and hence $W_{\infty} > 0$ and then $\beta \leq \beta_c$, so $\beta_2 \leq \beta_c$.
\end{definition}

\begin{theorem}(M.Birkner and R.Sun for $d\geq 4$, )
    In dimension $d\geq 3$, we have $\beta_c > \beta_2$.
\end{theorem}
$Remark$. To sum up, we have
\[\beta_{L^2} \leq \beta_c \leq \bar{\beta}_c\]

\begin{proposition}
    If there exists $n$ such that
    \[
    \mathbb{E}[\sqrt{W_n}] < (2n+1)^{-d}
    \]
    then very strong disorder holds.
\end{proposition}
\begin{proof}
    We have`
    \[
    f(\beta) = \lim_{m\to\infty} \dfrac{1}{nm}\mathbb{E}(\ln W_{nm}) \leq \limsup_{m\to\infty} \dfrac{2}{nm} \ln \mathbb{E}(\sqrt{W_{nm}})
    \]
    Given $x_1,\cdots,x_m \in\mathbb{Z}^d$ and define
    \[
    \widehat{W}_{nm}(x_1,\cdots,x_m) := E\exp(H_{nm}(\omega, X))\chi(X_{ni} = x_i, i\in[1,m])
    \]
    then we know
    \[
    \begin{aligned}
           \mathbb{E}\sqrt{W_{nm}} &= \mathbb{E} \sqrt{\sum\limits_{(x_1,\cdots,x_m)\in (\mathbb{Z}^d)^m} \widehat{W}_{nm}(x_1,\cdots,x_m)} \\
           &\leq \sum\limits_{(x_1,\cdots,x_m)\in (\mathbb{Z}^d)^m} \mathbb{E}\sqrt{\widehat{W}_{nm}(x_1,\cdots,x_m)} \\
           &=  \sum\limits_{(x_1,\cdots,x_m)\in (\mathbb{Z}^d)^m}\mathbb{E}\sqrt{E\prod_{i=1}^m \exp(\omega(ni,x_i))\chi(X_{ni} = x_i)} \\
           &= \sum\limits_{(x_1,\cdots,x_m)\in (\mathbb{Z}^d)^m}\mathbb{E}\sqrt{\widehat{W}_n(x_i-x_{i-1})} \\
           &= \left(\sum_x \mathbb{E}\sqrt{\widehat{W}_n(x)}\right)^m
    \end{aligned}
    \]
    and since $\widehat{W_n}(x) \leq W_n$ and we have
    \[
    \sum_x \mathbb{E}\sqrt{\widehat{W}_n(x)} = \sum_{|x|\leq n}\mathbb{E}\sqrt{\widehat{W}_n(x)} \leq (2n+1)^d \mathbb{E}\sqrt{W_n}
    \]
    and hence
    \[
    f(\beta) \leq \dfrac{2}{n}\ln\left(\sum_x \mathbb{E}\sqrt{\widehat{W}_n(x)}\right) \leq \dfrac{2}{n}\ln\left((2n+1)^d\mathbb{E}\sqrt{W_n}\right)
    \]
    and we are done.
\end{proof}

\begin{lemma}
    For any measurable event $A$, it holds that
    \[
    \mathbb{E}\sqrt{W_n^{\beta}} \leq \sqrt{\mathbb{P}(A)} + \sqrt{\tilde{\mathbb{P}}_n(A^c)} 
    \]
\end{lemma}
\begin{proof}
    For any measurable positive function $f$, we have
    \[
    \mathbb{E}(\sqrt{W_n^{\beta}})^2 \leq \mathbb{E}(W_n^{\beta}f(\omega))\mathbb{E}(f(\omega)^{-1}) = \widetilde{\mathbb{E}}_n(f(\omega))\mathbb{E}(f(\omega)^{-1})
    \]
    and we consider $f$ to be some $\alpha \chi(A) + \alpha^{-1}\chi(A^c)$ and we will have
    \[
    \begin{aligned}
        \widetilde{E_n}(f(\omega))\mathbb{E}(f(\omega)) &= (\alpha \widetilde{\mathbb{P}}_n(A) + \alpha^{-1}\widetilde{\mathbb{P}}_n(A^c))(\alpha^{-1}\mathbb{P}_n(A) + \alpha\mathbb{P}(A^c)) \\
        &\leq (\alpha + \alpha^{-1}\widetilde{\mathbb{P}}_n(A^c))(\alpha^{-1}\mathbb{P}_n(A) + \alpha) \\
        &= \alpha^2 +\alpha^{-2}\widetilde{\mathbb{P}}_n(A^c)\mathbb{P}_n(A) + \mathbb{P}_n(A) + \widetilde{\mathbb{P}}_n(A^c)
    \end{aligned}
    \]
    and let $\alpha = (\widetilde{\mathbb{P}}_n(A^c)\mathbb{P}(A))^{1/4}$, we will have
    \[
    \widetilde{E_n}(f(\omega))\mathbb{E}(f(\omega)) \leq \left(\sqrt{\mathbb{P}(A)}+\sqrt{\widetilde{\mathbb{P}}_n(A^c)}\right)^2
    \]
    and we are done.
\end{proof}


\begin{definition}
    (Size-Biased measure)\par
    \[
    \widetilde{\mathbb{P}}_n(d\omega):= W_n^{\beta}\mathbb{P}(d\omega)
    \]
\end{definition}

\begin{definition}
    (Size-biased environment)\par
    Define $(\widehat{\omega}_i)$ a sequence of i.i.d random variable with distribution given by
    \[
    \widehat{\mathbb{P}}(\hat{\omega} \in \cdot) = \mathbb{E}(\exp(\beta\omega - \lambda)\chi(\omega \in \cdot))
    \]
    and with a random walk $X$ we may define $\widetilde{\omega} = \widetilde{\omega}(X,\omega,\hat{\omega})$ by
    \[
    \tilde{\omega}_{i,x} := \begin{cases}
        \omega_{i,x}\quad&\text{if }x\neq X_i,\\
        \hat{\omega}_{i,x}&\text{if }x=X_i
    \end{cases}
    \]
\end{definition}

\begin{lemma}
    It holds that
    \[
    \widetilde{P}_n((\omega_{i,x})_{i\in[1,n],x\in\mathbb{Z}^d}\in \cdot) = P\otimes \mathbb{P}\otimes\widehat{\mathbb{P}}((\widetilde{\omega}_{i,x})_{i\in[1,n],x\in\mathbb{Z^d}}\in \cdot)
    \]
    which means for any bounded measurable $f:\mathbb{R}^{\mathbb{N}\times \mathbb{Z}^d}\to \mathbb{R}$ we have
    \[
    \widetilde{E}_n f(\omega) = E\otimes \mathbb{E}\otimes\widehat{\mathbb{E}} f(\widetilde{\omega})
    \]
\end{lemma}
\begin{proof}
    Given a bounded measurable function $f$ which depends only on the first $n$ time environments and then
    \[
    \widetilde{\mathbb{P}}_n(f(\omega)) = \mathbb{E}(W_n^{\beta}f(\omega)) = E\mathbb{E}(\exp(\beta H_n(\omega,X) - \lambda(\beta)f(\omega)))
    \]
    by Fubini. Let $\tilde{P}_{n,X}(\omega) = \exp(\beta H_n(\omega,X) - \lambda(\beta))\mathbb{P}(d\omega)$ and we have
    \[
    \widetilde{\mathbb{P}}(f(\omega)) = E\tilde{\mathbb{E}}_{n,X}(f(\omega))
    \]
    and the distribution of $(\omega_{i,X_i})$ is given by $\hat{\mathbb{P}}$ and the original one for the rest. Extend the RHS from finite dimension to the all dimensions.
\end{proof}
\newpage
\subsection{Main results}

\begin{proposition}
    Assume that strong disorder holds. Then there exist $C>0, n_0\in\mathbb{N}$ such that for all $n\geq n_0$ there exists $s = s_n \in [0,C\ln n]$ such that
    \[
    A_n:= \{\exists (m,y)\in[0,n]\times[-n,n]^d, \theta_{m,y} W_s^{\beta} \geq n^{4d}\}
    \]
    it holds that
    \[
    \mathbb{P}(A_n) \leq \dfrac{1}{8(2n+1)^{2d}},\quad \tilde{\mathbb{P}}_n(A_n^c) \leq \dfrac{1}{8(2n+1)^{2d}}
    \]
\end{proposition}

\begin{theorem}
    For any $d\geq 3$, if the environment is bounded from above, then strong disorder and very strong disorder are equivlent. That is \begin{enumerate}
        \item $\beta_c = \bar{\beta}_c$
        \item $W_{\infty}^{\beta_c} > 0$ for $\mathbb{P}$ a.s.
    \end{enumerate}
\end{theorem}
\begin{proof}
    We have if the strong disorder holds, then
    \[
    \mathbb{E}(\sqrt{W_n^{\beta}}) \leq \sqrt{\mathbb{P}(A_n)} + \sqrt{\tilde{\mathbb{P}}_n(A_n^c)} \leq \dfrac{1}{\sqrt{2}(2n+1)^d} 
    \]
    and hence by the propostion 7.2.1, we have for any $\beta > \beta_c$, the very strong disorder holds and hence $\beta_c = \bar{\beta}_c$. If $W_{\infty}^{\beta_c} = 0$, then the very strong disorder holds and which contradiction that $f(\beta_c) = 0$.
\end{proof}

\subsection{Proof of proposition 7.2.1}

\begin{proposition}
    If strong disorder holds then for any $\epsilon > 0$ there exists $C(\epsilon), u_0(\epsilon) > 0$ such that every $u\geq u_0$, there exists $s\in [0,C\ln u]$ such that
    \[
    \mathbb{P}(W_s^{\beta} \geq u) \geq u^{-(1+\epsilon)}
    \]
\end{proposition}

Let $u = n^{4d}$ and $\epsilon = 1/(12d)$ and we consier $s\in[0,4C\ln n]$, which is such that
\[
\mathbb{P}(W_s^{\beta} \geq n^{4d}) \geq n^{-4d(1+\epsilon)}
\]
which means
\[
\tilde{\mathbb{P}}_s(W_s^{\beta} \geq n^{4d}) = \mathbb{E}(W_s^{\beta} \chi(W_s^{\beta} \geq n^{4d})) \geq n^{4d}\mathbb{P}(W_s^{\beta} \geq n^{4d}) \geq n^{-4d\epsilon} = n^{-1/3}
\]
and we introduce
\[
A_{m,n}:= \{\max_{x\in[-n,n]^d}\theta_{m-s,x}W_s^{\beta} \geq n^{4d}\}
\]

\begin{lemma}
    For any $m \in [s,n]$, we have a.s.
    \[
    \tilde{\mathbb{P}}_n(A_{m,n}|\mathcal{F}_{m-s}) \geq n^{-1/3}
    \]
\end{lemma}

With this lemma, we know
\[
\tilde{\mathbb{P}}_n(\bigcap_{i=1}^j A_{is,n}^c) = \tilde{\mathbb{P}}_n(\bigcap_{i=1}^j A_{is,n}^c|\mathbb{F}_{(j-1)_s}) \leq (1-n^{-1/3})\tilde{\mathbb{P}}_n(\bigcap_{i=1}^{j-1} A_{is,n}^c) \leq (1-n^{-1/3})^j
\]
for any $js\leq m$.\par
Since $A_n = \cup_{m=s}^n A_{m,n}^c$, we have
\[
\tilde{\mathbb{P}}_n(A_n^c) \leq \tilde{P_n}(\cap_{i=1}^{[n/s]}A_{is,n}^c) \leq (1-n^{1/3})^{[n/s]} \leq \exp(-[n/s]n^{-1/3}) \leq e^{-n^{1/2}}
\]
for $n$ sufficently large, and then choose $n$ sufficently large such that $e^{-n^{1/2}} \leq \dfrac{1}{8(2n+1)^{2d}}$.\par
For $\mathbb{P}(A_n)$, we have
\[
\begin{aligned}
    \mathbb{P}(A_n) &\leq \sum\limits_{(m,y) \in [0,n]\times[-n,n]^d} \mathbb{P}(\theta_{m,y}W_s^{\beta} \geq n^{4d}) \\ &= (n+1)(2n+1)^d\mathbb{P}(W_s^{\beta} \geq n^{4d}) \\  &\leq (n+1)n^{-4d}(2n+1)^d
\end{aligned}
\]
and let $n$ large enought such that
\[
(n+1)n^{-4d}(n+1/2)^{3d} \leq 1
\]
we will have the bound for $\mathbb{P}(A_n)$ in proposition 7.2.1.\par
\vspace{0.5em}

Now we begin the proof of \textbf{lemma 7.3.2}.\par
\begin{proof}
    Let $(\mathcal{G}_n)$ denote the natural filtration of $(\omega,\hat{\omega},X)$. We let $\tilde{W}_s^{\beta}$ the partition function constructed from $\tilde{\omega}$ and then $\widetilde{\mathbb{P}}_s(A_{m,n})$
    \[
    P\otimes\mathbb{P}\otimes\widehat{\mathbb{P}}(\tilde{\omega}\in A_{m,n}|\mathcal{G}_{m-s}) \geq P\otimes\mathbb{P}\otimes\widehat{\mathbb{P}}(\theta_{m-s,X_{m-s}}\widetilde{W}_s^{\beta}\geq n^{4d}|\mathcal{G}_{m-s})
    \]
    Notice \[\theta_{k,X_k}\tilde{\omega}(m,x) = \tilde{\omega}(m+k,x+X_k) = \begin{cases}
        \hat{\omega}(m+k,x+X_k)\quad&\text{if }X_{m+k} - X_k= x\\
        \omega(m+k,x+X_k)&\text{if }X_{m+k} - X_k\neq x
    \end{cases}\]
    which will have the same distribution with $\tilde{\omega}$ and hence
    \[
    \begin{aligned}
        P\otimes\mathbb{P}\otimes\widehat{\mathbb{P}}(\theta_{m-s},X_{m-s}\widetilde{W}_s^{\beta}\geq n^{4d}|\mathcal{G}_{m-s}) &= P\otimes\mathbb{P}\otimes\widehat{\mathbb{P}}(\widetilde{W}_s^{\beta}\geq n^{4d}|\mathcal{G}_{m-s}) \\
        &=\widetilde{\mathbb{P}}_s(W_s^{\beta} \geq n^{4d}) \geq n^{-1/3}
    \end{aligned}
    \]
\end{proof}

\subsection{Proof of proposition 7.3.1}

\begin{theorem}
    If strong disorder holds, then there exists $c>0$ such that for all $u\geq 1$
    \[
    \mathbb{P}(\max_{n\geq 0, x\in\mathbb{Z}^d}\widehat{W}_n^{\beta}(x) \geq u) \geq \dfrac{c}{u}
    \]
\end{theorem}

\begin{proposition}
    If strong disorder holds. For $\epsilon > 0$, there exist $C(\epsilon), u_0(\epsilon) > 0$ such that for all $u\geq u_0$
    \[
    \mathbb{P}(\max_{(n,x)\in [0,C\ln u]\times\mathbb{Z}^d}\widehat{W}_n^{\beta}(x) \geq u) \geq u^{-(1+\epsilon)}
    \]
\end{proposition}

Then with the above proposition, we will know
\[
 \mathbb{P}(\max_{(n,x)\in [0,C\ln u]\times\mathbb{Z}^d}\widehat{W}_n^{\beta}(x) \geq u) \leq \mathbb{P}(\max_{n\in[0,C\ln u]} W_n^{\beta}) \leq C\ln u \max_{n\in [0,C\ln u]}\mathbb{P}(W_n^{\beta}\geq u)
\]
then we have for all $u\geq u_0$ we have there is some $s\in[0,C\ln u]$ such that
\[
\mathbb{P}(W_s^{\beta} \geq u) \geq u^{-1+\epsilon}/(C\ln u)
\]
and we may let $\epsilon' = \epsilon/2$ and then there is some $u_0'(\epsilon')$ such that $C\ln u \leq u^{\epsilon'}$ for $u\geq u_0(\epsilon')$ and we will have
\[
\mathbb{P}(W_s^{\beta} \geq u) \geq u^{-1+\epsilon}/(C\ln u) \geq u^{-1+\epsilon'}
\]
for all $u\geq u_0(\epsilon')$ where $C(\epsilon) = C(\epsilon')$.