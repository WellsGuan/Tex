\section{The Localized Phase}

\subsection{Checklist}
\begin{itemize}
    \item Prove the integration by parts for Gaussian.
\end{itemize}

\subsection{Useful Conclusions}

\begin{lemma}
    (Integration by Part)\par
    If $X$ is centered normal, and $f$ is smooth with
    \[
    \lim_{|x|\to\infty} f(x)\exp(-x^2/(2EX^2)) = 0,
    \]
    then
    \[E(Xf(X)) = E(X^2)E(f'(X))\]
\end{lemma}

\begin{corollary}
    (Integration by Part for Gaussian Vectors)\par If $(X,X-1,\cdots,X_n)$ is a centered, gaussian vector , and $F$ is smooth with
    \[
    \lim_{||x||\to\infty} F(x)\exp(-ax^2) = 0
    \]
    for all $a>0$, then
    \[
    E(XF(X_1,\cdots,X_n)) = \sum\limits_{i=1}^n E(XX_i)E(F_{x_i}(X_1,\cdots,X_n))
    \]
\end{corollary}

\begin{theorem}
    (Chernoff's bound)\par
    For a r.v., assume all the required moment exist, then
    \[
    P(X\geq a) \leq \inf_{t>0}E(e^{tX})e^{-ta}
    \]
\end{theorem}

\begin{definition}
    (Legendre-Fenchel Transform)\par
    Consider a function $f:\mathbb{R}\to\mathbb{R}$, we define
    \[
    f^*(k) = \sup_{x\in \mathbb{R}}(kx-f(x))
    \]
\end{definition}

\begin{definition}
    (Supporting Line)\par
    We call $f:\mathbb{R}\to\mathbb{R}$ has a supporting line at $x$ if there exists $\alpha \in \mathbb{R}$
\end{definition}

\begin{proposition}
    
\end{proposition}

\subsection{Path Localization}

\begin{definition}
    In this chapter, we consider Gaussian environment
    \[
    \omega(t,x) \sim \mathcal{N}(0,1)
    \]
    and for $y:\mathbb{N}\to\mathbb{Z}^d$ and $S$ a path, we define
    \[
    N_n(S,y) = \sum\limits_{t=1}^n \chi_{\{S_t = y_t\}}
    \]
    and
    \[\mathcal{F} = \{\beta > 0, p\text{ is differentiable at }\beta,p'(\beta) < \lambda(\beta)\}\]
\end{definition}

\begin{theorem}
    Assume that the environment is Gaussian. There exists $y^{(n)}:[0,n] \to\mathbb{Z}^d$ such that
    \[
    \liminf_{n\to\infty} \mathbb{E}E_n^{\beta,\omega}\left(\dfrac{N_n(S,y^{(n)})}{n}\right) \geq 1 - \dfrac{p'}{\lambda'}(\beta) > 0
    \]
    for all $\beta \in \mathcal{F}$. Moreover,
    \[
    \lim_{\beta\to\infty} \liminf_{n\to\infty} \mathbb{E}E_n^{\beta,\omega}\left(\dfrac{N_n(S,y^{(n)})}{n}\right) = 1
    \]
\end{theorem}
\begin{proof}
    We have
    \[
    \begin{aligned}
    \dfrac{d}{d\beta}\mathbb{E}p_n(\omega,\beta) &= \dfrac{1}{n}\dfrac{d}{d\beta}\mathbb{E}\ln\left(\sum\limits_{x}P(S^{(n)} = x) \exp(\beta H_n(x))\right) \\
    & = \dfrac{1}{n}\mathbb{E}\dfrac{1}{Z_n}\sum\limits_{x}P(S^{(n)} = x)\exp(\beta H_n(x))H_n(x) \\
    &= \dfrac{1}{n}\sum\limits_{t\leq n,x}\mathbb{E}\left(P_n^{\beta,\omega}(S_t = x)\omega(t,x)\right)
    \end{aligned}
    \]
    which has a uniform $L_1$ bound.\par
    Let \[\begin{aligned}F(\omega(t,x))_{t\leq n, x} &= P_n^{\beta,\omega}(S_t = x) \\
    &= \sum\limits_{S_t = x} P(S)\exp(\beta\sum\limits_{i=1}^n \omega(i,S_i))
    \end{aligned}\]
    and
    \[
    \lim_{x\to\infty} \exp(\beta \sum\limits_{i=1}^n x_{i,S_i} - a|x|^2) \leq \lim_{x\to\infty}\exp(|x|(\beta n-a|x|)) = 0
    \]
    which means we may use the integration by parts. Notice
    \[
    \begin{aligned}
    P_n^{\beta,\omega}(S_t = x) &= \dfrac{1}{Z_n}\sum\limits_{x, x_t = x}P(S^{(n)} = x)\exp(\beta H_n(x)) \\ &= \dfrac{Z_nP_n^{\beta,\omega}(S_t = x)}{Z_nP_n^{\beta,\omega}(S_t = x)+(Z_n - Z_nP_n^{\beta,\omega}(S_t = x))} 
    \end{aligned}
    \]
    and hence
    \[
    \begin{aligned}
        \dfrac{dP_n^{\beta,\omega}(S_t = x)}{d\beta} &= \dfrac{1}{Z_n^2}\left(\beta Z_n P_n^{\beta,\omega}(S_t = x)Z_n - \beta Z_n^2P_n^{\beta,\omega}(S_t = x)^2\right) \\
        & = \beta(P_n^{\beta,\omega}(S_t = x)-P_n^{\beta,\omega}(S_t = x)^2)
    \end{aligned}
    \]
    so we have
    \[
    \begin{aligned}
    \dfrac{d}{d\beta}\mathbb{E}p_n(\omega,\beta) &= \dfrac{\beta}{n} \sum\limits_{t\leq n, x} \mathbb{E}(P_n^{\beta,\omega}(S_t = x)-P_n^{\beta,\omega}(S_t = x)^2) \\
    &=\beta\dfrac{1}{n}\mathbb{E}(\sum\limits_{t\leq n} {P_n^{\beta,\omega}}^{\otimes 2}(S,\tilde{S}\text{ do not coincide at time }t)) \\
    & = \beta(1 - \dfrac{1}{n}\mathbb{E}\sum\limits_{t\leq n}{ P_n^{\beta,\omega}}^{\otimes 2}(S,\tilde{S}\text{ coincide at time }t))   
    \end{aligned}
    \]
    and consider how many contributions to ${P_n^{\beta,\omega}}^{\otimes 2}(S,\tilde{S}\text{ coincide at }t_1,\cdots,t_k)$ is $k$ times and hence
    \[
    \beta\left(1 - \mathbb{E}{ E_n^{\beta,\omega}}^{\otimes 2}\left(\dfrac{N_n(S,\tilde{S})}{n}\right)\right)
    \]
    Since $\lambda(\beta) = \beta^2/2$ and we have
    \[
    \lim_{n\to\infty}\mathbb{E}{E_n^{\beta,\omega}}^{\otimes 2}\left(\dfrac{N_n(S,\tilde{S})}{n}\right) = 1 - p'(\beta)/\beta = 1 -\dfrac{p'}{\lambda '}(\beta)
    \]
    For fixed $n,\beta,\omega$, we may define
    \[
    y^{(n)}(t) = arg\max_{x} P_n^{\beta,\omega}(S_t = x)
    \]
    and then
    \[
    {P_n^{\beta,\omega}}^{\otimes 2}(S_t = \tilde{S}_t) \leq P_n^{\beta,\omega}(S_t = y_t^{(n)})
    \]
    and then
    \[
    \mathbb{E}{E_n^{\beta,\omega}}^{\otimes 2}\left(\dfrac{N_n(S,\tilde{S})}{n}\right) \leq \mathbb{E}P_n^{\beta,\omega}\left(\dfrac{N_n(S,y^{(n)})}{n}\right)
    \]
    Recall that $p$ is convex and $p$ is almost linear by
    \[
    p(\beta) \leq \beta\inf_{b\in(0,\beta]}\dfrac{\lambda(b)+\ln(2c)}{b} - \ln(2d)
    \]
    and hence $p'(\beta)$ should be bounded when $\beta$ is large and hence
    \[
    1\geq \lim_{\beta\to\infty} \liminf_{n\to\infty} \mathbb{E}E_n^{\beta,\omega}\left(\dfrac{N_n(S,y^{(n)})}{n}\right) \geq \lim_{\beta\to \infty} 1 - \dfrac{C}{\beta}
    \]
    for some constant $C$.
\end{proof}

\begin{theorem}
    Assuem $d = 1$ or $d=2$. Then for all $\beta \neq 0, p(\beta) < \lambda(\beta)$ and therefore $W_{\infty} = 0$.
\end{theorem}
\begin{proof}
    Notice for all $z\in \mathbb{Z}^d$, we have
    \[
    {P_{t-1}^{\beta,\omega}}^{\otimes 2}(S_t = \tilde{S_t}+z) = \sum\limits_{x} P_{t-1}^{\beta,\omega}(S_t= x)P_{t-1}^{\beta,\omega}(S_t= x+z) \leq {P_{t-1}^{\beta,\omega}}^{\otimes 2}(S_t = \tilde{S_t}) = I_t 
    \]
    Notice when $d=1$,
    \[
    1 = \sum_{z, 2|z,|z| \leq 2t} {P_{t-1}^{\beta,\omega}}^{\otimes 2}(S_t = \tilde{S_t}+z) \leq 2t+1 I_t
    \]
    and hence $\sum I_t$ diverge, by
    \[
    -\ln W_n \sim \sum I_t
    \]
    and we know $W_{\infty} = 0$.\par
    For $d=2$, if $W_{\infty} > 0$ a.s., then let
    \[
    A_n = \{|S_n^{(1)}|\leq K\sqrt{n\ln n},|S_n^{(2)}|\leq K\sqrt{n\ln n}\}
    \]
    and
    \[
    X_n = E(\exp(\beta H_{n-1} - (n-1)\lambda(\beta)); A_n^c).
    \]
    Then
    \[
    \mathbb{P}\left(X_n \geq \exp(-K^2n\ln n/4)\right) \leq e^{K^2 n\ln n /4} \mathbb{E}(X_n) = e^{K^2 n\ln n /4}P(A_n^c),
    \]
    and by the Chernov's bound:
    \[
    \begin{aligned}
        P(\pm S_n^{(1)} > K\sqrt{n\ln n}) &\leq \inf_{t>0} E(\exp(tS_n^{(1)}))e^{-tK(\sqrt{n\ln n})} \\
        &=\exp\left[\inf_{t\geq 0} \left(\ln E\exp(tS_n^{(1)}) - tK(\sqrt{n\ln n})\right)\right] \\
        &=\exp\left(- \gamma^*(K\sqrt{n\ln n})\right)
    \end{aligned}
    \]
    and $\gamma^*$ is the LF transform of
    \[
    \gamma(u):=\ln E\exp(uS_n^{(1)}) = \ln\dfrac{1+\cosh u}{2} \leq u^2/2
    \]
    and hence
    \[
    \gamma^*(v) \geq v^2/2
    \]
    and then
    \[
    P(A_n^c) \leq \exp(-K^2 n\ln n)
    \]
    and
    \[
    \mathbb{P}(X_n \geq \exp(-K^2 n \ln n /4)) \leq e^{-3K^2 n\ln n/4}.
    \]
    By BC-lemma, we have $X_n \to 0$ $\mathbb{P}$-a.s., which implies
    \[
    Y_n:= P_{n-1}^{\beta,\omega}(A_n^c) = X_n/W_n \to 0/W_{\infty} = 0
    \]
    for $\mathbb{P}$-a.s. Denote $\mathcal{C}(n,K) = [-K\sqrt{n\ln n}, K\sqrt{n\ln n}]^2$ and
    \[
    \begin{aligned}
        (1-Y_n)^2 &= \sum\limits_{x,y\in \mathcal{C}(n,K)}{P_{n-1}^{\beta,\omega}}^{\otimes 2}(S_n = x, \tilde{S}_n = y) \\
        &\leq \sum\limits_{z\in \mathcal{C}(n,2K)}{P_{n-1}^{\beta,\omega}}^{\otimes 2}(S_n = \tilde{S}_n + z) \\
        &\leq (4K\sqrt{n\ln n})^2I_n
    \end{aligned}
    \]
    and hence $\sum I_n$ diverge and we are done.
\end{proof}

\begin{theorem}
    Assume $\omega(t,x)$ has mean $0$ and variance $1$. For $d = 1$, as $\beta \to 0$ we have
    \[
    \lambda(\beta) - p(\beta) = O(\beta^4)
    \]
    and $d= 2, \beta \to 0$, we have
    \[\lambda(\beta) - p(\beta) = \exp(-\pi\beta^{-2}(1+o(1)))\]
\end{theorem}

