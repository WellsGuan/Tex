%!TEX program = xelatex
\documentclass[lang=en,11pt,a4paper,citestyle =authoryear]{elegantpaper}

% 标题
\title{Homework01 - MATH 725}
\author{Boren(Wells) Guan}

% 本文档命令
\usepackage{array,url,stix}
\usepackage{subfigure}
\newcommand{\ccr}[1]{\makecell{{\color{#1}\rule{1cm}{1cm}}}}
\newcommand{\code}[1]{\lstinline{#1}}
\newcommand{\prvd}{$\hfill \qedsymbol$}
\newcommand{\Z}{\mathbb{Z}}
\newcommand{\R}{\mathbb{R}}
\newcommand{\N}{\mathbb{N}}
\newcommand{\C}{\mathbb{C}}
\newcommand{\Q}{\mathbb{Q}}
\newcommand{\M}{\mathcal{M}}
\newcommand{\B}{\mathcal{B}}
\newcommand{\X}{\mathcal{X}}
\newcommand{\Hil}{\mathcal{H}}
\newcommand{\range}{\mathcal{R}}
\newcommand{\nul}{\mathcal{N}}

% 文档区
\begin{document}

% 标题
\maketitle

\subsection*{Before Reading:}\par
To make the proof more readable, I will miss or gap some natural or not important facts or notations during my writing. If you feel it hard to see, you can refer the appendix after the proof, where I will try to explain some simple conclusions (will be marked) more clearly. In case that you misunderstand the mark, I will add the mark just after those formulas between \$ and before those between \$\$.\par
And I have to claim that the appendix is of course a part of my assignment, so the reference of it is required. Enjoy your grading!

\subsection*{Section 5.5 Ex.54} 
For any nonempty set $A$, $l^2(A)$ is complete.
\vspace{0.5em}\\
\textbf{Sol.} \par
It is sufficient to show any absolutely convergent series $\sum\limits_{i=1}^{\infty} f_i$ converges and we assume $f = \sum\limits_{i=1}^{\infty} f_i$, where we know
\[||\sum\limits_{i=1}^{\infty}|f_i|||_{l^2(A)}\leq \sum\limits_{i=1}^{\infty} ||f_i||_{l^2(A)} < \infty\]
by Minkovski's inequality and hence $\sum\limits_{i=1}^{\infty} |f_i|$ exists a.e. on $(A,\mathcal{P}(A),\mu)$ where $\mu$ is a counting meaure, which means $f(\alpha)$ exists for each $\alpha \in A$. Then we denote $s_n = \sum\limits_{i=1}^n f_i$ and we know for any $\epsilon > 0$, there exists an integer $N$ such that for any $n \geq N$, $\sum\limits_{i=n}^{n+m}||f_i||_{l^2(A)}<\epsilon/2$ for any integer $m$ and hence for any finite subset $F\subset A$
\[\Big(\sum_{\alpha \in F} (s_n-f)^2(\alpha)\Big)^{\tfrac{1}{2}} \leq \Big(\sum_{\alpha \in F} (s_n-s_{n+m})^2(\alpha)\Big)^{\tfrac{1}{2}}+\Big(\sum_{\alpha \in F} (s_{n+m}-f)^2(\alpha)\Big)^{\tfrac{1}{2}} < \epsilon/2 + \epsilon/2 = \epsilon\]
for $n\geq N$ and $m$ large enough since $s_{n+m} \to f, m \to\infty$. Therefore, 
\[||s_n-f||_{l^2(A)} = \Big(\sum\limits_{\alpha \in A}(s_n-f)^2(\alpha)\Big)^{\tfrac{1}{2}} \to 0, n\to\infty,\]
which means $\sum\limits_{i=1}^{\infty}$ converges to $f$ in $l^2(A)$.
\prvd
\vspace{0.5em}

\subsection*{Section 5.5 Ex.56} 
If $E$ is a subset of a Hilbert space $\Hil$, $(E^{\perp})^{\perp}$ is the smallest closed subspace of $\Hil$ containing $E$.
\vspace{0.5em}\\
\textbf{Sol.} \par
Consider $\M$ a closed subspace of $\Hil$ containing $E$, then we know 
\[\Hil = M \oplus M^{\perp}\]
by the theorem 5.24 on Folland and $\M^{\perp} \subset E^{\perp}$ since for any $x\in \M^{\perp}, y\in \M$, we know $\langle x,y\rangle = 0$ and hence $x\in E^{\perp}$. Then we know $(E^{\perp})^{\perp} \subset (\M^{\perp})^{\perp}$ similarly, and we know $(\M^{\perp})^{\perp} = \M$ since if $x= y+z \in \Hil, y\in \M, z\in\M^{\perp}, x\in (\M^{\perp})^{\perp}$, we know
\[\langle z,z\rangle = \langle x-y,z\rangle = \langle x,z\rangle + \langle y,z\rangle = 0,\]
which means $x = y \in \M$, in particular, $(\M^{\perp})^{\perp} \subset \M$ and hence $(\M^{\perp})^{\perp} = \M$.\par
Then notice for any subset $A\subset\Hil$, $A^{\perp}$ is a closed subspace of $\Hil$ since
\[
\begin{aligned}
\langle x+\lambda y,z\rangle &= \langle x,z\rangle +\lambda\langle y,z\rangle = 0 \\
<r,z> &= \lim <r_{i}, z> = 0 \\
\end{aligned}
\] 
for any $x,y,r\in A^{\perp}, \lambda \in K, z\in \{r_{i}\}_{i\in I}\subset A$ a net converges to $r$ by proposition 5.21. Therefore, $(E^{\perp})^{\perp}$ is the smallest closed subspace of $\Hil$ containing $E$.
\prvd
\vspace{0.5em}

\subsection*{Section 5.5 Ex.57} 
Suppose that $\Hil$ is a Hilbert space and $T\in L(\Hil,\Hil)$.\par
a. There is a unique $T^* \in L(\Hil,\Hil)$ called the adjoint of $T$, such that $\langle Tx,y\rangle = \langle x, T^*y\rangle$ for all $x,y\in\Hil$.\par
b. $||T^*|| = ||T||, ||T^*T|| = ||T||^2, (aS+bT)^* = \overline{a}S^* + \overline{b}T^{*}, (ST)^* = T^*S^*$ and $T^{**} = T$.\par
c. Let $\mathcal{R}$ and $\mathcal{N}$ denote range and nullspace; then $\mathcal{R}(T)^{\perp} = \mathcal{N}(T^*)$ and $\mathcal{N}(T)^{\perp} = \overline{\mathcal{R}(T^*)}$.\par
d. $T$ is unitary iff $T$ is invertible and $T^{-1} = T^*$.
\vspace{0.5em}\\
\textbf{Sol.} \par
a. Firstly, notice that $<Tx,y>\leq ||Tx||||y|| \leq ||T||||y||||x||$ and
\[\langle T(x+\lambda z), y\rangle = \langle Tx+\lambda Tz, y\rangle = \langle Tx,y\rangle + \lambda \langle Tz,y\rangle\]
and hence $x\mapsto \langle Tx,y\rangle \in \Hil^*$ and hence there exists an element $u_y \in \Hil$ such that $\langle Tx, y\rangle = \langle x, u_y\rangle$ for all $x\in \Hil$ by thorem 5.2.5 Define $T^* y = u_y$ and then we know $T^* (x+\lambda y)$ satisfies that
\[\langle z, T^*(x+\lambda y)\rangle = \langle Tz,x\rangle + \overline{\lambda}\langle Tz,y\rangle = \langle z,u_x\rangle + \langle z,\lambda u_y\rangle = \langle z,(T^*x+\lambda T^* y)\rangle\]
for all $z\in \Hil$ and hence $T^*(x+\lambda y) = T^*x+\lambda T^*y$ for any $x,y\in\Hil$, which can be proved easily by taking $z = T^*(x+\lambda y)-(T^*x+\lambda T^*y)$. And we know
\[\langle Tu_x,x\rangle = ||u_x||^2 \leq ||T|||u_x|||x||\]
which means $||u_x|| \leq ||T|||x||$ and hence $T^* \in L(\Hil,\Hil)$.\par
If there exists $T'$ satifies the equality, we know $\langle x, T^* y\rangle = <Tx , y> = \langle x, T' y\rangle$ for any $x,y\in\Hil$ and we may know $T'=T^*$ easily by taking $x = T^*y-T'y$, which means $T^*$ is unique.\par
b. We have already proved that $||T^*|| \leq ||T||$ and notice
\[||Tx||^2 = \langle x,T*Tx\rangle \leq ||x||||T^*||||Tx|| \]
which implies that $||T||\leq ||T^*||$ and hence $||T||=||T^*||$.\par
Then we notice that
\[\begin{aligned}
    ||T^*Tx, T^*Tx||^2 &\leq ||T||\ ||x||\ ||T||\ ||T^*Tx|| = ||T^*||\ ||T||\ ||x||\ ||T^*Tx|| \\
    ||Tx||^2 &\leq ||x||\ ||T^*T||\ ||x|| \\ 
\end{aligned}\]
and hence $||T||||T^*|| = ||T||^2 = ||T^*T||$.\par
Then by the conclusion in (a), notice
\[
\begin{aligned}
\langle x, (\overline{a}S^* +\overline{b}T^*) y\rangle &= a \langle Sx,y\rangle + b\langle Tx,y\rangle = \langle (aS+bT)x,y\rangle \\
\langle x, T^*S^* y \rangle &= \langle Sx,T^*y\rangle = \langle TSx,y\rangle \\
\langle T^* x,y\rangle &= \overline{\langle y, T^*x} = \langle x,Ty\rangle
\end{aligned}
\]
and hence we know $(aS+bT)^* = \overline{a}S^* + \overline{b}T^{*}, (ST)^* = T^*S^*$ and $T^{**} = T$.\par
c. For $x\in \nul(T^*)$, we know $\langle Ty,x\rangle = \langle y,T^*x\rangle = 0$ for any $y\in\Hil$ and hence $\nul(T^*)\subset \rangle(T)^{\perp}$. Similarly, we know for any $x\in \rangle(T)^{\perp}$, $0 = \langle Ty,x\rangle = \langle y,T^*x \rangle$ for any $y\in\Hil$ and we know $T^* x =0$ by taking $y= T^*x$ and hence $\rangle(T)^{\perp}\subset\nul(T^*)$ and hence $\rangle(T)^{\perp} = \nul(T^*)$.\par
Then we know $\nul(T)^{\perp} = (\range(T^*)^{\perp})^{\perp} \supset \overline{\range(T^*)}$ since $(\range(T^*)^{\perp})^{\perp}$ is a closed subspace of $\Hil$ containing $\range(T^*)$ by the conclusion in Section 5.5 Ex.56. Also we know $(\range(T^*)^{\perp})^{\perp} \subset \overline{\range(T^*)}$ by the conclusion in Sec 5.5 Ex.56 since $\overline{\range(T^*)}$ is a closed subspace, which can be verified like this: for any $x\in \overline{\range(T^*)}$, there exists $\{x_i\}_{i\in I}\subset \range(T^*)$ converges to $x$ in $\Hil$, where $x_i = x_j$ can be true for $i\neq j, i,j\in I$, and we may know $x_i+\lambda y_i \to x+\lambda y$ for any $x,y\in\Hil,\lambda\in K$.\par
d. Notice that if $T$ then we know for any $x,y\in\Hil$,
\[\langle x, T^*Ty\rangle = \langle Tx,Ty\rangle = \langle x,y\rangle\]
and hence $T^*T = id_{\Hil}$ which can be verified by taking $x = T^*Ty-y$ and hence $T^* = T^*TT^{-1} = T^{-1}$.\par
If $T$ is invertible and $T^{-1}T^*$, then we know $\langle Tx,Ty\rangle = \langle x, T^*Ty\rangle = \langle x,y\rangle$ and hence $T$ is unitary.
\vspace{0.5em}

\subsection*{Section 5.5 Ex.58} 
Let $\M$ be a closed subspace pf the Hilbert space $\Hil$, and for $x\in\Hil$ let $Px$ be the element of $\M$ such that $x-Px\in \M^{\perp}$ as in Theorem 5.24.\par
a. $P\in L(\Hil,\Hil)$, and in the notation of Ex.57 we have $P^* = P, P^2 = P,\range(P) = \M$ and $\nul(P) = \M^{\perp}$. $P$ is called the orthogonal projection onto $\M$.\par
b. Conversely, suppose that $P\in L(\Hil,\Hil)$ satisfies $P^2 = P^* = P$. Then $\range(P)$ is closed and $P$ is the orthogonal projection onto $\range(P)$.\par
c. If $\{u_{\alpha}\}$ is an orthonormal basis for $\M$, then $Px = \sum\langle x,u_{\alpha}\rangle u_{\alpha}$.
\vspace{0.5em}\\
\textbf{Sol.} \par
a. We know for any $x,y\in \Hil$,
\[\langle Px, y \rangle  = \langle Px, y+(Py-y)\rangle = \langle Px,Py\rangle = \langle x-(x-Px),Py\rangle = \langle x, Py\rangle\]
and hence $P^* = P$ by the conclusion in Sec 5.5 Ex.57.\par
Then we know for any $x,y\in \Hil$,
\[\langle P^2x, y\rangle = \langle Px, Py\rangle = \langle {Px,y}\rangle\]
by the equality above and hence $P^2 = P$ by taking $y=P^2x-Px$.\par
To show $\range(P) = \M$, it suffices to show that $\M \subset \range(P)$. For any $x\in\M$, we know $Px = x$ by the Theorem 5.24 and hence $\range(P) = \M$. Then by the conclusion in Ex.57, we know $\nul(P) = \rangle(P)^{\perp} = \M^{\perp}$.\par
b. If $\{x_i\}_{i\in I}\subset \Hil$ such that $Px_i \to y$, then we know $Px_i = P^2 x_i \to Py$ and hence $Py = y$, which means $y \in \range(P)$. Therefore, $\range(P)$ is closed. Then notice for any $x,y\in \Hil$,
\[ \langle x-Px, Py\rangle = \rangle \langle Px,y\rangle -\langle P^2x,y\rangle = 0\]
we know $Px$ is the element in $\range(P)$ such that $x-Px \in \range(P)^{\perp}$ and hence $P$ is the orthogonal projection onto $\range(P)$.\par
c. By the theorem 5.24, we know $Px$ is the unique element in $\M$ such that $x-Px\in\M^{\perp}$ and hence $Px = \sum\langle x, u_{\alpha}\rangle u_{\alpha}$ since 
\[\langle x- \sum\langle x,u_{\alpha}\rangle u_{\alpha}, u_{\beta} \rangle = \langle x,u_{\beta} \rangle - \langle x,u_{\beta}\rangle = 0\]
where $\langle \sum \langle x,u_{\alpha}\rangle u_{\alpha}, u_{\beta} \rangle$ is the limit of $\langle \sum\limits_{i=1}^n \langle x,u_{\alpha_i}\rangle u_{\alpha_i}, u_{\beta}\rangle,n\to\infty, \{\alpha_i\}_{i=1}^{\infty}$ are all indices such that $\langle x,u_{\alpha}\rangle \neq 0$ and hence it is $\langle x,u_{\beta}\rangle$ for any $\beta \in A$. Therefore, for any $y\in \M$, we know
\[\langle x - \sum\langle x,u_{\alpha}\rangle u_{\alpha}, y\rangle = \langle x - \sum\langle x,u_{\alpha}\rangle u_{\alpha},\sum\langle y,u_{\alpha}\rangle u_{\alpha}\rangle = 0 \]
and hence $x - \sum\langle x,u_{\alpha}\rangle u_{\alpha} \in M^{\perp}$, which means $Px = \sum\langle x,u_{\alpha}\rangle u_{\alpha}$.\par
\prvd
\vspace{0.5em}

\subsection*{Section 5.5 Ex.62} 
In this exercise the measure defining the $L^2$ spaces is Lebesgue measure.\par
a. $C([0,1])$ is dense in $L^2([0,1])$.\par
b. The set of polynomials is dense in $L^2([0,1])$.\par
c. $L^2([0,1])$ is separable.\par
d. $L^2(\R)$ is separable. \par
e. $L^2(\R^n)$ is separable. 
\vspace{0.5em}\\
\textbf{Sol.} \par
a. For $f\in L^2([0,1])$, we know there exists $\{\phi_n\}_{i=1}^n$ such that $|\phi_n| \uparrow |f|$, $\phi_n\to f$ a.e. and $\phi_n$ simple for any integer $n$. We know for any $E$ measurable in $[0,1]$, for any $\epsilon > 0$, there exists $U$ open, containing $E$ and $m(U-E) <\epsilon$ and hence for any $\epsilon > 0$ we may find $U_{\epsilon}$ open, containing $E$ and define $\varphi_{\epsilon}$ continuous on $U$ such that
\[\int |\chi_E - \varphi_{\epsilon}|^2 dm < m(U_{\epsilon}-E) < \epsilon\]
and hence we may find $\varphi_n$ continuous on $[0,1]$ such that
\[\int |\phi_n-\varphi_n|^2 dm < 2^{-n}\]
for any integer $n$ and hence $\varphi_n \to f, n\to\infty$ in $L^2([0,1])$, which means $C([0,1])$ is dense in $L^2([0,1])$.\par
b. We may know $\mathcal{A} = \overline{\{p\text{ polynomials on }[0,1]\}} = C([0,1])$ under the uniform norm by the Stone-Weierstrass Theorem and hence we may find $\{p_n\}_1^{\infty}$ on $[0,1]$ such that $||p_n-f|| < n^{-1}$ for any $f\in C([0,1])$. Then we know $||p_n-f||_{L^2([0,1])} < \int n^{-2}dm = n^{-2} \to 0$ and hence $p_n\to f$ in $L^2([0,1])$, where it is easy to check $C([0,1])\subset L^2([0,1])$. Therefore, for any $g\in L^2([0,1])$, we may find $\{\phi_n\}_1^{\infty}\subset C([0,1]), \{p_n\}_1^{\infty} \subset \{\text{Polynomials on }[0,1]\}$ such that $||\phi_n-f||_{L^2([0,1])} < n^{-1}, ||\phi_n-p_n||_{L^2([0,1])} < n^{-1}$, and hence
\[||f - p_n||_{L^2([0,1])} \leq ||f-\phi_n||_{L^2([0,1])}+||p_n-\phi_n||_{L^2([0,1])} < 2n^{-1} \to 0, n\to\infty\]
which means the set of polynomials is dense in $L^2([0,1])$.\par
c. Consider a polynomial $p(x) = \sum\limits_{i=0}^n a_ix^i$ and $\{r_{ki}\}_{i\geq 1, 0\leq k \leq n}\subset \Q$ such that $|r_{ki}| \uparrow |a_k|, r_{ki} \to a_k, i\to\infty, 0\leq k \leq n$ and we know
\[|\sum\limits_{k=0}^n r_{ik} x^i - p(x)| \leq 2\sum\limits_{k=0}^n |a_k| \in L^2([0,1])\]
on $[0,1]$ for any $i\geq 1$ and hence $\sum\limits_{k=0}^n r_{ik} x^i \to p(x)$ in $L^2([0,1])$ by the Dominated Convergence Theorem. Therefore, for any $f\in L^2([0,1])$, there exist $\{p_n\}_1^{\infty}, \{q_n\}_1^{\infty}$ where $p_n$ are polynomials and $q_n$ are the polynomials with rational coefficients such that $||p_n-f||_{L^2([0,1])} < n^{-1},||p_n-q_n||_{L^2([0,1])} < n^{-1}$ and hence 
\[||q_n-f||_{L^2([0,1])} <||p_n-f||_{L^2([0,1])} + ||p_n-q_n||_{L^2([0,1])} < 2n^{-1} \to 0,n\to\infty\]
which means the set of polynomials with rational coefficients is dense in $L^2([0,1])$. Therefore, $L^2([0,1])$ is separable.\par
d. Similarly, we may know $L^2([n,n+1])$ is separable since we may consider $\phi: L^2([0,1])\to L^2([n,n+1])$, which is easy to be checked an isometry and hence $L^2([n,n+1])$ is separable. By Sec.5.5 Ex.60 we know $L^2(\R)$ is separable.\par
e. We may use the induction to the dimension $n$, by Sec.5.5. Ex.61, we know $L^2(\R^k)$ is separable may imply $L^2(\R^{k+1})$ is separable. Therefore, $L^2(\R^n)$ is separable for any integer $n$ since $L^2(\R)$ is separable.

\prvd
\vspace{0.5em}

\addappheadtotoc

\end{document}
