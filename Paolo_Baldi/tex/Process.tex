\newcommand{\F}{\mathcal{F}}
\newcommand{\PR}{\mathbb{R}^+}
\newcommand{\G}{\mathcal{G}}
\newcommand{\Borel}{\mathcal{B}}

\section{Stochastic Processes}

\subsection{General facts}

\begin{definition}
    (Stochastic Process)\par
    A \textbf{stochastic process} is an object of the form
    \[
    X = (\Omega,\F,(\F_t)_{t\in T}, (X_t)_{t\in T},P)
    \]
    where
    \begin{itemize}
        \item $(\Omega,\F,P)$ is a probability space
        \item $T$ is a subset of $\PR$
        \item $(\F_t)_{t\in T}$ is a filtration, i.e. an increasing family of sub-$\sigma$-algebras of $\F$
        \item $(X_t)_{t\in T}$ is a family of r.v.'s on $(\Omega,\F)$ taking values in a measurable space $(E,\mathcal{E})$ \textbf{adapted} to $(\F_t)$.
    \end{itemize}
    The \textbf{natural filtration} $(\G_t)_t$ is defined as
    \[\G_t = \sigma(X_s,s\leq t)\]
    and the \textbf{augmented natural filtration} $(\overline{\G}_t)$ is defined by
    \[
    \overline{\G}_t = \sigma(\G_t,\mathcal{N})
    \]
    where $\mathcal{N} = \{A;A\in \F,P(A) = 0\}$. Denote $\F_{\infty} = \sigma(\bigcup_t \F_t)$ for a filtration $(\F_t)_t$
\end{definition}

\begin{definition}
    (Space of paths)\par
    $\Omega$ can be considered as a subset of $E^T:=\{\text{all functions }T\to E\}$ by the map \[\omega\mapsto(t\mapsto X_t(\omega))\]
    and hence is called the \textbf{space of paths} and $E$ is called the \textbf{state space}.
\end{definition}

\begin{definition}
    (Equivalent and modification)\par
    For two processes $(\Omega,\F,(\F_t)_{t\in T}, (X_t)_{t\in T},P)$ and $ (\Omega',\F',(\F'_t)_{t\in T}, (X_t')_{t\in T},P')$, they are \textbf{equivalent} if for any $t_1,\cdots,t_m\in T,(X_{t_1},\cdots,X_{t_m})$ and $(X_{t_1}',\cdots,X_{t_m}')$ have the same law.\par
    $X$ is called a \textbf{modification} of $X'$ if $(\Omega,\F,(\F_t)_{t\in T},P) = (\Omega',\F',(\F_t')_{t\in T},P')$ and for every $t\in T$, $X_t = X_t', P$-a.s., and they are \textbf{indistinguishable} if $X$ is a modification of $X'$ and 
    \[P(X_t = X_t'\text{ for every }t\in T) = 1\]
\end{definition}

\begin{example}
    Here is a counter example that if $X$ is a modification of $X'$, then $X$ and $X'$ are not necessarily indistinguishable. Let $\Omega = [0,1]$ and $\F = \Borel([0,1])$ and $P$ the Lebesgue measure, and
    \[X_t(\omega) = 1_{\{\omega\}}(t),\quad X_t'(\omega) = 0\]
\end{example}

\begin{definition}
    (Topological state space)\par
    Assume the state space is a topological space endowed with its Borel $\sigma$-algebra $\Borel(E)$ and $T$ an intercal of $\PR$.\par
    A process is said to be (a.s.) \textbf{continuous} if for every (a.e.) $\omega$ the map $t\mapsto X_t(\omega)$ is continuous. And the definitions of one side continuity is similar.\par
    $X$ is \textbf{measurable} if the map $(t,\omega)\mapsto X_t(\omega)$ is measurable $(T\times \Omega,\Borel (T)\otimes \F) \to (E,\Borel(E))$. It is said to be \textbf{progressively measurable} if for every $u \in T$ the map $(t,\omega) \to X_t(\omega)$ is measurable $([0,u]\times\Omega, \Borel([0,u])\otimes \F_u) \to (E,\Borel(E))$.
\end{definition}

\begin{proposition}
    Let $X = (\Omega,\F,(\F_t)_{t\in T},(X_t)_t,P)$ be a right-continuous process. Then it is progressively measurable.
\end{proposition}
\Pf\par
    For a fixed $u\in T$, we define $X^{(n)}$ by
    \[
    X_s^{(n)} = X_{(k+1)u/2^n}\text{ for }s\in [ku/2^n,(k+1)u/2^n)\quad\text{ and }X_s^{(n)} = X_u\text{ if }s\geq u
    \]
    and then we know $X_s^{(n)} = X_{s_n}$ for some $s_n > s$ and $|s_n - s| \leq u/2^n$ if $s \leq u$ and then we know $X_s^{(n)} \to X_s$ as $n\to \infty$ for $s\leq u$ since $s_n \downarrow s$. Consider $B \in \Borel(E)$ and then
    \[
    \begin{aligned}
       & \{X^{(n)} \in B\}\cap \{s\leq u\} \\ =& \left(\bigcup_{k=0}^{2^n-1}\{X^{(n)} \in B\}\cap \{s\in [ku/2^n,(k+1)u/2^n)\}\right) \cup \left(\{X^{(n)} \in B\}\cap \{s=u\}\right)\\
        =& \left(\bigcup_{k=0}^{2^n-1} [ku/2^n,(k+1)u/2^n)\times\{X_{(k+1)u/2^n} \in B\}\right) \cup \left(\{u\}\times \{X_u \in B\}\right)\in \Borel([0,u],\F_u)\\
    \end{aligned}
    \]
    which means $X^(n)$ is progressively measurable and hence $X$ is progressively measurable.

    \begin{definition}
        (Standard process)\par
        Denote $\F_{t+} = \cap_{\epsilon > 0} \F_{t+\epsilon}$ and we say the filtration is \textbf{right-contunuous} if $\F_{t+} = \F_t$ for every $t$.\par
        A process is said to be \textbf{standard} if $(\F_t)$ is right-continuous and $\F_t$ contains the negligible events of $\F$ for each $t$.
    \end{definition}

    \subsection{Kolmogorov's continuity theorem}

    \begin{theorem}
        (Kolmogorob's continuity theorem)\par
        Let $D\subset \mathbb{R}^m$ an open set and $(X_y)_{y\in D}$ a family of $d$-dimensional r.v.'s on $(\Omega,\F,P)$ such that there exists $\alpha > 0, \beta > 0, c>0$ such that
        \[
        E[|X_y - X_z|^{\beta}] \leq c|y-z|^{m+\alpha}
        \]
        Then there exists a family $(\widetilde{X}_y)_t$ of $\mathbb{R}^d$-valued r.v.'s such that
        \[X_y = \widetilde{X}_y\]
        a.s. for every $y\in D$ and that the map $y\mapsto \widetilde{X}_y(\omega)$ is Holder continuous with exponent $\gamma$ for every $\gamma < \alpha/\beta$ on every compaact subset of $D$ for every $\omega \in \Omega$.
    \end{theorem}