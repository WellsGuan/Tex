\section{Introduction to Sobolev Spaces}
\subsection{Sobolev spaces}
\begin{definition}
Let \(u\in D^{\prime}(U)\) , where \(U\subset \mathbb{R}^{d}\) is open. The \(k\) - th order \(L^{p}\) - based Sobolev norm of \(u\) is
\[
\| u\|_{W^{k,p}(U)} := \sum_{|\alpha |\leq k} \| D^{\alpha} u\|_{L^{p}}, \quad 1 \leq p < \infty ,
\]
where we are using the distributional derivative and assume that \(D^{\alpha} u\) is an \(L^{p}\) function.
\end{definition}


Expressions of the form \(\| D^{\alpha} u\|_{L^{p}}\) arise in the energy method for PDEs with \(p = 2\) .

\begin{definition}
Let \(U\) be an open subset of \(\mathbb{R}^{d}\) . The \(L^{p}\) - based Sobolev space of order \(k\) on \(U\) is
\[
W^{k,p}(U) = \{u\in D'(U):\| u\|_{W^{k,p}(U)}< \infty \} .
\]
Note that \(C_{c}^{\infty}(U) \subseteq W^{k,p}(U)\) . This allows us to introduce the following definition:
\end{definition}

\begin{definition}
The set of \(u \in W^{k,p}(U)\) that vanish (to appropriate orders) on \(\partial U\) is
\[
W_{0}^{k,p}(U) = \overline{C_{c}^{\infty}(U)}^{\| \cdot \|}_{W^{k,p}2}\subset W^{k,p}(U).
\]
When \(p = 2\) , we introduce the notation
\[
H^{k}(U) = W^{k,2}(U), \qquad H_{0}^{k}(U) = W_{0}^{k,2}(U).
\]
We can define an inner product on \(H^{k}\) by
\[
\langle u,v\rangle_{H^{k}} = \sum_{|\alpha |\leq k}\langle D^{\alpha}u,D^{\alpha}v\rangle_{L^{2}}.
\]
\end{definition}

\begin{proposition}
1. For all \(k \in \mathbb{Z}_{\geq 0}\) and \(1 \leq p \leq \infty\) , the following Sobolev spaces \((W^{k,p}(U), \| \cdot \|_{W^{k,p}})\) and \((W_{0}^{k,p}(U), \| \cdot \|_{W^{k,p}})\) are Banach spaces.

2. For all \(k \in \mathbb{Z}_{\geq 0}\) , \((H^{k}(U), \langle \cdot , \cdot \rangle_{H^{k}})\) and \((H_{0}^{k}(U), \langle \cdot , \cdot \rangle_{H^{k}})\) are Hilbert spaces.

3. (Fourier-analytic characterization of \(H^{k}\) ) Given \(u \in H^{k}(U)\)
\[
\| u\|_{H^{k}} \simeq \| \hat{u} \|_{L^{2}} + \| \xi |^{k} \hat{u} \|_{L^{2}} \simeq \| (1 + |\xi |^{2})^{k / 2} \hat{u} \|_{L^{2}},
\]
where \(A \simeq B\) means \(A \lesssim B\) and \(B \lesssim A\) .
\end{proposition}

\section{A Priori Estimates and Approximation Theorems}
\subsection{Relationship between a priori estimates, existence, and uniqueness}
Here is a concrete thing to keep in mind: Often, we prove a priori estimates for a PDE, i.e. if \(u\in X\) with \(P u = f\) , then
\[
\| u\|_{X}\leq C\| f\|_{Y}.
\]

\begin{proposition}
Let \(X,Y\) be Banach spaces, and let \(P:X\to Y\) be a bounded, linear operator. Denote by \(P^{*}:Y^{*}\to X^{*}\) the adjoint of \(P\) , i.e.
\[
\langle v,P u\rangle = \langle P^{*}v,u\rangle \quad \mathrm{for~all~}u\in X,v\in Y^{*}.
\]
Suppose there exists a constant \(C > 0\) such that
\[
\| u\|_{X}\leq C\| Pu\|_{Y}\quad \mathrm{for~all~}u\in X.
\]
Then:
1. (Uniqueness for \(P u = f\) ) If \(u\in X\) and \(P u = 0\) , then \(u = 0\) .

2. (Existence for \(P^{*}v = g\) ) For all \(g\in X^{*}\) , there exists a \(v\in Y^{*}\) such that \(P^{*}v = g\) and
\[
\| v\|_{Y^{*}}\leq C\| g\|_{X^{*}}.
\]
\end{proposition}

\begin{proof}
The proof of 1. is clear, since \(\| u\|_{X}\leq 0\) implies \(u = 0\) since \(X\) is normed. We proceed with the proof of the second part 2., via the Hahn- Banach theorem. In particular, our objective is to find \(v\in Y^{*}\) such that \(P^{*}v = g\) , which is equivalent to
\[
\langle P^{*}v,u\rangle = \langle v,P u\rangle = \langle g,u\rangle \quad \mathrm{for~all~}u\in X.
\]
Define \(\ell :P(X)\to \mathbb{R}\) by
\[
\ell (P u):= \langle g,u\rangle .
\]
Since \(P\) is injective (by 1. ), this is well- defined. Moreover,
\[
\begin{array}{r l}
& {|\ell (P u)| = |\langle g,u\rangle |}\\
& {\qquad \leq \| g\|_{X^{*}}\| u\|_{X}}\\
& {\qquad \leq C\| g\|_{X^{*}}\| P u\|_{Y}}\\
& {\qquad \leq C\| g\|_{X^{*}}}
\end{array}
\]
so \(\ell\) is bounded. By the Hahn- Banach theorem, there exists \(v\in Y^{*}\) such that
\[
\langle v,P u\rangle = l(p u) = \langle g,u\rangle \quad \forall u\in X,
\]
and \(\| v\|_{Y^{*}}\leq C\| g\|_{X^{*}}\).
\end{proof}

\subsection{Approximation by smooth functions and smooth partition of unity}
\begin{lemma}
Let \(\phi\) be smooth, compactly supported i.e. \(\phi \in C_{0}^{\infty}(U)\) , and have
\[
\int \phi dx = 1.
\]
Let \(u\in L^{p}(\mathbb{R}^{d})\) with \(1\leq p< \infty\) . Denote mollifiers
\[
\phi_{\epsilon}(x) = \frac{1}{\epsilon^{d}}\phi (x / \epsilon)\quad (\mathrm{so}\int \phi_{\epsilon} = 1).
\]
Then
\[
\| \phi_{\epsilon}*u - u\|_{L^{p}}\xrightarrow[\epsilon\to0]{}\mathbb{0},
\]
where
\[
\phi_{\epsilon}*u = \int \phi_{\epsilon}(x - y)u(y)dy.
\]
\end{lemma}

\begin{proof}
The key ingredient is the continuity of the translation operator on \(L^{p}\) . Define for \(z \in \mathbb{R}^{d}\) and \(u \in L^{p}\) the translation operator
\[
\tau_{z}u(x) = u(x - z).
\]
Then
\[
\lim_{|z|\to 0}\| \tau_{z}u - u\|_{L^{p}} = 0,
\]
which you can check. Now
\[
\phi_{\epsilon}*u(x) - u(x) = \int u(x - y)\phi_{\epsilon}(y)dy - u(x).
\]
Since \(\int \phi_{\epsilon} = 1\)
\[
= \int (u(x - y) - u(x))\phi_{\epsilon}(y)dy.
\]
Taking the \(L^{p}\) norm, we have
\[
\| \phi_{\epsilon}*u(x) - u(x)\|_{L^{p}} = \left\| \int (u(x - y) - u(x))\phi_{\epsilon}(y)dy\right\|_{L^{p}}
\]
\[
\qquad \leq \int \| u(\cdot -y) - u(\cdot)\|_{L^{p}}|\phi_{\epsilon}(y)|dy.
\]
Since \(\phi\) has compact support, \(\operatorname {supp}\phi_{\epsilon}\to \{0\}\) as \(\epsilon \to 0\) . Thus, the integrand goes to 0 as \(\epsilon \to 0\) So we may apply the dominated convergence theorem to get
\[
\| \phi_{\epsilon}*u - u\|_{L^{p}}\xrightarrow[\epsilon\to 0]{}0.
\]
\end{proof}

\begin{lemma}
Suppose \(\{U_{\alpha}\}_{\alpha \in A}\) be an open covering of \(U\) in \(\mathbb{R}^{d}\) . There exists a smooth partition of unity \(\{\chi_{\alpha}\}_{\alpha \in A}\) on \(U\) subordinate to \(\{U_{\alpha}\}_{\alpha \in A}\) , i.e.
1. \(\sum_{\alpha}\chi_{\alpha}(x) = 1\) on \(U\) and for all \(x\in U\) there exist only finitely many nonzero \(\chi_{\alpha}(x)\)
2. \(\operatorname {supp}\chi_{\alpha}\subseteq U_{\alpha}\)
3. \(\chi_{\alpha}\) is smooth.
\end{lemma}

\begin{proof}
Start from a continuous partition of unity and apply the previous lemma to approximate by smooth functions.
\end{proof}

\section{Approximation in Bounded Domains and the Extension Theorem}
\subsection{Approximation theorems in bounded domains}
\begin{theorem}
Let \(k \geq 0\) be an integer and \(1 \leq p < \infty\).
(i) \(C^{\infty}(\mathbb{R}^{d})\) is dense in \(W^{k,p}(\mathbb{R}^{d})\).
(ii) \(C^{\infty}_{c}(\mathbb{R}^{d})\) is dense in \(W^{k,p}(\mathbb{R}^{d})\).
\end{theorem}

\begin{proof}
(a) This is an application of mollification.
(b) Approximate by \(f\chi(1/R)\), letting \(R \to \infty\), where \(\chi \in C^{\infty}_{c}(\mathbb{R}^{d})\) is such that \(\chi(0) = 1\).
\end{proof}

\begin{theorem}
Let \(k \geq 0\) be an integer, \(1 \leq p < \infty\), and \(U\) an open subset of \(\mathbb{R}^{d}\). Then \(C^{\infty}(U)\) is dense in \(W^{k,p}(U)\).
\end{theorem}

\begin{proof}
Let \(u \in W^{k,p}(U)\), and fix \(\epsilon > 0\) . We want to find \(v \in C^{\infty}(U)\) such that
\[
\| u - v\|_{W^{k,p}} \leq \epsilon .
\]
Define
\[
U_{j}:= \{x\in U:\operatorname {dist}(x,\partial U) > 1 / j\} ,\qquad V_{j} = U_{j}\backslash \overline{U_{j + 1}}.
\]
Then \(U \subset \bigcup_{j = 1}^{\infty} V_{j}\) , so there is a smooth partition of unity \(\chi_{j}\) subordinate to \(V_{j}\) .

Now split
\[
u = \sum_{j = 1}^{\infty} u \chi_{j} := \sum_{j = 1}^{\infty} u_{j}.
\]
Then, as \(\operatorname {supp} \chi_{j} \subset V_{j}\) , we have \(\operatorname {supp} u_{j} = \operatorname {supp}(u \chi_{j}) \subset V_{j}\) . Moreover, \(u_{j} \in C_{c}^{\infty}(\mathbb{R}^{d})\) .

If we let \(\phi \in C_{c}^{\infty}(\mathbb{R}^{d})\) with \(\int \phi = 1\) and \(\operatorname {supp} \phi \subset B_{1}(0)\) be a mollifier, let \(v_{j} = \phi_{\epsilon_{j}} * u_{j}\) , where \(\epsilon_{j}\) is chosen to achieve
\[
\| u_{j} - v_{j}\|_{W^{k,p}} \leq 2^{-j} \epsilon , \qquad \operatorname {supp} v_{j} \subset \widetilde{V}_{j} = U_{j - 1} \backslash \overline{U_{j + 2}}.
\]
Here, we make use of the fact that
\[
\operatorname {supp}(f * g) \subseteq \operatorname {supp} f + \operatorname {supp} g = \{x + y \in \mathbb{R}^{d}: x \in \operatorname {supp} f, y \in \operatorname {supp} g\}.
\]
Now take
\[
v = \sum_{j = 1}^{\infty} v_{j}.
\]
This is well- defined, as \(\widetilde{V}_{j}\) is locally finite. This is also smooth, so \(v \in C^{\infty}(U)\) . On the other hand,
\[
\| v - u\|_{W^{k,p}} \leq \sum_{j = 1}^{\infty} \| v_{j} - u_{j}\|_{W^{k,p}} \leq \sum_{j = 1}^{\infty} 2^{-j} \epsilon = \epsilon .
\]
\end{proof}

\subsection{The extension theorem}
\begin{theorem}[Extension theorem]
Let \(k\geq 0\) be a nonnegative integer, \(1\leq p< \infty\) , and \(U\) a bounded domain with \(C^k\) boundary. Let \(V\) be an open set such that \(V\supseteq \bar{U}\) . Then there exists an operator
\[
\mathcal{E}:W^{k,p}(U)\to W^{k,p}(\mathbb{R}^{d})
\]
such that
(i) (Extension) \(\mathcal{E}u|_U = u\)
(ii) (Linear and bounded) \(\mathcal{E}\) is linear, and \(\| \mathcal{E}u\|_{W^{k,p}(\mathbb{R}^{d})} \leq C\| u\|_{W^{k,p}(U)}\) .
(iii) (Support prescription) \(\operatorname{supp} \mathcal{E}u \subseteq V\) .
\end{theorem}

\begin{proof}
Observe that, by the previous approximation theorem, it suffices to consider \(u \in C^{\infty}(\bar{U})\) (by density and the boundedness property (ii)).

Remark (what we will actually construct). We first build an extension locally near each boundary chart. The resulting locally extended functions will have support in a small neighborhood of the boundary chart. We then glue all pieces together (using a partition of unity) and finally multiply by a cutoff supported in \(V\) to enforce the support condition (iii).

Step 1 (Reduction to the half- ball case). As in Step 1 in the proof of the previous theorem, construct the open sets \(U_{0}, U_{1}, \ldots , U_{K}\) and the partition of unity \(\chi_{0}, \chi_{1}, \ldots , \chi_{K}\) .

Define \(u_{k} := \chi_{k} u\) , and observe that
\[
\bullet u_{0}\mathrm{~is~already~in~}W^{k,p}(\mathbb{R}^{d})\mathrm{~and~}\operatorname {supp}u_{0}\subseteq U_{0}\subseteq V,
\]
\[
\bullet u_{k}\in C^{\infty}(U)\mathrm{~and~}\operatorname {supp}u_{k}\subseteq B_{r_{0}}\subseteq U_{k}\cap U.
\]

What the boundary chart means: For each \(k \geq 1\) , because \(\partial U\) is \(C^{k}\) , after a rigid motion of coordinates (translation and rotation) we may assume that in \(U_{k}\) the boundary is given by a graph
\[
\partial U\cap U_{k} = \{(x^{\prime},x_{d}):x_{d} = \gamma (x^{\prime})\} ,\qquad U\cap U_{k} = \{(x^{\prime},x_{d}):x_{d} > \gamma (x^{\prime})\} ,
\]
where \(x^{\prime} = (x^{1}, \ldots , x^{d - 1})\) and \(\gamma \in C^{k}\) . The point \(x_{0} = (x_{0}^{\prime}, x_{0}^{\prime}) \in \partial U\) is chosen in that chart. The purpose is to "flatten" this graph.

The way we do this flattening is via the following change of variables
\[
y^{\prime} = x^{\prime} - x_{0}^{\prime},\qquad y_{d} = x_{d} - \gamma (x^{\prime}),
\]
then \(U_{k} \cap U\) gets mapped into \(\{y \in B_{\tilde{r}}(0): y^{d} > 0\}\) .

One can wonder why this is the right change of variables? It might sound like over explaining but this is the idea: The map \(x \mapsto y\) is designed so that the boundary graph becomes flat: if \(x_{d} = \gamma (x^{\prime})\) (a boundary point), then \(y_{d} = 0\) . If \(x_{d} > \gamma (x^{\prime})\) (an interior point), then \(y_{d} > 0\) . Thus the image of \(U \cap U_{k}\) is a region above the hyperplane \(\{y_{d} = 0\}\) . The translation \(y^{\prime} = x^{\prime} - x_{0}^{\prime}\) just moves the boundary point to the origin. Because we work in a small ball around \(x_{0}\) and \(\gamma\) is \(C^{k}\) , the image is contained in a half- ball \(B_{\tilde{r}}(0) \cap \{y_{d} > 0\}\) for some \(\tilde{r} > 0\) .

In fact \(x \mapsto y\) is a \(C^{k}\) diffeomorphism (locally). Write \(\Phi (x^{\prime}, x_{d}) = (y^{\prime}, y_{d}) = (x^{\prime} - x_{0}^{\prime}, x_{d} - \gamma (x^{\prime}))\) . Then
\[
D\Phi = \begin{pmatrix} I_{d-1} & 0 \\ -\nabla \gamma & 1 \end{pmatrix}.
\]
Hence \(D\Phi\) is invertible everywhere in the chart, and \(\Phi\) is locally bijective with explicit inverse
\[
\Phi^{-1}(y^{\prime},y_{d}) = (y^{\prime} + x_{0}^{\prime},y_{d} + \gamma (y^{\prime} + x_{0}^{\prime})),
\]
which is also \(C^{k}\) . So \(\Phi\) is a \(C^{k}\) diffeomorphism between the chart neighborhood and its image.

Note that the change of variables \(x \mapsto y\) is \(C^{k}\) , and \(u_{k}\) is smooth, so \(\tilde{u}_{k}(y) = u_{k}(x(y))\) satisfies, by the chain rule,
\[
\| \tilde{u}_{k}\|_{W_{y}^{k,p}(\tilde{U})}\leq C\| u_{k}\|_{W_{x}^{k,p}}.
\]

Step 2 (Extension in the half- ball case). Now we have \(U = B_{r}^{+}(0)\) , \(W = B_{r / 2}^{+}(0)\) , and \(\operatorname {supp}u\subseteq W\) , and we want to extend \(u\) . The idea is the higher order reflection method. After flattening, we may assume \(u\) is defined on the upper half- ball
\[
B_{r}^{+}(0) = \{x\in B_{r}(0):x_{d} > 0\} ,
\]
and moreover \(\operatorname {supp}u\subset B_{r / 2}^{+}(0)\) so it stays away from the spherical boundary \(\partial B_{r}(0)\) . We only need to extend across the flat boundary \(\{x_{d} = 0\}\) , while keeping \(W^{k,p}\) regularity up to order \(k\) .

Define
\[
\tilde{u}(x) = \sum_{j=0}^{k} \alpha_j u(x', -\beta_j x_d), \quad x_d < 0,
\]
where the scaling factor \(0< \beta_{j}< 1\) is chosen so that \((x^{1},\ldots ,x^{d - 1}, - \beta_{j}x^{d})\in B_{r}^{+}(0)\).

Remark: Why we must "shrink" (the role of \(\beta_{j}< 1\) ). If \(x\in B_{r}(0)\) has \(x_{d}< 0\) , then the reflected point \((x^{\prime}, - x_{d})\) might lie outside \(B_{r}(0)\) when \(x\) is close to the spherical boundary. Choosing \(0< \beta_{j}< 1\) ensures that \(- \beta_{j}x_{d}\) is smaller in magnitude than \(- x_{d}\) , i.e. it moves the point closer to the plane \(\{x_{d} = 0\}\) . Because \(\operatorname {supp}u\subset B_{r / 2}^{+}(0)\) , we can choose a fixed finite set of \(\beta_{j}\in (0,1)\) so that whenever \(x\in B_{r}(0)\) and \(x_{d}< 0\) , all points \((x^{\prime}, - \beta_{j}x_{d})\) remain inside \(B_{r}^{+}(0)\) (indeed inside the region where \(u\) is defined). This is the local reason for the shrink: it avoids leaving the chart.

We need to match the normal derivatives on \(\{x^{d} = 0\}\) up to order \(k\) . Observe that
\[
\partial_{x^{d}}^{j}(u(x^{1},\ldots ,x^{d - 1}, - \beta_{j}x^{d})) = (-\beta_{j})^{j}(\partial_{x^{d}}^{j}u)(x^{1},\ldots ,x^{d - 1}, - \beta_{j}x^{d}).
\]
Deriving the matching conditions carefully. For \(x_{d}< 0\) , we define
\[
\tilde{u} (x^{\prime},x_{d}) = \sum_{m = 0}^{k}\alpha_{m}u(x^{\prime}, - \beta_{m}x_{d}).
\]
Fix \(\ell \in \{0,1,\ldots ,k\}\) . Differentiate \(\ell\) times in the normal variable:
\[
\partial_{x_{d}}^{\ell}\tilde{u} (x^{\prime},x_{d}) = \sum_{m = 0}^{k}\alpha_{m}\partial_{x_{d}}^{\ell}\Big(u(x^{\prime}, - \beta_{m}x_{d})\Big) = \sum_{m = 0}^{k}\alpha_{m}(-\beta_{m})^{\ell}\left(\partial_{x_{d}}^{\ell}u\right)(x^{\prime}, - \beta_{m}x_{d}).
\]
Now take the limit \(x_{d}\to 0^{- }\) : then \(- \beta_{m}x_{d}\to 0^{+}\) , so
\[
\partial_{x_{d}}^{\ell}\tilde{u} (x^{\prime},0^{-}) = \sum_{m = 0}^{k}\alpha_{m}(-\beta_{m})^{\ell}\left(\partial_{x_{d}}^{\ell}u\right)(x^{\prime},0^{+}).
\]
On the other hand, for \(x_{d} > 0\) , \(\tilde{u} = u\) , hence
\[
\partial_{x_{d}}^{\ell}\tilde{u} (x^{\prime},0^{+}) = (\partial_{x_{d}}^{\ell}u)(x^{\prime},0^{+}).
\]
To ensure \(\tilde{u}\in W^{k,p}\) across the interface, we impose equality of these traces:
\[
\partial_{x_{d}}^{\ell}\tilde{u} (x^{\prime},0^{-}) = \partial_{x_{d}}^{\ell}\tilde{u} (x^{\prime},0^{+}),\qquad \ell = 0,\ldots ,k.
\]
Since \((\partial_{x_{d}}^{\ell}u)(x^{\prime},0^{+})\) is common to both sides, it is enough to require
\[
\sum_{m = 0}^{k}\alpha_{m}(-\beta_{m})^{\ell} = 1,\qquad \ell = 0,\ldots ,k.
\]
This is equivalent to
\[
\begin{pmatrix}
1 & 1 & \cdots & 1 \\
-\beta_0 & -\beta_1 & \cdots & -\beta_k \\
(-\beta_0)^2 & (-\beta_1)^2 & \cdots & (-\beta_k)^2 \\
\vdots & \vdots & \ddots & \vdots \\
(-\beta_0)^k & (-\beta_1)^k & \cdots & (-\beta_k)^k
\end{pmatrix}
\begin{pmatrix}
\alpha_0 \\ \alpha_1 \\ \vdots \\ \alpha_k
\end{pmatrix}
=
\begin{pmatrix}
1 \\ 1 \\ \vdots \\ 1
\end{pmatrix}.
\]
Written in matrix form, this is a linear system involving a Vandermonde matrix. So the \(\beta_{j}\) are known and then we want to solve for the \(\alpha\) 's this system is solvable (explicit determinant), because the matrix is a Vandermonde matrix with nodes \(t_{j} = - \beta_{j}\) . Its determinant is
\[
\operatorname*{det}V = \prod_{0\leq i< j\leq k}(t_j - t_i) = \prod_{0\leq i< j\leq k}((-\beta_j) - (-\beta_i)) = \prod_{0\leq i< j\leq k}(\beta_i - \beta_j).
\]
If the \(\beta_{j}\) are pairwise distinct, then \(\operatorname*{det}V\neq 0\) , so \(V\) is invertible and the system has a unique solution \((\alpha_{0},\ldots ,\alpha_{k})\) . Importantly, these coefficients depend only on the chosen \(\beta_{j}\) (hence only on \(k\) ), not on \(u\) .

Why this produces a \(W^{k,p}\) extension. Because the matching conditions hold for \(\ell = 0,\ldots ,k\) the traces of \(\partial_{x_{d}}^{\ell}\tilde{u}\) from above and below coincide at \(x_{d} = 0\) up to order \(k\) . Away from \(x_{d} = 0\) , \(\tilde{u}\) is a finite sum of smooth compositions of \(u\) , hence has the same \(W^{k,p}\) regularity. The only potential problem is the interface \(x_{d} = 0\) ; the derivative matching prevents the creation of distributional boundary terms when taking weak derivatives. Concretely, if you test \(\partial_{x_{d}}^{\ell}\tilde{u}\) against a smooth compactly supported test function and integrate by parts separately on \(\{x_{d} > 0\}\) and \(\{x_{d}< 0\}\) , the boundary terms at \(x_{d} = 0\) cancel precisely because the traces match. Therefore all weak derivatives of \(\tilde{u}\) up to order \(k\) lie in \(L^{p}\) , so \(\tilde{u}\in W^{k,p}(B_{r}(0))\) . Moreover, since the construction is linear and involves only finitely many terms,
\[
\| \tilde{u}\|_{W^{k,p}(B_r(0))}\leq C\| u\|_{W^{k,p}(B_r^+ (0))},
\]
with \(C\) depending on \(k,p\) and the chosen \(\beta_{j}\) .

Now use the fact that if all the \(\beta_{j}\) are distinct, then this matrix is invertible. This means that there is a choice of \((\alpha_{0},\ldots ,\alpha_{K})\) so that these equations hold. This defines \(\tilde{u}\) on \(B_{r}(0)\) , which extends \(u\) and matches all derivatives up to order \(K\) on the boundary \(\{x^{d} = 0\}\) .

Step 3 (Extension to \(\mathbb{R}^{d}\) ). Final localization and support in \(V\) . At this stage, \(\tilde{u}\) is defined only in the local chart ball \(B_{r}(0)\) . To obtain a global extension in \(\mathbb{R}^{d}\) and enforce \(\operatorname {supp}(\mathcal{E}u)\subset V\) , we proceed as follows:

1 Extend \(\tilde{u}\) by zero outside \(B_{r}(0)\) . Because \(\operatorname{supp}u\subset B_{r / 2}^{+}(0)\) and we may choose a cutoff later, this does not create boundary issues (the cutoff makes the function vanish near \(\partial B_{r}(0)\) ). Choose a smooth cutoff \(\chi_{V}\in C_{c}^{\infty}(V)\) such that \(\chi_{V}\equiv 1\) on \(\overline{U}\) (or at least on a neighborhood containing all supports of the glued pieces inside \(U\) ). Define \(\mathcal{E}u:= \chi_{V}\tilde{u}\) (and similarly for each localized boundary piece). This forces the global support to lie inside \(V\) while preserving the values on \(U\) because \(\chi_{V}\equiv 1\) there.

Finally, put an appropriate smooth cutoff \(\chi_{V} = 1\) on \(U\) with \(\operatorname{supp}\chi_{V}\subseteq V\) to define \(\mathcal{E}u\) , i.e.
\[
\mathcal{E}u = \chi_{V}\tilde{u}.
\]
(How the full operator is obtained). Repeat the construction for each boundary chart \(U_{k}\) (with its own flattening map \(\Phi_{k}\) ), extend each \(u_{k} = \chi_{k}u\) , and then sum the finitely many extended pieces together with the interior extension \(u_{0}\) . Linearity is clear at every step, and the global \(W^{k,p}\) bound follows from summing finitely many estimates. \(\square\)
\end{proof}

\section{Trace and Extension Theorems and Introduction to Sobolev Inequalities}
\subsection{The trace theorem}
\begin{definition}
For \(u\in C^{1}(\overline{U})\) , we define the trace to be
\[
\mathrm{tr}_{\partial U}u:= u|_{\partial U}.
\]
We wish to extend this operation to all of \(W^{1,p}(U)\) . Note that \(\mathrm{tr}_{\partial U}\) is linear, so we can extend it if we know it is bounded.
\end{definition}

\begin{theorem}[Trace theorem, non- sharp]
Let \(U\) be a bounded open subset of \(\mathbb{R}^{d}\) with \(C^{1}\) boundary \(\partial U\) , and let \(1< p< \infty\) . Then for \(u\in C^{1}(\overline{U})\)
\[
\| \mathrm{tr}_{\partial U}u\|_{L^p (\partial U)}\leq C\| u\|_{W^{1,p}(U)}.
\]
(i) As a consequence, \(\mathrm{tr}_{\partial U}\) extends uniquely by continuity (and density of \(C^{1}(\overline{U})\subset W^{1,p}(U)\) ) to a bounded linear map
\[
\mathrm{tr}_{\partial U}:W^{1,p}(U)\to L^p (\partial U).
\]
(ii) Moreover, \(u\in W_{0}^{1,p}(U)\) if and only if \(\mathrm{tr}_{\partial U}u = 0\).
\end{theorem}

The map \(\mathrm{tr}_{\partial U}:W^{1,p}(U)\to L^{p}(\partial U)\) is not surjective.

\begin{theorem}[Sharp trace theorem]
For \(u \in C^{\infty}(\overline{\mathbb{R}_{+}^{d}}) \cap H^{1}(\mathbb{R}_{+}^{d})\) ,
\[
\| \operatorname {tr}_{\partial U}u\|_{H^{1 / 2}(\mathbb{R}^{d - 1})}\leq C\| u\|_{H^{1}(\mathbb{R}_{+}^{d})}.
\]
\end{theorem}

\begin{proof}
Let \(u \in C^{\infty}(\overline{\mathbb{R}_{+}^{d}}) \cap H^{1}(\mathbb{R}_{+}^{d})\) .

Step 0 (What the extension procedure is, and why we use it). The trace we want to estimate is the boundary value \(u(x^{\prime},0)\) , which lives on \(\mathbb{R}^{d - 1}\) . The Fourier proof below is most transparent if we can treat \(u\) as a function on all of \(\mathbb{R}^{d}\) , because then we can use the full Fourier transform in \(\xi = (\xi^{\prime},\xi_{d})\) and identify the \(H^{1}\) norm with \(\| (1 + |\xi |^{2})^{1 / 2}\tilde{u}\|_{L_{\xi}^{2}}\) . So we first extend \(u\) from the half- space \(\mathbb{R}_{+}^{d}\) to a function \(\tilde{u}\) on \(\mathbb{R}^{d}\) with a controlled \(H^{1}\) norm.

Extension procedure (for the half- space). Define the even reflection across the boundary \(\{x_{d} = 0\}\) by
\[
\tilde{u} (x^{\prime},x_{d}):= u(x^{\prime},|x_{d}|),\qquad (x^{\prime},x_{d})\in \mathbb{R}^{d - 1}\times \mathbb{R}.
\]
Then \(\tilde{u} \in H^{1}(\mathbb{R}^{d})\) and \(\tilde{u} = u\) on \(\mathbb{R}_{+}^{d}\) . Moreover,
\[
\| \tilde{u}\|_{L^{2}(\mathbb{R}^{d})}^{2} = \int_{\mathbb{R}^{d - 1}}\int_{\mathbb{R}}|\tilde{u} (x^{\prime},x_{d})|^{2}d x_{d}d x^{\prime} = 2\int_{\mathbb{R}^{d - 1}}\int_{0}^{\infty}|u(x^{\prime},x_{d})|^{2}d x_{d}d x^{\prime} = 2\| u\|_{L^{2}(\mathbb{R}_{+}^{d})}^{2},
\]
and for tangential derivatives \(\partial_{x_{j}}\) , \(j \leq d - 1\) ,
\[
\| \partial_{x_{j}}\tilde{u}\|_{L^{2}(\mathbb{R}^{d})}^{2} = 2\| \partial_{x_{j}}u\|_{L^{2}(\mathbb{R}_{+}^{d})}^{2}.
\]
For the normal derivative, observe that for \(x_{d} \neq 0\) ,
\[
\partial_{x_{d}}\tilde{u} (x^{\prime},x_{d}) = \operatorname {sgn}(x_{d})\partial_{x_{d}}u(x^{\prime},|x_{d}|),
\]
hence
\[
\| \partial_{x_{d}}\tilde{u}\|_{L^{2}(\mathbb{R}^{d})}^{2} = 2\| \partial_{x_{d}}u\|_{L^{2}(\mathbb{R}_{+}^{d})}^{2}.
\]
Therefore
\[
\| \tilde{u}\|_{H^{1}(\mathbb{R}^{d})}\leq C\| u\|_{H^{1}(\mathbb{R}_{+}^{d})}, \quad (1)
\]
and the extension procedure is justified: it is linear, preserves the boundary values (since \(\tilde{u} (x^{\prime},0) = u(x^{\prime},0)\) ), and gives a uniform \(H^{1}\) bound.

Step 1 (Write the trace using Fourier in the normal variable). Since \(\tilde{u}\) is defined on \(\mathbb{R}^{d}\) and \(\tilde{u} (x^{\prime},0) = u(x^{\prime},0)\) , we have
\[
\operatorname {tr}_{\partial U}u(x^{\prime}) = u(x^{\prime},0) = \tilde{u} (x^{\prime},0).
\]
Using the inverse Fourier transform in the \(x_{d}\) variable (with the convention \(\mathcal{F}_{x_{d}}f(\xi_{d}) = \int_{\mathbb{R}}e^{- i x_{d}\xi_{d}}f(x_{d})d x_{d}\) ), we obtain
\[
\tilde{u} (x^{\prime},0) = \int_{\mathbb{R}}\mathcal{F}_{x_{d}}\tilde{u} (x^{\prime},\xi_{d})\frac{d\xi_{d}}{2\pi}.
\]
Taking the Fourier transform in \(x^{\prime}\) , Fubini gives
\[
\mathcal{F}_{x^{\prime}}(\operatorname {tr}_{\partial U}u)(\xi^{\prime}) = \int_{\mathbb{R}}\tilde{\tilde{u}} (\xi^{\prime},\xi_{d})\frac{d\xi_{d}}{2\pi},
\]
where \(\tilde{\tilde{u}} = \mathcal{F}_{x}\tilde{u}\) denotes the full Fourier transform in \(\mathbb{R}^{d}\) .

On Wednesday Feb 4 we will pick up from here.
\end{proof}

\end{document}
