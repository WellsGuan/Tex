\section{Parabolic Equation for the decoupling function}

\subsection{Main Theorem}

\emph{Remark.} The space
    \[
    L^1(\mathcal{K};W^{2,\infty}(U;\mathcal{H}_+^m))
    \]
means some functions which are $L^1$ with fixed space point and time varying and $W^{2,\infty}(U:\mathcal{H}_+^m)$ with fixed time point and space varying, with the norm
\[||f|| = \int ||f(t,\cdot)||_{W^{2,\infty}}dt.\]
For two normed spaces, their intersection will be equipped with a norm obtained by adding both the norms together.

\begin{definition}
    An \emph{almost classical solution} to $\eqref{paraeq}$ on a time interval $[0,Q_0)$ is a continuous function $H:[0,Q_0)\times\mathbb{R}^m \to \mathcal{H}_+^m$ such that the following conditions hold:
    \begin{enumerate}
        \item For every compact $\mathcal{K}\subset (0,Q_0)$ and bounded open $U \subset \mathbb{R}^m$,
        \[
        H_{\mathcal{K}\times U} \in L^1(\mathcal{K};W^{2,\infty}(U;\mathcal{H}_+^m)) \cap C^1(\mathcal{K}\times U ;\mathcal{H}^m_+).
        \]
        where $W^{2,\infty}(U;\mathcal{H}^m_+)$ is the Sobolev space of functions on $U$ taking values in $\mathcal{H}_+^m$ with weak second derivative mearuable ans essentially bounded.
        \item We have $H(0,b) = \sigma^2(b)$ for all $b\in\mathbb{R}^m$;.
        \item We have
        \[\partial_qH(q,b) = \tfrac{1}{2}[H(q,b):\nabla_b^2]H(q,b)\quad\text{for almost all }(q,b) \in (0,Q_0) \times \mathbb{R}^m.\]
    \end{enumerate}
\end{definition}

\begin{proposition}
    Fix $Q_0 \in (0,\Qbar{(\sigma)})$. Suppose $H$ is an almost classical solution of $\eqref{paraeq}$ on $[0,Q_0]$ such that $\sup_{q\in[0,Q_0]} \Lip(\sqrt{H(q,\cdot)}) < \infty$. Also assume that for each compact $\mathcal{K} \subset (0,Q_0)$, there exists a constant $C(\mathcal{K}) \subset(0,Q_0)$, there exists a constant $C(\mathcal{K})$ such that for all $R>0, U_R = \{b:|b| < R\}$ the open call in $\mathbb{R}^m$ of radius $R$,
    \[
     ||H|_{\mathcal{K}\times U_R}|| _{L^1(\mathcal{K};W^{2,\infty}(U_R;\mathcal{H}_+^m)) \cap C^1(\mathcal{K}\times U_R ;\mathcal{H}^m_+)} \leq CR^C
    \]
    then $\sqrt{H} = J_{\sigma}$ on $[0,Q_0] \times \mathbb{R}^m$.
\end{proposition}