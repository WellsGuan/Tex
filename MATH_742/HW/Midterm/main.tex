%!TEX program = xelatex
\documentclass[lang=en,11pt,a4paper,citestyle =authoryear]{elegantpaper}

% 标题
\title{Midterm - MATH 742}
\author{Boren(Wells) Guan}

% 本文档命令
\usepackage{array,url,stix}
\usepackage{subfigure}
\usepackage{tikz}
\usepackage{tikz-cd}
\newcommand{\ccr}[1]{\makecell{{\color{#1}\rule{1cm}{1cm}}}}
\newcommand{\code}[1]{\lstinline{#1}}
\newcommand{\prvd}{$\hfill \qedsymbol$}
\newcommand{\Z}{\mathbb{Z}}
\newcommand{\R}{\mathbb{R}}
\newcommand{\N}{\mathbb{N}}
\newcommand{\C}{\mathbb{C}}
\newcommand{\Q}{\mathbb{Q}}
\newcommand{\M}{\mathcal{M}}
\newcommand{\B}{\mathcal{B}}
\newcommand{\X}{\mathcal{X}}
\newcommand{\Hil}{\mathcal{H}}
\newcommand{\range}{\mathcal{R}}
\newcommand{\nul}{\mathcal{N}}

% 文档区
\begin{document}

% 标题
\maketitle

\subsection*{Before Reading:}\par
To make the proof more readable, I will miss or gap some natural or not important facts or notations during my writing. If you feel it hard to see, you can refer the appendix after the proof, where I will try to explain some simple conclusions (will be marked) more clearly. In case that you misunderstand the mark, I will add the mark just after those formulas between \$ and before those between \$\$.\par
And I have to claim that the appendix is of course a part of my assignment, so the reference of it is required. Enjoy your grading!

\subsection*{Problem.1}
\vspace{0.5em}
\textbf{Sol.} \par
(1) T (2) T (3) T (4) $\mathbb{Z}/4\mathbb{Z}$ (5) 2
\par 
\vspace{0.5em}

\subsection*{Problem.2} 
Let $R\to S$ be a homomorphism of rings. Suppose $S$ is projective as an $R$-module, and $M$ is a projective $S$-module. Show that $M$ is projective as an $R$-module.
\vspace{0.5em}\\
\textbf{Sol.} \par
    We know there exists an $R$-module $K$ such that $K\oplus S \cong R^{\Lambda}$ for some index set $\Lambda$ and an $S$-module $K'$ such that $K'\oplus M \cong S^{\Sigma}$ for some index set $\Sigma$, then we may know that \[(K^{\Sigma} \oplus K')\oplus M \cong K^{\Sigma} \oplus (K'\oplus M) \cong K^{\Sigma} \oplus S^{\Sigma} \cong (K\oplus S)^{\Sigma} \cong R^{\Lambda \times \Sigma}\] 
    and we are done.
\par 
\vspace{0.5em}

\subsection*{Problem.3} 
Consider the ring $R = \mathbb{Q}[x]$ and $S = \mathbb{Q}[t]$. Turn $S$ into an $R$-algebra using the homomorphism
\[\phi:R\to S:f(x) \mapsto f(t^2)\]
\begin{itemize}
\item Fix $a\in\mathbb{Q}$ and consider the ring
    \[S_a = S \otimes_R R/(x-a)\]
    How many maximal ideals does the ring have.
\item Find values of $a\in\mathbb{Q}$ for which $S_a$ is not reduced and describe the nilradical $\text{nil}(S_a)$.
\end{itemize}
\vspace{0.5em}
\textbf{Sol.} \par
    (1) We claim that
    \[S_a \cong S/(t^2-a)\]
    where define $\phi:S_a \to S/(t^2 - a)$ by $f\otimes r \mapsto [rf]$, which is obviously a well-defined linear homomorphism because $(f,r) \mapsto [rf]$ is bilinear. If $[rf] = 0$, then $ f\in (t^2 - a)$ or $r=0$, and we have $f\otimes r = 0$, which means $\phi$ is injective and obviously it is surjective, so the claim goes. Then we may know that if $a$ is not a square of some rational number, then $S_a$ has no maximal ideal and if else, $S_a$ will have only one maximal ideal.\par
    (2) We know any nilpotent element $pt+q$ will satisfy that there exists $p',q'\in\mathbb{Q}$ such that $[(pt+q)(p't+q')]= [pp' t^2 + (pq'+p'q)t + qq'] = [(pq'+p'q)t + app'+qq'] = 0$, which means $pq' + p'q = app'+qq' = 0$. We may assume $p,q$ nonzero and then let $d = p'/p = - q'/q$ and then $(ap^2 - q^2 )d = 0$ and hence $a = q^2/p^2$, so $a$ has to be a square of some rational number. If $a\neq 0$, assume $pt+q$ is nilpotent, we may know $p,q$ nonzero and hence $(pt+q)^k = p_kt + q_k$ satisfies that $p_k/p = - q_k/q$, however notice that $p_kq_k pq$ is always positive and hence a contradiction, so only for $a = 0$, $S_a$ is reduced and it is easy to chec khat $\text{nil}(S_0) = rt$ for some rational number $r$.
\vspace{0.5em}

\subsection*{Problem.4} 
Let $R$ be a ring and let $S, T \subset R$ be two multiplicative sets.\par
\begin{itemize}
    \item Suppose that, for any $s \in S$, there exists $t \in T$ such that $t \in (s)$. Construct a homomorphism of $R$-algebras $R[S^{-1}] \to R[T^{-1}]$.

    \item Conversely, suppose that a homomorphism of $R$-algebras $R[S^{-1}] \to
R[T^{-1}]$ exists. Show that $S$ and $T$ satisfy the above condition.
\end{itemize}
\vspace{0.5em}
\textbf{Sol.} \par
(a) Define $\phi:S^{-1}R \to T^{-1}R$ by $x/s \mapsto xr/t$ where $t = sr$ for some $r\in R$, then if $p = sm, m\in R$, then $xm/p = xr/t$ since $xmt = xmrs = xrp$ and hence $\phi$ is well-defined. Notice if $t = sr, t' = s'r'$, we have $\phi(x/s + y/s') = \phi((xs’+ys)/ss') = (xs'+ys)rr'/tt' = (xrt' + yr't)/tt' = xr/t + yr'/t' = \phi(x/s)+\phi(y/s') $ and $\phi(xy/ss') = xyrr'/tt' = xr/t\cdot yr'/t' = \phi(x/s)\phi(y/s')$. Finally, for any $q\in R$, $\phi(qx/s) = xrq/t = q\phi(x/s)$ and we know $\phi$ is a homomorphism.\par
(b) Assume $\phi$ is a homomorphism from $S^{-1}R \to T^{-1}R$, assume $\phi(1/s) = r/t$ and we know $sr/t = 1/1$ and hence $srv = tv$ for some $v \in T$ since $\phi$ is an $R$-algebra homomorphism and we are done.
\par 
\vspace{0.5em}

\subsection*{Problem.5} 
Show that any torsion theory $\mathcal{T} \subset R$-mod is a “tensor ideal”: for any $M \in \mathcal{T}$ and $N \in R$-mod, we have $M \otimes N \in \mathcal{T} $.
\vspace{0.5em}\\
\textbf{Sol.} \par
We know if there exists $\rho: M \to N$ where $M\in\mathcal{T}$, then $N\in \mathcal{T}$. Consider $\phi: M^{\oplus N} \to R^{M \oplus N}/K$ defined by $\phi\left(\sum\limits_{i=1}^k m_i^{(n_i)}\right) = \left[\sum\limits_{i=1}^k (m_i,n_i)\right]$, where $K = ((am+bm',n) - a(m,n)-b(m',n), m,m'\in M,n \in N,a,b\in R)$. Then $\phi$ is obviously a surjection and if $\left[\sum\limits_{i=1}^k (m_i,n_i)\right] = 0$ and hence $\sum\limits_{i=1}^k (m_i,n_i) = \sum\limits_{i=1}^m [(a_im_i+b_im'_i,n_i) - a_i(m_i,n_i)-b_i(m'_i,n_i)]$ for some $a_i,b_i\in R,m_i,m'_i\in M, n_i\in N$ which means that $\sum\limits_{i=1}^m (a_im_i+b_im_i' - a_i m_i - b_i m_i')^{(n_i)} = 0$ and hence $\phi$ is an isomorphism. Therefore, we have $R^{M\oplus N}/K \in \mathcal{T}$ and notice $M\otimes N$ is a quotient of $R^{M\oplus N}/ K$, then $M\otimes N \in \mathcal{T}$ and we are done. 
\par 
\vspace{0.5em}

\subsection*{Problem.6} 
Let $\mathcal{T} \in R$-mod be a torsion theory.\par
\begin{itemize}
    \item Show that any $M \in R$-mod contains a largest torsion submodule
$M_{\mathcal{T}} \subset M$ : that is, $M_{\mathcal{T}}$ is largest among all submodules $N \subset M$ such that $N \in T$ .
    \item Show that the correspondence $M\to M_{\mathcal{T}}$ defines a functor.
\end{itemize}
\vspace{0.5em}\par
\textbf{Sol.} \par
    (a) Let $M_{\mathcal{T}} = (N)_{N\in \mathcal{T}, N\subset M}$ and $M' = \bigoplus_{N\in \mathcal{T}, N\subset M} N$, then we define $\phi:\prod_{i=1}^k n_i\to \sum\limits_{i=1}^k n_i$ which is obviously a module homomorphism and surjection, so we know $M_{\mathcal{T}} \in \mathcal{T}$ and is obviously largest.\par
    (b) For any $\phi:M\to N$, let $\phi_{\mathcal{T}} = \phi|_{M_{\mathcal{T}}}: M_{\mathcal{T}} \to N_{\mathcal{T}}$ since $\phi(M_{\mathcal{T}}) \in \mathcal{T}$, which implies that $(1_M)_{\mathcal{T}} = 1_{M_{\mathcal{T}}}$ trivially. Then for any $\alpha:M\to N, \beta: N \to P$, we have $(\beta\alpha)_{\mathcal{T}} = (\beta\alpha)_{M_{\mathcal{T}}} = \beta_{N_{\mathcal{T}}}\alpha_{M_{\mathcal{T}}} = \beta_{\mathcal{T}} = \alpha_{\mathcal{T}}$ and we are done.
\par 
\vspace{0.5em}

\addappheadtotoc

\end{document}
