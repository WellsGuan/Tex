%!TEX program = xelatex
\documentclass[lang=en,11pt,a4paper,citestyle =authoryear]{elegantpaper}

% 标题
\title{Midterm - MATH 742}
\author{Boren(Wells) Guan}

% 本文档命令
\usepackage{array,url,stix}
\usepackage{subfigure}
\usepackage{tikz}
\usepackage{tikz-cd}
\newcommand{\ccr}[1]{\makecell{{\color{#1}\rule{1cm}{1cm}}}}
\newcommand{\code}[1]{\lstinline{#1}}
\newcommand{\prvd}{$\hfill \qedsymbol$}
\newcommand{\Z}{\mathbb{Z}}
\newcommand{\R}{\mathbb{R}}
\newcommand{\N}{\mathbb{N}}
\newcommand{\C}{\mathbb{C}}
\newcommand{\Q}{\mathbb{Q}}
\newcommand{\M}{\mathcal{M}}
\newcommand{\B}{\mathcal{B}}
\newcommand{\X}{\mathcal{X}}
\newcommand{\Hil}{\mathcal{H}}
\newcommand{\range}{\mathcal{R}}
\newcommand{\nul}{\mathcal{N}}

% 文档区
\begin{document}

% 标题
\maketitle

\subsection*{Before Reading:}\par
To make the proof more readable, I will miss or gap some natural or not important facts or notations during my writing. If you feel it hard to see, you can refer the appendix after the proof, where I will try to explain some simple conclusions (will be marked) more clearly. In case that you misunderstand the mark, I will add the mark just after those formulas between \$ and before those between \$\$.\par
And I have to claim that the appendix is of course a part of my assignment, so the reference of it is required. Enjoy your grading!

\subsection*{Problem.2} 
Let $R\to S$ be a homomorphism of rings. Suppose $S$ is projective as an $R$-module, and $M$ is a projective $S$-module. Show that $M$ is projective as an $R$-module.
\vspace{0.5em}\\
\textbf{Sol.} \par
    We know there exists an $R$-module $K$ such that $K\oplus S \cong R^{\Lambda}$ for some index set $\Lambda$ and an $S$-module $K'$ such that $K'\oplus M \cong S^{\Sigma}$ for some index set $\Sigma$, then we may know that \[(K^{\Sigma} \oplus K')\oplus M \cong K^{\Sigma} \oplus (K'\oplus M) \cong K^{\Sigma} \oplus S^{\Sigma} \cong (K\oplus S)^{\Sigma} \cong R^{\Lambda \times \Sigma}\] 
    and we are done.
\par 
\vspace{0.5em}

\subsection*{Problem 3} 
Consider the ring $R = \mathbb{Q}[x]$ and $S = \mathbb{Q}[t]$. Turn $S$ into an $R$-algebra using the homomorphism
\[\phi:R\to S:f(x) \mapsto f(t^2)\]
\begin{itemize}
\item Fix $a\in\mathbb{Q}$ and consider the ring
    \[S_a = S \otimes_R R/(x-a)\]
    How many maximal ideals does the ring have.
\item Find values of $a\in\mathbb{Q}$ for which $S_a$ is not reduced and describe the nilradical $\text{nil}(S_a)$.
\end{itemize}
\vspace{0.5em}
\textbf{Sol.} \par
    (1) We claim that
    \[S_a \cong S/(t^2-a)\]
    where define $\phi:S_a \to S/(t^2 - a)$ by $f\otimes r \mapsto [rf]$, which is obviously a well-defined linear homomorphism because $(f,r) \mapsto [rf]$ is bilinear. If $[rf] = 0$, then $ f\in (t^2 - a)$ or $r=0$, and we have $f\otimes r = 0$, which means $\phi$ is injective and obviously it is surjective, so the claim goes. Then we may know that if $a$ is not a square of some rational number, then $S_a$ has no maximal ideal and if else, $S_a$ will have only one maximal ideal.\par
    (2)
\vspace{0.5em}

\subsection*{Ex.3(9.22 on AK)} 
Let $R$ be a ring, $M$ a flat module, $R'$ an algebra. Show that $M\otimes_R R'$ is flat over $R'$.
\vspace{0.5em}\\
\textbf{Sol.} \par
Assume $N,L$ $R'$-modules and $\gamma: N\to L$ injection and we would like to check that the induced $\gamma':(M\otimes_R R')\otimes_{R'} N \to (M\otimes_R R')\otimes_{R'} L$ is injective. We consider the commutative diagram, which is easy to check and
\[
\begin{tikzcd}
    (M\otimes_R R')\otimes_{R'} N\arrow[d,"\cong"]\arrow[r] & (M\otimes_R R')\otimes_{R'} L\arrow[d,"\cong"] \\
M \otimes_R N\arrow[r] & M\otimes_R L \\
\end{tikzcd}
\]
where $\gamma''(m,n) \mapsto \gamma'((m,1),n) = ((m,1),\gamma n)\mapsto (m,\gamma_n)$
which means the map $M\otimes_R N \to M\otimes_R L$ in the commutative diagram is exact the induced map $\gamma''$ and which is injective by $M$ is flat and hence $(M\otimes_R R')\otimes_{R'} N\to (M\otimes_R R')\otimes_{R'} L$ is injective.
\par 
\vspace{0.5em}

\subsection*{Ex.4(9.27 on AK)} 
Let $R$ be a ring, $I$ an ideal. Assume $R/I$ is flat, then show $I = I^2$.
\vspace{0.5em}\\
\textbf{Sol.} \par
Consider $I\hookrightarrow R$ the inclusion and then we know $R/I \otimes I \to R/I \otimes I$ inclusion and hence we consider $(a+I,l)$ as an element in $R/I\otimes R$ which is $0$ and hence $R/I \otimes I$ is zero, which is isomorphic to $I/I^2$.
\par 
\vspace{0.5em}

\subsection*{Ex.5(Problem A)} 
Let $F$ be a functor from a category $\mathcal{C}$ to $((\text{Sets}))$. Show that the functor is represented by $a\in\mathcal{C}$ iff there exists an element $\alpha\in F(a)$ with the following property: for any $b\in\mathcal{C}$ and any $\beta$ in $F(b)$, there exists unique $f:a\to b$ such that $F(f)(\alpha) = \beta$.
\vspace{0.5em}\par
\textbf{Sol.} \par
If $F = \hom(a,\cdot)$, then conclusion is trivial. Conversely, assume $F$ satisfies the property, then we may consider $\theta(b):\hom(a,b) \to F(b)$ by $\theta(b)(\gamma) = F(\gamma)(\alpha)$ and $\theta'(b):F(b) \to \hom(a,b)$ as $\theta(b)(\beta)$ to be the unique $\gamma$ such that $F(\gamma)(\alpha) = \beta$, then consider the diagram
\[
\begin{tikzcd}
    \hom(a,m)\arrow[r]\arrow[d,"\theta(m)"] & \hom(a,n)\arrow[d,"\theta(n)"] \\    
    F(m)\arrow[r]\arrow[d,"\theta'(m)"] & F(n)\arrow[d,"\theta'(n)"] \\
    \hom(a,m)\arrow[r] & \hom(a,n) \\
\end{tikzcd}
\]
which we may know that for any $\gamma:a\to m$ and $\delta: m\to n$, we have
\[
\theta(n)(\delta(\gamma)) = F(\delta\gamma)(\alpha) = (F(\delta)F(\gamma))(\alpha) = F(\delta)(F(\gamma)(\alpha))
\]
and for any $\beta \in F(m)$
\[
\gamma\theta'(m)(\beta) = \gamma\phi = \theta'(n)(F(\gamma\phi)(\alpha)) = \theta'(n)(F(\gamma)(\beta))
\]
where $\phi$ is the unique map from $a$ to $m$ such that $F(\phi)(\alpha) = \beta$ and we are done by checking that $\theta'\theta = 1_{\hom(a,\cdot)}$ and $\theta\theta' = 1_F$.
\par 
\vspace{0.5em}

\subsection*{Ex.6(Problem B)} 
Show that the algebra $\mathbb{C}\otimes_{\mathbb{R}}\mathbb{C}$ is isomorphic to $\mathbb{C}\times\mathbb{C}$.
\vspace{0.5em}\\
\textbf{Sol.} \par
Notice that we have $1\otimes i, i\otimes 1, 1\otimes 1, i\otimes i$ is a free basis of $\mathbb{C}\otimes \mathbb{C}$. Define
\[
\phi: (a+bi)\otimes(c+di) \to (ac + bdi, ad+bci)
\]
where it is easy to check $\phi$ is well-defined and a lienar map, which is obviously a surjection and if $ac = bd = 0, ad = bc = 0$, we may know three of $a,b,c,d$ are $0$ and hence it is an injection and we are done.
\par 
\vspace{0.5em}

\addappheadtotoc

\end{document}
