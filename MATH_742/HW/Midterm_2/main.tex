%!TEX program = xelatex
\documentclass[lang=en,11pt,a4paper,citestyle =authoryear]{elegantpaper}

% 标题
\title{Midterm2 - MATH 742}
\author{Boren(Wells) Guan}

% 本文档命令
\usepackage{array,url,stix}
\usepackage{subfigure}
\usepackage{tikz}
\usepackage{tikz-cd}
\newcommand{\ccr}[1]{\makecell{{\color{#1}\rule{1cm}{1cm}}}}
\newcommand{\code}[1]{\lstinline{#1}}
\newcommand{\prvd}{$\hfill \qedsymbol$}
\newcommand{\Z}{\mathbb{Z}}
\newcommand{\R}{\mathbb{R}}
\newcommand{\N}{\mathbb{N}}
\newcommand{\C}{\mathbb{C}}
\newcommand{\Q}{\mathbb{Q}}
\newcommand{\M}{\mathcal{M}}
\newcommand{\B}{\mathcal{B}}
\newcommand{\X}{\mathcal{X}}
\newcommand{\Hil}{\mathcal{H}}
\newcommand{\range}{\mathcal{R}}
\newcommand{\nul}{\mathcal{N}}

% 文档区
\begin{document}

% 标题
\maketitle

\subsection*{Before Reading:}\par
To make the proof more readable, I will miss or gap some natural or not important facts or notations during my writing. If you feel it hard to see, you can refer the appendix after the proof, where I will try to explain some simple conclusions (will be marked) more clearly. In case that you misunderstand the mark, I will add the mark just after those formulas between \$ and before those between \$\$.\par
And I have to claim that the appendix is of course a part of my assignment, so the reference of it is required. Enjoy your grading!

\subsection*{Problem.1}
\ \ \ 
\vspace{0.5em}\\
\textbf{Sol.} \par
Only $(\mathbb{Z}/p\mathbb{Z})(z)$ is not.
\par 
\vspace{0.5em}

\subsection*{Problem.2} 
Set $\alpha = \sqrt{2-\sqrt{2}}$.
\begin{enumerate}
    \item [(a)] Find the minimal polynomial of $\alpha$ over $\mathbb{Q}$.
    \item [(b)] Show $\mathbb{Q}(\alpha) \supset \mathbb{Q}$ is a Galois extension.
    \item [(c)] Find the Galois group $\text{Gal}(\mathbb{Q}(\alpha)/\mathbb{Q})$.
\end{enumerate}
\vspace{0.5em}
\textbf{Sol.} \par
   (a)We know $\alpha$ is a root of $(x^2-2)^2 - 2 = x^4 - 4x^2 +2$, which is irreducible by choosing $p=2$ and use the Eisenstein's Criterion. So $f(x) = x^4-4x^2 + 2$ is a minimal polynomial of $\alpha$ over $\mathbb{Q}$.\par
   (b) We know roots of $x^4-4x^2 +2$ are $\pm\sqrt{2+\sqrt{2}},\pm\sqrt{2-\sqrt{2}}$ and notice $\sqrt{2+\sqrt{2}}\sqrt{2-\sqrt{2}} = \sqrt{2} = 2-\alpha^2$ which means $\sqrt{2+\sqrt{2}} \in \mathbb{Q}[\alpha]$ and hence $\mathbb{Q}[\alpha]$ is generated by $\pm\sqrt{2+\sqrt{2}},\pm\sqrt{2-\sqrt{2}}$, which means it is a splitting field of a separable polynomial. Therefore, it is a Galois extension.\par
   (c) Since $\mathbb{Q}[\alpha]\cong \mathbb{Q}[x]/(x^4-4x^2+2)$, it has the degree of $4$ and hence $\text{Gal}(\mathbb{Q}[\alpha]/\mathbb{Q})$ has the order of $4$. Assume $\phi \in \text{Gal}(\mathbb{Q}[\alpha]/\mathbb{Q})$, then $\phi(\alpha) \in \{\pm\sqrt{2+\sqrt{2}},\pm\sqrt{2-\sqrt{2}}\}$. Notice $\sqrt{2+\sqrt{2}} = (2-\alpha^2)\alpha^{-1}$, so $\phi$ is uniquely determined by $\phi(\alpha)$ and then there has to be some $\varphi \in \text{Gal}(\mathbb{Q}[\alpha]/\mathbb{Q})$ such that $\varphi(\alpha) = \sqrt{2+\sqrt{2}}$ then it is easy to check $\varphi^k, k = 0, 1,2,3$ are distinct with $\varphi^4 = id$ and hence $\text{Gal}(\mathbb{Q}[\alpha]/\mathbb{Q}) = D_2$.
\par 
\vspace{0.5em}

\subsection*{Problem.3} 
Let $E \supset \mathbb{Q}$ be a degree four extension. Show that it is impossible for it to have exactly two intermediate fields
$\mathbb{Q} \subsetneq M \subsetneq E$.\\
\vspace{0.5em}
\textbf{Sol.} \par
    Assume $\alpha \in M$ and then there are a basis $\{e_i\}_{i=1}^4$ of $E$ such that $e_1 = a$, then we know $[E:\mathbb{Q}[a]][\mathbb{Q}[a]:\mathbb{Q}] = 4$ and hence the minimal polynomial of $a$ can be only have degree of $1,2$ or $4$. Let $M_1,M_2$ be two intermediate extensions and they may only have elements with minimal polynomials of degree $2$ and $1$, and at least one element with minimal polynomial of degree $2$, or $M_i$ will have the same degree with $E$ or $\mathbb{Q}$, which hence a contradiction. Assume $\alpha \in M_1, \beta \in M_2$ with $\alpha,\beta$ have minimal polynomial of degree $2$, then we know $\{1,\alpha\},\{1,\beta\}$ can be relatively bases of $M_1,M_2$ and hence $\alpha \neq \beta$. Assume $(x+p)^2-q , x(x+m)^2-n, p,q,m,n\in \mathbb{Q}$ are relatively the minimal polynomials of $\alpha$ and $\beta$, we will know $(\alpha+p)(\beta+m)^2 = qn$. If $qn = r^2$ for some rational number $r$, then we know $(\alpha+p)^2(\beta+m)^2 = r^2$ and then $(\alpha+p)(\beta+m) = \pm r$ which means $b\in M_1$ and then $M_2\subset M_1$, similarly $M_1\subset M_2$ which is a contradiction, so $\gamma:=(\alpha+p)(\beta+m)$ is a root of $x^2 - qn$ which is irreducible in $\mathbb{Q}[X]$. If $\mathbb{Q}[\gamma] = M_1$, then we know $s\alpha+t = \alpha\beta + p\beta+m\alpha+pm$ and hence $\beta(\alpha+p) \in M_1$ and hence $\beta \in M_1$ which is a contradiction. Similarly $\mathbb{Q}[\gamma] \neq M_2$ and there is a contradiction and we are done.
\vspace{0.5em}

\subsection*{Problem.4} 
Let $n>1$, and set $E=\mathbb{C}(x)$ and $F=\mathbb{R}(x^n)$. Prove that $E \supset F$ is a finite Galois extension and find its Galois group.
\vspace{0.5em}\\
\textbf{Sol.} \par
Assume $\phi:x\mapsto e^{i2\pi/n}$ and $\varphi(\sum\limits_{i=0}^nc_i x^i) = \sum\limits_{i=0}^n\overline{c_i} x^i$. $\phi,\varphi$ are easy to be checked isomorphisms on $\mathbb{C}[x]$ and hence they induce isomorphisms on $\mathbb{C}(x)$, which can be seen by that $\phi,\varphi$ are obviously surjection and if $\phi(a/b) = \phi(a)/\phi(b) =  0$, then $\phi(a) = 0$ which means $a = 0$ and we know $\phi$ is injection, the same for $\varphi$. Let $G = [\phi,\varphi]$ and notice that $\mathbb{R}(x^n)$ is a subfield of $E^G$. Notice $\mathbb{C}(x)= \mathbb{R}(x)[i]$ and $\mathbb{R}(x) = \mathbb{R}(x^n)[x]$ where $X^2 + 1$ has a root $i$ and $Y^n - x^n$ has a root $x$. $X^2+1$ is irreducible in $\mathbb{R}(x)[X]$ or $i$ will be in $\mathbb{R}(x)$ which is a contradcition. $Y^n - x^n$ should be irreducible in $\mathbb{R}(x^n)[Y]$ or there should be some $g\in \mathbb{R}(x^n)[Y]$ with degree less than $n$ such that $g(x) = 0$, in particular, $f_0 + f_1 x + \cdots + f_k x^k = 0$ for some $f_i \in \mathbb{R}(x^n)$, we may assume $f_i \in \mathbb{R}[x^n]$ by multiplying the product of all the denominators of $f_i$ and then $f_i(0) = 0$, then we know $x^n|(f_0 + f_1 x + \cdots + f_k x^k)$ and we may use the induction to see that $f_i = 0$. Now we have $[E:F] = [E:\mathbb{R}(x)][\mathbb{R}(x^n):F] = 2n$. However $[E:F] = [E:E^G][E^G:F]$ and hence $F = E^G$, so $E$ is Galois over $F$. We claim that $G \cong \mathbb{Z}/2\mathbb{Z} \times\mathbb{Z}/n\mathbb{Z}$ since $\phi\varphi = \varphi\phi$ and the isomorphism is easy to be checked.
\par 
\vspace{0.5em}

\subsection*{Problem.5} 
Let $E \supset F$ be a Galois extension such that $\text{Gal}(E/F) \cong S_5$. Prove that $E$ is the splitting field of a degree five polynomial over $F$.
\vspace{0.5em}\\
\textbf{Sol.} \par
    Firstly we know $[E:F] = 5!$ and assume $M = E^{S_4}$, then we have $[E^{S_4}:F] = 5$ and then we may consider $\alpha \in E^{S_4} - F$ with a minimal polynomial, whose degree can be only a factor of $5$ and hence $5$. Let $E'$ the splitting field of the minimal polynomial of $\alpha$. So we know $1,\alpha,\alpha^2,\alpha^3,\alpha^4$ form a basis of $E^{S_4}$ and hence $E'/E^{S_4}$ Galois, which is from $E$ is Galois, so separable and hence the minimal polynomial of $\alpha$ separable and hence $E'$ is Galois over $F$. So $E'$ is normal and hence $\text{Gal}(E/E')$ is normal, which means it should be $A_5$ or $\{id\}$ only, however $[E:E'] \leq 5!/[E':F] \leq 24$ and hence it will be $\{id\}$ and we know $E= E'$.
\par 
\vspace{0.5em}

\subsection*{Problem.6} 
Let $E \supset F$ be a finite Galois extension, and suppose $f(x) \in F[x]$ is an irreducible polynomial. Let $f(x) = f_1(x)f_2(x)\cdots f_k(x)$ be the factorization of $f(x)$ in $E[x]$. Prove that all polynomials $f_i$ have the same degree, and that $k|[E:F]$.
\vspace{0.5em}\\
\textbf{Sol.} \par
   Assume $\phi$ is any element in $\text{Gal}(E/F)$ and let $\phi g$ to be the polynomila with $\phi$ acts on each coeeficient of $g$. We have $f = \phi f = \prod_{i=1}^k \phi f_i$. We know there is always some $\phi_i$ such that $\phi_i f_i = f_j$ for some $f_j \neq i$ or $f_i \in F[X]$ which is a contradiction since $f$ is irreducible in $F[X]$. We denote $f_i ~ f_j$ if there is $\phi$ such that $\phi f_i = f_j$, then we can show that $~$ is an equivalence relation and WLOG $f_1,\cdots,f_m$ are in a same equivalence class, then $f_1\cdots f_m$ is invariant under $G$ and hence in $F[X]$, so $m = k$ and notice any equivalent $f_i$ will have equal degree. $\{f_1,\cdots,f_k\} = Gf_1$ and then $k = |Gf_1| | |G| = [E:F]$ and we are done.
\par 
\vspace{0.5em}

\addappheadtotoc

\end{document}
