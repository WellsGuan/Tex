%!TEX program = xelatex
\documentclass[lang=en,11pt,a4paper,citestyle =authoryear]{elegantpaper}

% 标题
\title{Homework01 - MATH 742}
\author{Boren(Wells) Guan}

% 本文档命令
\usepackage{array,url,stix}
\usepackage{subfigure}
\usepackage{tikz}
\usepackage{tikz-cd}
\newcommand{\ccr}[1]{\makecell{{\color{#1}\rule{1cm}{1cm}}}}
\newcommand{\code}[1]{\lstinline{#1}}
\newcommand{\prvd}{$\hfill \qedsymbol$}
\newcommand{\Z}{\mathbb{Z}}
\newcommand{\R}{\mathbb{R}}
\newcommand{\N}{\mathbb{N}}
\newcommand{\C}{\mathbb{C}}
\newcommand{\Q}{\mathbb{Q}}
\newcommand{\M}{\mathcal{M}}
\newcommand{\B}{\mathcal{B}}
\newcommand{\X}{\mathcal{X}}
\newcommand{\Hil}{\mathcal{H}}
\newcommand{\range}{\mathcal{R}}
\newcommand{\nul}{\mathcal{N}}

% 文档区
\begin{document}

% 标题
\maketitle

\subsection*{Before Reading:}\par
To make the proof more readable, I will miss or gap some natural or not important facts or notations during my writing. If you feel it hard to see, you can refer the appendix after the proof, where I will try to explain some simple conclusions (will be marked) more clearly. In case that you misunderstand the mark, I will add the mark just after those formulas between \$ and before those between \$\$.\par
And I have to claim that the appendix is of course a part of my assignment, so the reference of it is required. Enjoy your grading!

\subsection*{Ex.1} 
Let $R$ be a ring. Suppose $f_0\in R, f_1\in R[x_1],\cdots,f_n \in R[x_1,\cdots,x_n]$ be polynomials. Prove that the quotient
\[
R[x_1,\cdots,x_{n+1}]/(x_1-f_0,x_2-f_1(x_1),\cdots,x_{n+1}-f_n(x_1,\cdots,x_n))
\]
is isomorphic to $R$.
\vspace{0.5em}\\
\textbf{Sol.} \par
Define $\pi:R[x_1,\cdots,x_{n+1}] \to R$ by \[\pi(f(x_1,\cdots,x_{n+1})) = f(f_0,f_1(f_0),\cdots,f_{n}(f_0,\cdots,f_{n-1}(\cdots)))\]
and then we know
\[
\pi(\langle x_1-f_0,\cdots,x_{n+1}-f_n(x_1,\cdots,x_n)\rangle) = 0
\]
Let show that for any $F\in R[x_1,\cdots,x_{n+1}]$, there exists $H_i, 1\leq i\leq n,r\in R$ such that \[F = (x_1-f_0)H_1+\cdots+(x_{n+1}-f_n(x_1,\cdots,x_n))H_n+r\]
Notice
\[
\begin{aligned}
    I & = \langle x_1-f_0,\cdots,x_{n+1}-f_n(x_1,\cdots,x_n)\rangle\\ &= \{
F|\exists h_i\in P, 1\leq i\leq n+1, F = (x_1-f_0)h_1 + \cdots (x_{n+1}-f_n(x_1,\cdots,x_n))h_{n+1}
\}
\end{aligned}
\]
where $P = R[x_1,\cdots,x_{n+1}]$ and it is easy to check that the right hand side is an ideal contained in $I$ and containing $x_1-f_0,\cdots,x_{n+1}-f_n(x_1,\cdots,x_n)$. Notice $x_1 = (x_1-f_0)+f_0 = s_1 + f_0$ where $s_1\in I$, and if $x_i = g_i+r_i, g_i\in I, r_i\in R,1\leq i\leq k$ and then $x_{k+1} = (x_{k+1}-f_k(x_1,\cdots,x_k)) + f_k(x_1,\cdots,x_k) = s_k + f_k(g_1+r_1,\cdots,g_n+r_n) = s_{k+1}+f_k(r_1,\cdots,r_n)$ where $s_k,s_{k+1}\in I$ and then we may use the induction to show that there exists $s_k\in I,r_k\in R$ such that $x_k = s_k+r_k, 1\leq k\leq n+1$. So $F(x_1,\cdots,x_{n+1}) = F_g+F(r_1,\cdots,r_{n+1})$ for some $F_g\in I$ and hence we know $\pi(F) = 0$ iff $F(r_1,\cdots,r_{n+1}) = 0$ iff $F\in I$ and hence
\[
R[x_1,\cdots,x_{n+1}]/I = R
\]
by (1.5) on AK. 
\par 
\vspace{0.5em}

\subsection*{Ex.2} 
An $R$-algebra $S$ is said to be finitely generated if it is isomorphic to  quotient of a polynomial algebra $R[x_1,\cdots,x_n]$ for some integer $n$.\par
Show that a quotient of a finitely generated algebra is finitely generated.
\vspace{0.5em}\\
\textbf{Sol.} \par
Consider $S\cong (P/I)/I'$ where $P$ is a polynomial algebra, $I$ is an ideal in $P$ and $I'$ is an ideal in $P/I$. Let $\kappa: P\to P/I$ the quotient map and we know $J := \kappa^{-1}(I')$ is an ideal in $P$ and then $J/I = \kappa(J) = I'$ and hence
\[
S \cong (P/I)/I' \cong (P/I)/(J/I) \cong (P/J)
\]
by (1.9) on AK.
\par 
\vspace{0.5em}

\subsection*{Ex.3} 
Show that the set $\mathbb{R}+(xy) \subset \mathbb{R}[x,y]$ is an $\mathbb{R}$-subalgebra, and that it is not finitely generated.
\vspace{0.5em}\\
\textbf{Sol.} \par
For any $r_1,r_2\in \mathbb{R}, g_1,g_2\in (xy)$, we have
\[
\begin{aligned}
    (r_1+g_1)+(r_2+g_2) &= (r_1+r_2)+(g_1+g_2)\in \mathbb{R}+(xy) \\ (r_1+g_1)(r_2+g_2) &= r_1r_2 + (r_1g_2+r_2g_1+g_1g_2)\in\mathbb{R}+(xy)
\end{aligned}
\]
and $1\in \mathbb{R}+(xy)$ is a unit and hence $\mathbb{R}+(xy)$ if a subring of $\mathbb{R}[x,y]$, and obviously the commutative diagram
\[
\begin{tikzcd}
\mathbb{R}\arrow[r]\arrow[rd] & \mathbb{R} + (xy)\arrow[hookrightarrow]{d}\\
&\mathbb{R}[x,y]
\end{tikzcd}
\]
commutes and hence $\mathbb{R}+(xy)$ is a subalgebra.\par
Assume
\[
\mathbb{R}+(xy)\overset{\phi}{\cong} \left(\mathbb{R}/[x_1,\cdots,x_n]\right)/I
\]
then assume
\[
\phi^{-1}(x_i+I) = r_i+\sum\limits_{k=2}^{\infty}\sum\limits_{j=1}^{k-1}m_{kj}^{(i)}x^jy^{k-j}
\]
where $r_i,m^{(i)}_{kj} \in \mathbb{R}$ and $m_{kj}^{(i)} = 0$ for all $k\geq k_i \in \mathbb{N}$ for some integer $k_i$. Notice for any product of $\{x_i\}$, we have
\[
\phi^{-1}(\prod_{\lambda \in\Lambda} x_\lambda+I) = r+\sum\limits_{\lambda \in \Lambda}\sum\limits_{k=2}^{\infty}r_{\lambda}m_{k1}^{(\lambda)}xy^{k-1} + g
\]
where $g\in (x^2)\cap\left(\mathbb{R}+(xy)\right)$
which means there can not be linear combination of products in $\{x_i\}_{i=1}^n$ such that the combination equals to $xy^M$ where $M = \max\{k-1, m_{k1}^{\lambda} \neq 0, 1\leq \lambda \leq n\}$ and which is a contradiction by isomorphism. Therefore $\mathbb{R}+(xy)$ can not be finitely generated.
\par 
\vspace{0.5em}

\subsection*{Ex.4(1.14 on AK)} 
Let $\phi:R\to R'$ be a map of rings, $I$ an ideal of $R$ and $J$ an ideal of $R'$. Prove the following statements
\begin{itemize}
    \item $I^{ec}\supset I$ and $J^{ce}\subset J$
    \item $I^{ece} = I^e, J^{cec} = J^c$
    \item If $J$ is an extension, then $J^c$ is the largest ideal of $R$ with extension $J$
    \item If two extensions have the same contraction, then they are equal
\end{itemize}
\vspace{0.5em}
\textbf{Sol.} \par
(a) Notice $I\subset \phi^{-1}(\phi(I))\subset^{\phi^{-1}I^e} = I^ec$ and $\phi(J^c) \subset J$ implies $J \supset J^{ce}$ since $J^{ce}$ is the minimal ideal containing $J^{c}$.\par
(b) By (a), we have $(I^{e})^{ce} \subset I^e$ and $(I^{ec})^e \supset I^e$. Similarly $(J^{ce})^c  \subset J^c$ and $(J^c)^{ec} \supset J^c$.\par
(c) If $J = I^c$ for some $I\subset R$ an ideal and $I'\subset R$ is an ideal such that $I'^e = J$, then $J^c = I'^{ec} \supset I'$ and we are done.\par
(d) If $I,I'$ ideals of $R$ such that $J:= I^e, J':= I'^e$ satisfy $J^c = J'^c$, then by (c) we know $J^c \supset J'c$ and hence $J\supset J'$ and similarly $J'\supset J$ and we are done.
\par 
\vspace{0.5em}

\subsection*{Ex.5(1.21 on AK)} 
Let $R$ be a ring.
\begin{itemize}
    \item Let $I$ and $J$ be comaximal ideals, i.e. $I+J = R$. Then
        \[IJ = I\cap J\quad\text{and}\quad R/IJ = (R/I)\times (R/J)\]
    \item Let $I$ be comaximal to both $J$ and $J'$. Show $I$ is also comaximal to $JJ'$
    \item Given $m,n\geq 1$, show $I$ and $J$ are comaximal iff $I^m$ and $J^n$ are
    \item Let $I_1,\cdots,I_n$ be pairwise comaximal. Show (a)$I_1$ and $I_2I_3\cdots I_n$ are comaximal, (b)$I_1\cap\cdots\cap I_n = I_1\cdots I_n$ and (c) $R/(I_1\cdots I_n) \cong \prod (R/I_i)$
\end{itemize}
\vspace{0.5em}
\textbf{Sol.} \par
(1) Since for any $a\in I, b\in J$ $ab\in I\cap J$ and $I\cap J$ is an ideal, we know $IJ \subset I\cap J$. For $c\in I \cap J$, since $I+J = R$, $1 = e_I+e_J$ for some $e_I\in I, e_J\in J$ and then $a = ae_I + ae_J \in IJ$ and we are done.\par
Consider 
\[
\phi:R/IJ \to (R/I) \times (R/J), a+IJ\mapsto (a_J+I)\times (a_I+J)
\]
where $a_I\in I + a_J\in J = a$. This map is well-defined since if there exists $b_I\in I, b_J\in J$ such that $b = b_I+b_J$ and $a-b\in IJ$, then $a_I+a_J - b_I+b_J \in IJ$ which implies $a_I-b_I \in I, a_J-b_J \in J$ and hence $\phi$ is well-defined. To see it is a ring homomorphism, notice for $a,b\in R$,
\[
\begin{aligned}
    \phi((a+b)+IJ) &= ((a_J+b_J)+I,(a_I+b_I)+J) = (a_J+I,a_I+J)+(b_J+I,b_I+J)\\ &= \phi(a+IJ)+\phi(b+IJ) \\
    \phi(a+IJ)\phi(b+IJ) &= (a_J+I,a_I+J)(b_J+I,b_I+J) = (a_Jb_J + I, a_Ib_I+J)\\ &= (a_Jb_J+a_Ib_J+I,+a_Ib_I+a_Jb_I+J) = \phi(ab+IJ) 
\end{aligned}
\]
It obvious that $\phi$ is surjective and If $a\in \ker(\phi)$, then $a_J\in I, a_I\in J$ and hence $a\in IJ$ and we are done.\par
(2) For any $a\in R$, there exists $a_I\in I,a_J\in J,e_{I}'\in I,e_{J'}\in J'$ such that 
\[a_I+a_J = a,\quad e_I+e_{J'} = 1\]
and then
\[
a = a\cdot 1 = (a_I+a_J)(e_I+e_{J'}) = (a_Ie_I+a_Ie_{J'}+e_Ia_J) + (a_Je_{J'}) \in I+JJ'
\]
which measn $I$ is comaximal to $JJ'$.\par
(3) For the sufficiency, we may know $I$ is comaximal to $J^n$  for any integer $n$ and then $I^m$ is comaxial to $J^n$ for any integer $m$. For the necessity, notice $I^m \subset I, J^n \subset J$ and hence $I+J\supset I^m+J^n =R$ and we are done.\par
(4)(a) By $(2)$ we know if $I_1$ is comaximal with $I_2\cdots I_k$, then it is comaximal wtih $I_2\cdots I_{k+1}$ and we are done by induction. We may similarly prove (b), assume $I_1\cap\cdots\cap I_k = I_1\cdots I_k$, then $I_1\cap\cdots \cap I_{k+1} = (I_1\cdots I_k)\cap I_{k+1} = (I_1\cdots I_k)I_{k+1} = I_1\cdots I_{k+1}$ by (1) and (2) and we are done. For (c), assume $R/(I_1\cdots I_k) \cong \prod (R/I_i)$, then \[\prod_{i=1}^{k+1} (R/I_i) \cong (R/(I_1\cdots I_{k}))\times(R/I_{k+1}) \cong R/[(I_1\cdots I_k)I_{k+1}] = R/(I_1\cdots I_{k+1})\]
and we are done.
\par 
\vspace{0.5em}

\subsection*{Ex.6(1.22 on AK)} 
\begin{itemize}
    \item Given a prime number $p$ and $k\geq 1$, find the idempotents in $\mathbb{Z}/\langle p^k\rangle$
    \item Find the idempotents in $\mathbb{Z}/\langle 12\rangle$
    \item Find the number of idempotents in $\mathbb{Z}/\langle n\rangle$ where $n = \prod_{i=1}^N p_i^{n_i}$ with $p_i$ distinct prime numbers.
\end{itemize}
\vspace{0.5em}
\textbf{Sol.} \par
(a) If $p^k|e(e-1)$, then we know either $p|e$ or $p|e-1$ and hence either $p^k|e$ or $p^k|e-1$, which means $\text{Idem}(\mathbb{Z}/\langle p^k \rangle) = \{0,1\}$.\par
(b) If $12|e(e-1)$, then $12|e$ or $12|e-1$ or $3|e,4|e-1$ or $4|e,3|e-1$ and hence $e\in\{0,1,4,9\}$.
\par 
(c) We know $n|e(e-1)$. There are $2^M$ idempotents in $\mathbb{Z}/\langle n \rangle$. (Use the complete system of residues).
\vspace{0.5em}

\subsection*{Ex.7(1.23 on AK)} 
Let $R:=R'\times R''$ be a product of rings, $I\subset R$ an ideal. Show $I = I'\times I''$ with $I'\subset R'$ and $I''\subset R''$ ideals. Show $R/I = (R'/I')\times(R''/I'')$.
\vspace{0.5em}\\
\textbf{Sol.} \par
For any $(a,b) \in I$, we know
\[
(a,0_{R''}) = (a,b)(1_{R'},0_{R''}) \in I,\quad (0_{R'},b) = (a,b)(0_{R'},1_{R''})\in I
\]
Denote
\[
I' = \{r\in R',\exists r'\in R'', (r,r')\in I\},\quad I'' = \{r\in R',\exists r'\in R', (r',r)\in I\}
\]
and it is easy to check $I',I''$ are ideals in $R$. For any $a\in I',b\in I''$, by the conclusion before we may know $(a,0),(0,b) \in I$ and hence $(a,b)\in I$, which means $I\supset I'\times I''$, and obviously $I\subset I'\times I''$ and we are done.\par
Assume
\[
\phi:(R'/I')\times (R''/I'') \to R/I
\]
defined by
\[
(a+I',b+I'')\mapsto ((a,b)+I)
\]
which is easily to be checked a well-defined homomorphism. Also, $\phi$ is naturally surjective and if $(a+I',b+I'')\in \ker(\phi)$, we will have $(a,b) \in I$ and hence $a\in I',b\in I''$ which means $\phi$ is injective and we are done.
\vspace{0.5em}

\subsection*{Ex.8(2.23 on AK)} 
Let $I$ and $J$ be ideals and $P$ a prime ideal. Prove that the followings are equivalent
\begin{itemize}
    \item $I\subset P$ or $J\subset P$
    \item $I\cap J \subset P$
    \item $IJ \subset P$
\end{itemize}
\vspace{0.5em}
\textbf{Sol.}\par
(1) $\implies$ (2) $\implies$ (3) is trivial. Only need to check (3) $\implies$ (1), if there is $a\in I, b\in J$ such that $a,b\notin P$, then $ab\in IJ \subset P$ implies $ab\in P$, which means $a\in P$ or $b\in P$ which is a contradiction and hence $I\subset P$ or $J\subset P$.
\vspace{0.5em}


\addappheadtotoc

\end{document}
