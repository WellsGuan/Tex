%!TEX program = xelatex
\documentclass[lang=en,11pt,a4paper,citestyle =authoryear]{elegantpaper}

% 标题
\title{Homework09 - MATH 742}
\author{Boren(Wells) Guan}

% 本文档命令
\usepackage{array,url,stix}
\usepackage{subfigure}
\usepackage{tikz}
\usepackage{tikz-cd}
\newcommand{\ccr}[1]{\makecell{{\color{#1}\rule{1cm}{1cm}}}}
\newcommand{\code}[1]{\lstinline{#1}}
\newcommand{\prvd}{$\hfill \qedsymbol$}
\newcommand{\Z}{\mathbb{Z}}
\newcommand{\R}{\mathbb{R}}
\newcommand{\N}{\mathbb{N}}
\newcommand{\C}{\mathbb{C}}
\newcommand{\Q}{\mathbb{Q}}
\newcommand{\M}{\mathcal{M}}
\newcommand{\B}{\mathcal{B}}
\newcommand{\X}{\mathcal{X}}
\newcommand{\Hil}{\mathcal{H}}
\newcommand{\range}{\mathcal{R}}
\newcommand{\nul}{\mathcal{N}}

% 文档区
\begin{document}

% 标题
\maketitle

\subsection*{Before Reading:}\par
To make the proof more readable, I will miss or gap some natural or not important facts or notations during my writing. If you feel it hard to see, you can refer the appendix after the proof, where I will try to explain some simple conclusions (will be marked) more clearly. In case that you misunderstand the mark, I will add the mark just after those formulas between \$ and before those between \$\$.\par
And I have to claim that the appendix is of course a part of my assignment, so the reference of it is required. Enjoy your grading!

\subsection*{Ex.1(2.3 on JSM)}
Construct a splitting field for $X^5-2$ over $\mathbb{Q}$. What is its degree over $\mathbb{Q}$?
\vspace{0.5em}\\
\textbf{Sol.} \par
\iffalse 
    It is $\mathbb Q(\sqrt[5]{2},e^{\frac{2\pi i}{5}})$. Clearly $\sqrt[5]{2},\sqrt[5]{2}e^{\frac{2\pi i}{5}}$ are in the splitting field, so $e^{\frac{2\pi i}{5}}$ is in the splitting field. However, $\sqrt[5]{2}(e^{\frac{2\pi i}{5}})^j$ for $j=0,1,2,3,4$ are all the roots. 

Clearly, $[\mathbb Q(\sqrt[5]{2}):\mathbb Q]=5$ since $x^5-2$ is irreducible over $\mathbb Q$. 

On the other hand, $[\mathbb Q(e^{\frac{2\pi i}{5}}):\mathbb Q]=4$, as it is the root to $x^4+x^3+x^2+x+1$. Since $\gcd(4,5)=1$, we have $$[\mathbb \mathbb Q(\sqrt[5]{2},e^{\frac{2\pi i}{5}}):\mathbb Q]=20.$$
\fi
    We have $X^5-2 = \prod_{i=0}^4 (X - 2^{1/5}e^{\tfrac{2\pi}{5}i}) \in \mathbb{C}[X]$ which means $\mathbb{Q}[2^{1/5},e^{2\pi i/5}]$ will be a splitting field since $e^{2^{1/5},2\pi i/5}$ are algebraic over $\mathbb{Q}$, whose degree is less than $25$. Also notice that $[\mathbb {Q}(e^{\frac{2\pi i}{5}}):\mathbb {Q}]=4$, since it is the root of $x^4+x^3+x^2+x+1$ which is irreducible and hence $[\mathbb {Q}(e^{\frac{2\pi i}{5}}):\mathbb {Q}]\cong \mathbb{Q}/(x^4+x^3+x^2+x+1)$. Then we know that $[\mathbb{Q}[2^{1/5},e^{2\pi i/5}],\mathbb{Q}]$ has to be a multiple of $20$ and hence it is $20$, we are done.
\par 
\vspace{0.5em}

\subsection*{Ex.2(2.5 on JSM)}
Let $f\in F[X]$, where $F$ is a field of characteristic $0$. Let $d(X) = gcd(f,f')$. Show that $g(X):=f(X)d(X)^{-1}$ has the same roots as $f(X)$, and these are all simple roots of $g(X)$.
\vspace{0.5em}\\
\textbf{Sol.} \par
\iffalse
Since char$(F)=0$, $F$ is perfect.  Clearly, all roots of $g(x)$ are roots of $f(x)$. 

Notice that in the splitting field of $f=\prod_{i=1}^k(x-x_i)^{n_i}$, we have $f'=\sum_{i=1}^k n_i\frac{f(x)}{x-x_i}$, so $\gcd(f,f')=\prod_{n_i>1}(x-x_i)^{n_i-1}$. By comparing the roots, we see every root of $f(x)$ will be a root of $g(x)$, and every root of $g(x)$ has single multiplicity because its degree is 1.
\fi
It is trivial when $f$ is constant. Consider $f(X) = \prod_{i=1}^k (X-\alpha_i)\prod_{j=1}^m (X-\beta_j)^{m_j}$ with $\alpha_i,\beta_j$ distinct in some splitting field $E$ of it where $m_j \geq 2$ and then we will know that $\prod_{j=1}^m (X-\beta_j)^{m_j-1}| f,f$ in $E[X]$ and hence $g(X)$ will only have simple roots since $gcd(f,f')$ does not change with the extension. Then it is easy to check $\alpha_i$ is not a root of $f'$ and hence a root of $g$ and it is easy to check $(X-\beta_j)^{m_j}$ do not divide $f'$, since $F$ has $0$ characteristic and hence $(X-\beta_j)|g$ and hence $\beta_j$ is a root of $g$. Since any root of $g$ will be a root of $f$ and hence we know $g$ have the same roots with $f$ with all simple and we are done.
\par 
\vspace{0.5em}

\subsection*{Ex.3(2.6 on JSM)}
Let $f(X)$ be an irreducible polynomial in $F[X]$, where $F$ has characteristic $p$. Show that $f(X)$ can be written $f(X) = g(X^{p^e})$ where $g(X)$ is irreducible and separable. Deduce that every root of $f(X)$ has the same multiplicity $p^e$ in any splitting field.
\vspace{0.5em}\\
\textbf{Sol.} \par
\iffalse
    If $f(x)$ is irreducible, then $\gcd(f,f')=1$ unless $f'=0$.
\begin{enumerate}[\text{Case }1.]
\item $\gcd(f,f')=1$. Then, $f$ is separable and irreducible, hence $g(x^{p^e})=f(x)$ for $g=f$ and $e=0$.
\item $f'=0$. Then, $f(x)=g_0(x^p)$ for some $g_0$. Clearly, $g_0$ is irreducible. We show $g_0$ is separable. If not, $\gcd(g_0,g_0')\ne 1$ unless $g_0'=0$. If $g_0'\ne 0$, then there is $d$ such that $d\ |\ g_0$ and $d\ |\ g_0'$, contradicts with the irreducibility of $g_0$. Otherwise, $g_0'=0$, hence $g_0(x)=g_1(x^p)$ for some $g_1$. Then, $$f(x)=g_0(x^p)=g_1((x^p)^p)=g_1(x^{p^2}).$$
Similarly, $g_1$ is irreducible. Since $\deg g_0<\deg f$, in finite steps, we must have $f(x)=g_0(x^p)=g_m(x^{p^e})$ for some $e$. Then, $g_m$ is clearly irreducible. It is separable, since $g_m'\ne 0$. 

Then, we know $g(x)=\prod_{i=1}^{n}(x-x_i)$ in the splitting field of $g$ for some distinct roots $x_i$, where $n=\deg g$. For every $i$, let $\alpha_i$ be a root of $x^{p^e}-x_i$, then as a consequence of char$(F)=p$, we have $$(x-\alpha_i)^{p^e}=x^{p^e}-\alpha_i^{p^e}=x^{p^e}-x_i.$$ So,$$f(x)=\prod_i(x-\alpha_i)^{p^e}.$$
This proves the claim.
\end{enumerate}
\fi
    If $f$ is separable, then it is obvious that $e=0$. Assume $f$ has multiple root, then we will know $f(X) = g(X^p)$ for some $g\in F[X]$ which has to be irreducible since if not, we will know $f(X) = p(X^p)q(X^p)$ for some $p,q\in F[X]$ which is a contradiction. We know $\deg g = \deg f/p$, so we continue this process and then we will some $h$ such that $f(X) = h(X^{p^e})$ for some integer $e$ and $\deg h<p$, then $h$ can not have multiple roots or $\deg h \geq p$ and then we know $h$ is irreducible and separable and we are done.
\par 
\vspace{0.5em}

\subsection*{Ex.4(Problem A)}
Let $E\supset F$ be an algebraic extension. Choose an algebraic closure $\overline{F}$ of $F$ such that $E \subset \overline{F}$. Prove that the following conditions are equivalent:\par
\begin{enumerate}
    \item $E$ is the splitting field of a family of polynomials $S \subset F[x] - \{0\}$;
    \item For any $F$-homomorphism $\phi : E \to \overline{F} , \phi(E) \subset E$;
    \item The minimal polynomial of every $a \in E$ over $F$ splits over $E$ 
\end{enumerate}
\vspace{0.5em}
\textbf{Sol.} \par
\iffalse
1 $\Rightarrow$ 2: If $E$ is the splitting field of a family of polynomials $S\subset F[x]-\{0\}$, then any $\alpha\in E$ is a linear combination of roots of polynomials in $S\subset F[x]-\{0\}$. Therefore, any $F$-homomorphism fixing $F$ doesn't change $S$, which sends roots of polynomials to some roots of the same polynomials. Since $E$ is the splitting field, all these roots are in $E$, hence $\phi(E)\subset E$.

2 $\Rightarrow$ 3: For $a\in E$, the minimal polynomial over $F$, denoted by $m_{a, F}$ is irreducible. Since $a\in E$, we may regard $E$ as an extension of $F(a)$. Since $\phi$ sends $E$ to $E$, we know the conjugates of $a$, i.e. other roots of $m_{a,F}$, are in $E$ as well. Therefore, $m_{a,F}$ splits over $E$.

3 $\Rightarrow$ 1: Since for any $a\in E$, $m_{\alpha, F}$ splits over $F$, we may regard $E$ as the splitting field of $m_{a, F}$ for all $a\in E$.

We are done.
\fi
(1) implies (2): Assume $E$ is the splitting field of a polynomial $f\in F[X]-\{0\}$ with roots $\{\alpha_i\}_{i=1}^k$ and then we know $0 = \phi(f(\alpha_i)) = f(\phi(\alpha_i))$ and hence $\phi(\alpha_i) = \alpha_j$ for some $j$ for any $F$-homomorphism. Therefore any element in $E$ will be mapped to an element in $E$. Then since any element $E$ is contained in some splitting field contained in $E$ for general $E$, so we are done.\par
(2) implies (3): Assume $a\in E$ and $f$ the minimal polynomial and $f(X) = \prod_{i=1}^m (X-\alpha_i) \in \overline{F}[X]$ where $\alpha_1 = a$. We know there exists $\phi_i:F[a] \to\overline{F}$ $F$-homomorphisms with $\phi_i(a) = \alpha_i$ and we know there exists $\varphi_i:E\to \overline{F}$ such that $\varphi_i|_{F[a]} = \phi_i$ and hence we know any $\alpha_i \in E$ and we are done.\par
(3) to (1) is obvious and we are done.
\par 
\vspace{0.5em}

\subsection*{Ex.5(Problem B)}\vspace{0.5em}
Let $F$ be a field of characteristic $p$. Prove that if $ a\in  F - F^p$, then
$X^p - a$ is irreducible.
\vspace{0.5em}\\
\textbf{Sol.} \par
   If $\alpha$ is a root of $X^p - a$ in some extension of $F$ and then $\alpha^p = a$ and we know $(X-\alpha)^p = X^p -\alpha^p = X^p - a$ and hence all roots of $X^p - a$ should be $\alpha$. If it is reducible, then we know there exist $1\leq k<p$ integer such that $(X-\alpha)^k \in F[X]$. Notice that if $f,g\in F[X], g|f$ then $f/g \in F[X]$ and hence there exists $X-\alpha \in F[X]$ since there are integers $m,n$ such that $km+(p-k)n = 1$ and then $X-\alpha = (X-\alpha)^{km+(p-k)n}\in F[X]$. Therefore, $\alpha \in F$ which is a contradiction.
\par 
\vspace{0.5em}

\subsection*{Ex.6(Problem C)}
Let $F$ be a perfect field. Prove that all finite extensions $E \supset F$ are perfect as well.
\vspace{0.5em}\\
\textbf{Sol.} \par
\iffalse
If char$(F)=0$, then $E/F$ is perfect automatically. Now assume char$(F)=p$. 

We prove that $E$ is perfect if and only if for any $\alpha$, we have $\sqrt[p]{\alpha}\in F$. For the ``$\Leftarrow$'', consider an irreducible polynomial over $F$, say $p(x)$. Assume $p$ is not separable, then we know that $p(x)=a_nx^{np}+a_{n-1}x^{(n-1)p}+\dots+a_1x^p+a_0$. For each $i$, find $b_i\in F$ such that $b_i^p=a_i$, then $f(x)=(b_nx^n+\dots+b_1x+b_0)^p$, then $f$ cannot be irreducible. 
Conversely, if $E$ is perfect, and $f(x)=x^p-a$ has no root in $E$, then it is not separable, but by the last problem, it is irreducible, so it is not perfect.

Since $F$ is perfect, $F^p=F$. Now, consider $F(a)$ for $a\in E$. We find $m_{a, F}$, the minimal polynomial of $a$ over $F$, then it is irreducible. Since $F$ is perfect, $m_{\alpha, F}$ is separable. \par
If $F$ is perfect, then every algebraic extension is separable. For any $\alpha\in E$, we have $m_{\alpha, E}\ |\ m_{\alpha, F}$, and since $m_{\alpha, F}$ is separable, so is $m_{\alpha, E}$. So, $\alpha$ is separable over $E$.
\fi
Assume $F$ has zero characteristic. Then $E/F$ obviously have zero characteristic. Then we assume $F$ has characteristic of $p$ and $E$ has a basis $\{e_i\}_{i=1}^n$. If there are $a_i \in F[X]$ such that $a_1e_1^p+a_2e_2^p+\cdots+a_ne_n^p = 0$, then assume $\beta_i^p = a_i$ and we have $(\beta_1e_1+\beta_2e_2+\cdots+\beta_ne_n)^p = 0$ which means $\beta_1e_1+\beta_2e_2+\cdots+\beta_ne_n = 0$ and hence a contradiction, so $e_1^p,\cdots,e_n^p$ is also a basis and hence $E$ is perfect by Frobenius homomorphism. 
\par 
\vspace{0.5em}


\addappheadtotoc

\end{document}
