%!TEX program = xelatex
\documentclass[lang=en,11pt,a4paper,citestyle =authoryear]{elegantpaper}

% 标题
\title{Homework09 - MATH 742}
\author{Boren(Wells) Guan}

% 本文档命令
\usepackage{array,url,stix}
\usepackage{subfigure}
\usepackage{tikz}
\usepackage{tikz-cd}
\newcommand{\ccr}[1]{\makecell{{\color{#1}\rule{1cm}{1cm}}}}
\newcommand{\code}[1]{\lstinline{#1}}
\newcommand{\prvd}{$\hfill \qedsymbol$}
\newcommand{\Z}{\mathbb{Z}}
\newcommand{\R}{\mathbb{R}}
\newcommand{\N}{\mathbb{N}}
\newcommand{\C}{\mathbb{C}}
\newcommand{\Q}{\mathbb{Q}}
\newcommand{\M}{\mathcal{M}}
\newcommand{\B}{\mathcal{B}}
\newcommand{\X}{\mathcal{X}}
\newcommand{\Hil}{\mathcal{H}}
\newcommand{\range}{\mathcal{R}}
\newcommand{\nul}{\mathcal{N}}

% 文档区
\begin{document}

% 标题
\maketitle

\subsection*{Before Reading:}\par
To make the proof more readable, I will miss or gap some natural or not important facts or notations during my writing. If you feel it hard to see, you can refer the appendix after the proof, where I will try to explain some simple conclusions (will be marked) more clearly. In case that you misunderstand the mark, I will add the mark just after those formulas between \$ and before those between \$\$.\par
And I have to claim that the appendix is of course a part of my assignment, so the reference of it is required. Enjoy your grading!

\subsection*{Ex.1(2.3 on JSM)}
\vspace{0.5em}
\textbf{Sol.} \par
    It is $\mathbb Q(\sqrt[5]{2},e^{\frac{2\pi i}{5}})$. Clearly $\sqrt[5]{2},\sqrt[5]{2}e^{\frac{2\pi i}{5}}$ are in the splitting field, so $e^{\frac{2\pi i}{5}}$ is in the splitting field. However, $\sqrt[5]{2}(e^{\frac{2\pi i}{5}})^j$ for $j=0,1,2,3,4$ are all the roots. 

Clearly, $[\mathbb Q(\sqrt[5]{2}):\mathbb Q]=5$ since $x^5-2$ is irreducible over $\mathbb Q$. 

On the other hand, $[\mathbb Q(e^{\frac{2\pi i}{5}}):\mathbb Q]=4$, as it is the root to $x^4+x^3+x^2+x+1$. Since $\gcd(4,5)=1$, we have $$[\mathbb \mathbb Q(\sqrt[5]{2},e^{\frac{2\pi i}{5}}):\mathbb Q]=20.$$
\par 
\vspace{0.5em}

\subsection*{Ex.2(2.5 on JSM)}
\textbf{Sol.} \par
Since char$(F)=0$, $F$ is perfect.  Clearly, all roots of $g(x)$ are roots of $f(x)$. 

Notice that in the splitting field of $f=\prod_{i=1}^k(x-x_i)^{n_i}$, we have $f'=\sum_{i=1}^k n_i\frac{f(x)}{x-x_i}$, so $\gcd(f,f')=\prod_{n_i>1}(x-x_i)^{n_i-1}$. By comparing the roots, we see every root of $f(x)$ will be a root of $g(x)$, and every root of $g(x)$ has single multiplicity because its degree is 1.
\par 
\vspace{0.5em}

\subsection*{Ex.3(2.6 on JSM)}
\textbf{Sol.} \par
    If $f(x)$ is irreducible, then $\gcd(f,f')=1$ unless $f'=0$.
\begin{enumerate}[\text{Case }1.]
\item $\gcd(f,f')=1$. Then, $f$ is separable and irreducible, hence $g(x^{p^e})=f(x)$ for $g=f$ and $e=0$.
\item $f'=0$. Then, $f(x)=g_0(x^p)$ for some $g_0$. Clearly, $g_0$ is irreducible. We show $g_0$ is separable. If not, $\gcd(g_0,g_0')\ne 1$ unless $g_0'=0$. If $g_0'\ne 0$, then there is $d$ such that $d\ |\ g_0$ and $d\ |\ g_0'$, contradicts with the irreducibility of $g_0$. Otherwise, $g_0'=0$, hence $g_0(x)=g_1(x^p)$ for some $g_1$. Then, $$f(x)=g_0(x^p)=g_1((x^p)^p)=g_1(x^{p^2}).$$
Similarly, $g_1$ is irreducible. Since $\deg g_0<\deg f$, in finite steps, we must have $f(x)=g_0(x^p)=g_m(x^{p^e})$ for some $e$. Then, $g_m$ is clearly irreducible. It is separable, since $g_m'\ne 0$. 

Then, we know $g(x)=\prod_{i=1}^{n}(x-x_i)$ in the splitting field of $g$ for some distinct roots $x_i$, where $n=\deg g$. For every $i$, let $\alpha_i$ be a root of $x^{p^e}-x_i$, then as a consequence of char$(F)=p$, we have $$(x-\alpha_i)^{p^e}=x^{p^e}-\alpha_i^{p^e}=x^{p^e}-x_i.$$ So,$$f(x)=\prod_i(x-\alpha_i)^{p^e}.$$
This proves the claim.
\end{enumerate}
\par 
\vspace{0.5em}

\subsection*{Ex.4(Problem A)}
\textbf{Sol.} \par
1 $\Rightarrow$ 2: If $E$ is the splitting field of a family of polynomials $S\subset F[x]-\{0\}$, then any $\alpha\in E$ is a linear combination of roots of polynomials in $S\subset F[x]-\{0\}$. Therefore, any $F$-homomorphism fixing $F$ doesn't change $S$, which sends roots of polynomials to some roots of the same polynomials. Since $E$ is the splitting field, all these roots are in $E$, hence $\phi(E)\subset E$.

2 $\Rightarrow$ 3: For $a\in E$, the minimal polynomial over $F$, denoted by $m_{a, F}$ is irreducible. Since $a\in E$, we may regard $E$ as an extension of $F(a)$. Since $\phi$ sends $E$ to $E$, we know the conjugates of $a$, i.e. other roots of $m_{a,F}$, are in $E$ as well. Therefore, $m_{a,F}$ splits over $E$.

3 $\Rightarrow$ 1: Since for any $a\in E$, $m_{\alpha, F}$ splits over $F$, we may regard $E$ as the splitting field of $m_{a, F}$ for all $a\in E$.

We are done.
\par 
\vspace{0.5em}

\subsection*{Ex.5(Problem B)}
\textbf{Sol.} \par
   Denote $f(x)=x^p-a$. So we need to prove that if $f$ is reducible, then $x^p-a$ has a root in $F$.

If $f$ reducible, then there is $g\ |\ f$ and $\deg g<\deg f$. In the splitting field of $F$, we may write $f=\prod_{i=1}^p (x-x_i)$. We also know $\prod_{i=1}^p(-x_i)=-a.$ 

On the other hand, $g=\prod_{j=1}^k (x-x_j)$ divides $f$. We also have $g(0)=\prod_{j=1}^k(-x_j)=b$. We also know that $b^p=\prod_{j=1}^k(-x_j)^p=(-1)^p\prod_{j=1}^k(x_j)^p=(-1)^pa^k,$ i.e. $a^k=(-b)^p$. 

We also know that $k=\deg g<\deg f=p$, hence $\gcd(k,p)=1$, so there are $m,n$ such that $km+pn=1$. We have $a=a^{km+pn}=a^{np}(-b)^{mp}=((-b)^ma^n)^p$, so $a\in F^p$. Contradiction.
\par 
\vspace{0.5em}

\subsection*{Ex.6(Problem C)}
\textbf{Sol.} \par
If char$(F)=0$, then $E/F$ is perfect automatically. Now assume char$(F)=p$. 

We prove that $E$ is perfect if and only if for any $\alpha$, we have $\sqrt[p]{\alpha}\in F$. For the ``$\Leftarrow$'', consider an irreducible polynomial over $F$, say $p(x)$. Assume $p$ is not separable, then we know that $p(x)=a_nx^{np}+a_{n-1}x^{(n-1)p}+\dots+a_1x^p+a_0$. For each $i$, find $b_i\in F$ such that $b_i^p=a_i$, then $f(x)=(b_nx^n+\dots+b_1x+b_0)^p$, then $f$ cannot be irreducible. 
Conversely, if $E$ is perfect, and $f(x)=x^p-a$ has no root in $E$, then it is not separable, but by the last problem, it is irreducible, so it is not perfect.

Since $F$ is perfect, $F^p=F$. Now, consider $F(a)$ for $a\in E$. We find $m_{a, F}$, the minimal polynomial of $a$ over $F$, then it is irreducible. Since $F$ is perfect, $m_{\alpha, F}$ is separable. \par
If $F$ is perfect, then every algebraic extension is separable. For any $\alpha\in E$, we have $m_{\alpha, E}\ |\ m_{\alpha, F}$, and since $m_{\alpha, F}$ is separable, so is $m_{\alpha, E}$. So, $\alpha$ is separable over $E$.
\par 
\vspace{0.5em}


\addappheadtotoc

\end{document}
