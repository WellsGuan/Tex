%!TEX program = xelatex
\documentclass[lang=en,11pt,a4paper,citestyle =authoryear]{elegantpaper}

% 标题
\title{Homework03 - MATH 742}
\author{Boren(Wells) Guan}

% 本文档命令
\usepackage{array,url,stix}
\usepackage{subfigure}
\usepackage{tikz}
\usepackage{tikz-cd}
\newcommand{\ccr}[1]{\makecell{{\color{#1}\rule{1cm}{1cm}}}}
\newcommand{\code}[1]{\lstinline{#1}}
\newcommand{\prvd}{$\hfill \qedsymbol$}
\newcommand{\Z}{\mathbb{Z}}
\newcommand{\R}{\mathbb{R}}
\newcommand{\N}{\mathbb{N}}
\newcommand{\C}{\mathbb{C}}
\newcommand{\Q}{\mathbb{Q}}
\newcommand{\M}{\mathcal{M}}
\newcommand{\B}{\mathcal{B}}
\newcommand{\X}{\mathcal{X}}
\newcommand{\Hil}{\mathcal{H}}
\newcommand{\range}{\mathcal{R}}
\newcommand{\nul}{\mathcal{N}}

% 文档区
\begin{document}

% 标题
\maketitle

\subsection*{Before Reading:}\par
To make the proof more readable, I will miss or gap some natural or not important facts or notations during my writing. If you feel it hard to see, you can refer the appendix after the proof, where I will try to explain some simple conclusions (will be marked) more clearly. In case that you misunderstand the mark, I will add the mark just after those formulas between \$ and before those between \$\$.\par
And I have to claim that the appendix is of course a part of my assignment, so the reference of it is required. Enjoy your grading!

\subsection*{Ex.1(3.44 on AK)} 
Let $R$ be a ring, $R'$ an algebra, $X$ a variable. Show $\text{nil}(R)R'\subset \text{nil}(R')$ and $\text{rad}(R[X]) = \text{nil}(R[X]) = \text{nil}(R)R[X]$
\vspace{0.5em}\\
\textbf{Sol.} \par
(1) We only need to check for any $a\in \text{nil}(R),b\in R'$, $ab\in \text{nil}(R')$, which is trivial since there is always an integer $n$ such that
\[
(ab)^n = a^nb^n = 0
\]
(where denote $\phi(a)$ as $a$ is more convenient).\par
(2) Notice we always have
\[
\text{nil}(R)R[X] \subset \text{nil}(R[X]) \subset \text{rad}(R[X])
\]
by (1). For $F\in\text{rad}(R[X])$, we know $G-FM$ is a unit for any unit $G$ and polynomial $M$, let $M = X$ and we may know $FX \in \text{nil}(R)(R[X])$ by Ex 3.42 on AK and hence $\text{rad}(R[X]) \subset \text{nil}(R)(R[X])$ and we are done.
\par 
\vspace{0.5em}

\subsection*{Ex.2(4.26 on AK)} 
Let $L,M,N$ be modules and there is a diagram
\[
\begin{tikzcd}
L\arrow[r,"\alpha"] & M\arrow[r,"\beta"]\arrow[l,"\rho"] & N\arrow[l,"\sigma"]
\end{tikzcd}
\]
Prove that
\[M = L\oplus N,\alpha= \iota_L,\beta = \pi_N,\sigma = \iota_N, \rho = \pi_L\]
iff
\[
\beta\alpha = 0, \beta\sigma = 1,\rho\sigma = 0, \rho\alpha = 1, \alpha\rho+\sigma\beta = 1
\]
\vspace{0.5em}\\
\textbf{Sol.} \par
It is quite easy to check the sufficiency, we only need to check the necessity.\par
Notice by the UMP of $L\oplus N$, there exists $\gamma,\delta$ such that the diagram
\[
\begin{tikzcd}
    L\arrow[rd,"\iota_L"]\arrow[r,"\alpha"] & M & N\arrow[ld,"\iota_N"]\arrow[l,"\sigma"] \\
     & L\oplus N\arrow[u,"\gamma"] & \\
\end{tikzcd}
\]
and
\[
\begin{tikzcd}
    L & M\arrow[l,"\rho"]\arrow[r,"\beta"]\arrow[d,"\delta"] & N \\
     & L\oplus N\arrow[lu,"\pi_L"]\arrow[ru,"\pi_N"] & \\
\end{tikzcd}
\]
commute and hence
\[
\begin{aligned}
    &\pi_N\delta\gamma\iota_L = \pi_L\delta\gamma\iota_N = 0,  \\
    &\pi_N\delta\gamma\iota_N = \pi_N\delta\gamma\iota_N = 1,
    \\
    &\gamma\iota_L\pi_L\delta + \gamma\iota_N\pi_N\delta = \gamma\delta = 1
\end{aligned}
\]
which means $\delta\gamma = 1, \gamma\delta = 1$ and we are done.
\par 
\vspace{0.5em}

\subsection*{Ex.3(4.27 on AK)} 
Let $L$ be a module, $\Lambda$ a nonempty set, $M_{\lambda}$ a module. Prove the injections $\iota_{\kappa}:M_{\kappa} \to \bigoplus M_{\lambda}$ induce an injection
\[
\bigoplus \text{Hom}(L,M_{\lambda}) \hookrightarrow \text{Hom}(L,\bigoplus{M_{\lambda}})
\]
and that it is an isomorphism if $L$ is finitely generated.
\vspace{0.5em}\\
\textbf{Sol.} \par
For $(\alpha_{\lambda})$, its image $\phi((\alpha_{\lambda}))$ is $\sum \iota_{\kappa}\alpha_{\kappa}$ and if is zero, then $\alpha_{\lambda} = 0$ for any $\lambda$ and hence $\ker\phi = 0$, which means $\phi$ is an injection.\par
If $L$ has generated $l_i, 1\leq i \leq n$, then $(\alpha_{\lambda})$ is determined by $(\alpha_{\lambda}l_i), 1\leq i \leq n$ uniquely and for any $\psi:L\to\oplus(M_{\lambda})$, let $(\alpha_{\lambda}(l_i)) = (\pi_{\lambda}\psi(l_i))$ and then the induced vector $(\beta_{\lambda})$ will have
\[
\sum\iota_{\kappa}\beta_{\lambda}(l_i) = \iota_{\kappa}\pi_{\lambda}\psi(l_i) = \psi(l_i)
\]
and hence the map between the homomorphism becomes a surjection.
\par 
\vspace{0.5em}

\subsection*{Ex.4(4.28 on AK)} 
Let $I$ be an ideal, $\Lambda$ a nonempty set, $M_{\lambda}$ a module for $\lambda \in \Lambda$. Prove $I(\bigoplus M_{\lambda}) = \bigoplus IM_{\lambda}$. Prove $I(\prod M_{\lambda}) = \prod I M_{\lambda}$ if $I$ is finitely generated.\par
\vspace{0.5em}
\textbf{Sol.} \par
Obviously $I(\bigoplus M_{\lambda}) \subset \bigoplus IM_{\lambda}$ since both the sets are submodules of $\bigoplus M_{\lambda}$. For $(m_{\lambda}) \in \bigoplus IM_{\lambda}$, assume $m_{\lambda_i}, 1\leq i \leq n$ are the only nonzero elements and since $m_{\lambda_i} \in IM_{\lambda_I}$, so $(0,\cdots,m_{\lambda_i},\cdots) \in I(\bigoplus M_{\lambda})$ and hence $I(\bigoplus M_{\lambda}) \supset \bigoplus IM_{\lambda}$ and we are done.\par
If $I$ has generators $a_i, 1\leq i\leq k$, then by similar proof of the first part of the above conclusion, we only need to check $I(\prod M_{\lambda}) \supset \prod I M_{\lambda}$, then assume $(m_{\lambda}) \in \prod I M_{\lambda}$, then $m_{\lambda} \in IM_{\lambda}$ and hence there exists $m_{\lambda}^{(i)}$ such that $m_{\lambda} = \sum a_i\sum r_{\lambda}^{(j)}m_{\lambda}^{(j)}$, $r_{\lambda}^{(j)} \in R$, and $a_i\sum r_{\lambda}^{(j)}m_{\lambda}^{(j)} \in I(\prod M_{\lambda})$ and notice the summation is finite and we are done. 
\par 
\vspace{0.5em}

\subsection*{Ex.5(A)} 
Let $R_1,R_2$ be two rings, and set $R = R_1\times R_2$. Prove that any $R$-module $M$ is of the form $M_1\times M_2 \cong M$ where $M_i$ is a $R_i$-module and $R$ acts on $M_1,M_2$ componentwise.
\vspace{0.5em}
\textbf{Sol.} \par
Consider $R_1,R_2$ as subring of $R$ canonically and let $M_i = \text{Ann}(R_i)$, then consider
\[
\phi:M_1\times M_2 \to M
\]
by $(m_1,m_2)\mapsto m_1+m_2$. If $m_1+m_2 = 0$, then $m_1 = -m_2 \in M_1\cap M_2$ and hence $1m_1 = 0$ since $R = R_1+R_2$, which means $\phi$ is injective. For any $m\in M$, let $m_1 = m - (1,0)m, m_2 = m-(0,1)m$ and then $m_1\in M_1,m_2\in M_2$ and $m_1+m_2 = 2m - (1,1) m = m$ and we have shown $\phi$ is surjective and we are done.
\par 
\vspace{0.5em}

\subsection*{Ex.6(B)} 
    Define a natural morphism $R[[t]] \to R$ and show the restriction along this morphsim gives a bijection between the maximal ideals of $R$ and $R[[t]]$.
\vspace{0.5em}\\
\textbf{Sol.} \par
    For an ideal $I\subset R[[t]]$, we define $\phi(I):=\{a,\text{ there is }F\in I\text{ such that }a = F(0)\}$ and then $\phi(I)$ is an ideal, since it is trivial that $\phi(I)$ is closed under addition and multiplication since if $a,b\in \phi(I)$, then there exist $F,G \in I$ such that $F(0) = a, G(0) =b$ and $ab = FG(0)$.\par
    For a maximal ideal $M$ of $R[[t]]$ and we show $\phi(I)$ still a maximal ideal. If there exists $J \supset \phi(I)$, then there exists $b\in R$ such that $b\in J - \phi(I)$ and then we will know $bR[[t]] + I \supset I$ and hence it is $R[[t]]$, and there exists $F\in R[[t]], G\in I$ such that $F+G$ is a unit, with a unit constant term and hence $bF(0)+G(0)$ is a unit in $R$ and hence there is a unit in $\phi(I)$, which is a contradiction.\par
    Now we only need to check $\phi$ is injective and surjective. For a maximal ideal $N$ in $R$, we consider $\phi^{-1}(R)$ which means all the series with constant term in $N$, which is easy to be checked an ideal and if there is a larger ideal $J$, consider $\phi(J)$ will be an ideal in $R$ and it has to be $I$, so $J\subset N$ and hence $\phi$ is a surjection. Then we may know any maximal ideal $M$ in $R[[t]]$ it has to be $\phi^{-1}\phi(M)$ by the maximal property, and hence $\phi$ is injective and we are done.
\vspace{0.5em}

\subsection*{Ex.7(C)} 
Let $R$ be a ring and $R'$ be an $R$-algebra. It is easy to see that any $R'$-module $M$ becomes an $R$-module via 'restriction of scalars': the product map $R\times M$ is the composition
\[R\times M \to R'\times M \to M\]
Let $N$ be an $R$-module and consider all possible ways to turn it into an $R'$-module. Show that the set of all such $R'$-module structures is identified with the set of homomorphisms of $R$-algeras
\[R' \to \text{End}_R(N)\]
\vspace{0.5em}\\
\textbf{Sol.} \par
    Define
    \[
    \phi:R' \to \text{End}_R(N)
    \]
    by
    \[
    \phi(r') = \mu_{r'} \in \text{End}_R(N)
    \]
    where $\mu_{r'}$ is the scalar multiplication with $r'$, and we only need to check the diagram
    \[
    \begin{tikzcd}
        R'\arrow[r] & \text{End}_R(N) \\
        R\arrow[hookrightarrow]{u}\arrow[hookrightarrow]{ru} & \\
    \end{tikzcd}
    \]
    commutes, where we know
    \[
    \mu_r = \mu_{\phi(r)}
    \]
    where $\phi$ is the structure map of $R'$, since the $R'$-module structure is compatiable with the $R$-module structure of $N$ and we are done.
\vspace{0.5em}


\addappheadtotoc

\end{document}
