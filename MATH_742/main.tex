\documentclass{article}

\usepackage{amsmath, amsthm, amssymb, amsfonts}
\usepackage{thmtools}
\usepackage{graphicx}
\usepackage{setspace}
\usepackage{geometry}
\usepackage{float}
\usepackage{hyperref}
\usepackage[utf8]{inputenc}
\usepackage[english]{babel}
\usepackage{framed}
\usepackage[dvipsnames]{xcolor}
\usepackage{tcolorbox}
\usepackage{tikz}
\usepackage{tikz-cd}

\colorlet{LightGray}{White!90!Periwinkle}
\colorlet{LightOrange}{Orange!15}
\colorlet{LightGreen}{Green!15}

\newcommand{\HRule}[1]{\rule{\linewidth}{#1}}
\newcommand{\Pf}[1]{$Proof.$\par}

\declaretheoremstyle[name=Definiton,]{thmsty}
\declaretheorem[style=thmsty,numberwithin=subsection]{definition}

\declaretheoremstyle[name=Theorem,]{thmsty}
\declaretheorem[style=thmsty,numberwithin=subsection]{theorem}


\declaretheoremstyle[name=Lemma,]{thmsty}
\declaretheorem[style=thmsty,numberlike=theorem]{lemma}

\declaretheoremstyle[name=Corollary,]{thmsty}
\declaretheorem[style=thmsty,numberlike=theorem]{corollary}

\declaretheoremstyle[name=Proposition,]{prosty}
\declaretheorem[style=prosty,numberlike=theorem]{proposition}

\declaretheoremstyle[name=Principle,]{prcpsty}
\declaretheorem[style=prcpsty,numberlike=theorem]{principle}

\declaretheoremstyle[name=Example,]{prcpsty}
\declaretheorem[style=prcpsty,numberwithin=subsection]{example}

\declaretheoremstyle[name=Ex,]{prcpsty}
\declaretheorem[style=prcpsty,numberwithin=section]{exercise}


\setstretch{1.2}
\geometry{
    textheight=9in,
    textwidth=5.5in,
    top=1in,
    headheight=12pt,
    headsep=25pt,
    footskip=30pt
}

% ------------------------------------------------------------------------------

\begin{document}

% ------------------------------------------------------------------------------
% Cover Page and ToC
% ------------------------------------------------------------------------------

\title{ \normalsize \textsc{}
		\\ [2.0cm]
		\HRule{1.5pt} \\
		\LARGE \textbf{\uppercase{Notes for Abstract Algebra}
		\HRule{2.0pt} \\ [0.6cm] \LARGE{Based on lectures provided by Dima Arinkin on MATH 742 2025 SPRING} \vspace*{10\baselineskip}}
		}
\date{}
\author{\textbf{Author} \\ 
		Wells Guan \\
		 \\
		}

\maketitle
\newpage

\tableofcontents
\newpage

% ------------------------------------------------------------------------------

\section{Rings and Ideals}

\subsection{Rings}

\begin{definition}(Ring)\par
    A ring $R$ is an abelian group with an associative multiplication distributive over the addition. (We always assume a ring has a multiplicative identity and commutative if not marked)\par
    A unit is an element $u$ with a reciprocal $1/u$ such that $u\cdot 1/u = 1$, which is also denoted $u^{-1}$ and called a numtiplicative inverse and the units form a multiplicative group, denoted $R^{\times}$.
\end{definition}

\begin{definition}
    (Homomorphism)\par
    A ring homomorphism is a ring map $\phi: R \to R'$ which preserving sums, products and $1$. If $R' = R$ we call $\phi$ an endomorphism and if it is also bijective we call it an automorphism.
\end{definition}

\begin{definition}
    (Subring)\par
    A subset $R'' \subset R$ is a buting if $R''$ is a ring and the inclusion $R'' \hookleftarrow R$ is a ring map. We call $R$ a extension of $R''$ and the inclusion an extension.\par
\end{definition}

\begin{definition}
    (Algebra)\par
    An $R$-algebra is a ring $R'$ that comes equipped with a ring homomorphism $\phi:R\to R'$ called the structure map. An $R$-algebta homormorphism $R'\to R''$ is a ring homomorphism between $R$-algebtas compatible with structure maps.\par
\end{definition}

\begin{definition}
    (Group action)\par
    A group $G $ is said to act on $R$ if there is a homomorphism given from $G$ into the group of automorphisms of $R$. The ring of invariants $R^G$ is the subring defined by
    \[R^G := \{x\in R|gx = g\text{ for all }g\in G\}\]
\end{definition}

\begin{definition}
    (Boolean)\par
    A ring $B$ is called Boolean if $f^2 = f$ for all $f\in B$, then $2f = 0$ since
    \[
    2f = (f+f)^2 = 4f
    \]
\end{definition}

\begin{definition}
    (Polynomial rings)\par
    Let $R$ be a ring, $P:=R[X_1,\cdots,X_n]$ the polynomial ring in $n$ variables. $P$ has the Universal Mapping Property (UMP), i.e. given a ring homomorphism $\phi:R\to R'$ and given an element $x_i$ of $R'$ for each $i$, there is a unique ring map $\pi:P\to R'$ with $\pi|_{R} = \phi$ and $\pi(X_i) = x_i$.\par
    Similarly, let $X:= \{X_{\lambda}\}_{\lambda \in \Lambda}$ be any set of variables. Set $P':=R[X]$ the elements of $P'$ are the polynomials in any finitely many of $X$.
\end{definition}

\begin{definition}
    (Ideals)\par
    Let $R$ be a ring. An ideal $I$ is a subset containing $0$ of $R$ such that $xa\in I$ for any $x\in R, a\in I$ and closed under addition.\par
    For a subset $S\subset R$, $\langle S\rangle$ means the smallest ideal containing $S$.\par
    Given a single element $a$, we say that the ideal $\langle a \rangle$ is principal. For a number of ideals $I_\lambda$, the sum $\sum I_{\lambda}$ mean the set of all finite linear combinations $\sum x_{\lambda}a_{\lambda}$ for $x_{\lambda} \in R, a_{\lambda}\in I_{\lambda}$. If $\Lambda$ is finite, then the product $\prod I_{\lambda}$ means the ideal generated by all products $\prod a_{\lambda}, a_{\lambda}\in I_{\lambda}$.\par
    For two ideals $I$ and $J$, the transporter of $J$ into $I$ mean the set
    \[(I:J):=\{x\in R|xJ\subset I\}\]
    If $I\subset J$ a subsring such that $I\neq J$, then we call $I$ proper.\par
    For a ring homomorphism $\phi:R\to R'$, $I\subset R$ a subring, denote by $IR'$ or $I^e$ the ideal of $R'$ generated by $\phi(I)$ can we call it the extension of $I$.\par
    Given an ideal $J$ of $R'$ and its preimage $\phi^{-1}(J)$ is an ideal of $R$ and we call ti the contraction of $J$ denoted with $J^c$.
\end{definition}

\begin{definition}
    (Residue Rings)\par
    Let $I$ be an ideal of $R$ and the cosets of $I$
    \[R/I := \{x+I|x\in R\}\]
    have a ring structure and it will be called the residue ring or quotient ring or factor ring of $R$ modulo $I$ and the quotient map:
    \[
    \kappa: R \to R/I,\quad \kappa(x) = x+I
    \]
    and $\kappa x$ is called the residue of $x$.
\end{definition}

\begin{proposition}\ \par
    For $I\subset R$ a subring and a ring homomorphism from $R$ to $R'$, then $\ker(\phi) \supset I$ implies that is a ring homomorphism $\psi:R/I \to R'$ with $\psi\kappa = \phi$.\par
    $\psi$ is surjective iff $\phi$ is surjective. $\psi$ is injective iff $I = \ker(\phi)$.
\end{proposition}

\begin{corollary}
    $R/\ker(\phi) \cong Im(\phi)$
\end{corollary}

\begin{proposition}\ \par
    $R/I$ is universal among $R$-algebras $R'$ such that $IR' = 0$, i.e. for $\phi:R\to R'$ such that $\phi(I) = 0$, there is a unique ring homomorphism $\psi:R/I \to R'$ such that $\psi \kappa = \phi$.
\end{proposition}

\begin{definition}
    The UMP serves to determine $R/I$ up to unique isomorphism, i.e. if $R'$ equipped with $\phi:R\to R'$ has the UMP too, then $R'$ is isomorphic to $R/I$.
\end{definition}
\Pf\par
    If $R'$ has the UMP among the $R$-algebras $R''$ such that $IR'' = 0$, then $\phi(I) = 0$ and hence there is a unique $\psi:R/I \to R'$ such that $\psi\kappa = \phi$ and since $\kappa I = 0$, we know there exists unique $\psi'$ such that $\psi'\phi = \kappa$ and then $(\psi'\psi)\kappa = \kappa$ and hence $\psi'\psi = 1$ and we are done by the uniqueness.

\begin{proposition}
    Let $R$ be a ring, $P:=R[X]$ the polynomial ring in one variable, $a\in R$ and $\pi:P\to R$ the $R$-algebra mao define by $\pi(X):=a$, then
    \begin{itemize}
        \item $\ker\pi = \{F(X)\in P|F(a) = 0\} = \langle X-a\rangle$
        \item $P/\langle X-a\rangle \cong R$
    \end{itemize}
\end{proposition}

\begin{definition}
    (Order of a polynomial)\par
    Let $R$ be a ring, $P$ the polynomial ring in variables $X_{\lambda}$ for $\lambda \in \Lambda$ and $(x_{\lambda}) \in R^{\Lambda}$ a vector. Let $\phi_{(x_{\lambda})}P\to P$ denote the $R$-algebra homomorphism defined by $\phi_{(x_{\lambda})}X_{\mu} := X_{\mu} + x_{\mu}$.\par
    The order of $F$ at the vector $(x_{\lambda})$ is defined as the smallest degree of monomials $M$ in $(\phi_{(x_{\lambda})}F)$.\par
    We know $\text{ord}_{(x_{\lambda})}F = 0$ iff $F(x_{\lambda}) \neq 0$.
\end{definition}

\begin{definition}
    Let $R$ be a ring, $I$ an ideal and $\kappa$ the quotient map. Given an ideal $J\supset I$ then the cosets
    \[
    J/I:=\{b+I|b\in J\} = \kappa(J)
    \]
    and then  $J/I$ is an ideal of $R/I$ and also $J/I = J(R/I)$.
\end{definition}

\begin{proposition}
    Given $J\supset I$ and we know
    \[
    \phi:R\to R/I\to (R/I)/(J/I)
    \]
    then we have the commutative diagram:
    \[
    \begin{tikzcd}
        R\arrow[r]\arrow[d] & R/J\arrow[d,"\cong"] \\
        R/I\arrow[r] & (R/I)/(J/I) \\
    \end{tikzcd}
    \]
\end{proposition}
\Pf\par
    Since $\phi(J) = 0$, so there exists unique $\psi:R/J \to (R/I)/(J/I)$ such that $\psi\kappa_J =\phi$ and since $\kappa_J(I) = 0$ and there exists $p$ such that $p\kappa_I = \kappa_J$ and consider $p(J/I) = 0$ and there exists $h$ such that $h\kappa_{(J/I)} = p$ and it is easy to check $h\psi = 1$ by uniqueness and we are done.

\begin{definition}\par
    Let $R$ be a ring. Let $e\in R$ be an idempotent, i.e. $e^2 = e$ then $Re$ is a ring with $e$ as multiplication unit, but $Re$ is not a subring unless $e=1$.\par
    Let $e':= 1-e$, then $e'$ is idempotent and $ee' = 0$ and we call them complementary idempotents.\par
    Denote $\text{Idem}(R)$ the set of all idempotents, which is close under a ring homomorphism.
\end{definition}

\begin{proposition}
    If $e_1,e_2\in R$ such that $e_1+e_2 = 1$ and $e_1e_2 =0$, then they are complementary idempotents.
\end{proposition}

\begin{definition}
    Let $R:R'\times R''$ be a product of two rings with componentwise operations.
\end{definition}

\begin{proposition}
    Let $R$ be a ring and $e',e''$ complementary idempotents. Set $R' := Re'$ and $R'' = Re''$. Define $\phi:R\to R'\times R''$ by $\phi(x) = (xe',xe'')$ and then $\phi$ is a ring isomorphism. $R' = R/Re''$ and $R'' = R/Re'$.
\end{proposition}
\Pf\par
    Check $\phi$ is surjective and injective.\par
    There is a natrual isomorphism between $I = \{(0,xe'')\} \subset R'\times R''$ and $R''$, and consider the diagram
    \[
    \begin{tikzcd}
        R\arrow[r] \arrow[d]& R'\times R''\arrow[l]\arrow[d] \\
        R/R'' & R'\times R''/I \\
    \end{tikzcd}
    \]
    and use the UMP.

\subsection{Prime Ideals}

\begin{definition}
    (Zerodivisors)\par
    Let $R$ be a ring. An element $x$ iscalled a zerodivisor if there is a nonzero $y$ such that $xy = 0$; otherwise, $x$ is called a nonzerodivisor. Denote the set of zerodivisors by $\text{z.div}(R)$ and the nonzerodivisors by $S_0$.
\end{definition}

\begin{definition}
    (Multiplicative subsets, prime ideals)\par
    Let $R$ be a ring. A subset $S$ is called multipliccative if $1\in S$ and $x,y\in S$ implies $xy\in S$.\par
    An ideal $P$ is called prime if its complement $R-p$ is multiplicative, or equivalentely, if $1\neq P$ and $xy\in P$ implies $x\in P$ or $y\in P$.
\end{definition}

\begin{definition}
    (Fields,domains)\par
    A ring is called a field if $1\neq 0$ and if every nonzero element is a unit.\par
    A ring is called an integral domain, or a domain if $\langle 0 \rangle$ or equivalently, if $R$ is nonzero and has no nonzero zerodivisors.\par
    Every domain $R$ is a subring of its fraction field $\text{Frac}(R):=\{x/y, x,y\in R\text{ and }y\neq 0\}$.
\end{definition}

\begin{proposition}
    Any subring $R$ of a field $K$ is a domain, and for a domain $R$, $\text{Frac}(R)$ has the UMP: the inclusion of $R$ into any field $L$ extends uniquely to an inclusion of $\text{Frac}(R)$ into $L$. 
\end{proposition}
\Pf\par
    For any subring $R$ of a field, $a,b\in R$, if $ab=0$, and $a$ nonzero, then $b = 0$ and we are done.\par
    If $\phi:R\hookrightarrow L$, then $\phi(x/y) = \phi(x)\phi(y)^{-1}$ is well-defined and obviously a ring homomorphism and we are done.

\begin{definition}
    (Polynomials over a domain)\par
    Let $R$ be a domain, $X$ a set of variable. $P:=R[X]$ and then $P$ is a domain, and $\text{Frac}(P)$ is called the rational functions.
\end{definition}

\begin{definition}
    (Unique factorization)\par
    Let $R$ be a domain, $p$ a nonzero nonunit. We call $p$ prime if $p|xy$ implies $p|x$ or $p|y$, which is equivalent with $\langle p\rangle$ is prime.\par
    For $x,y\in R$, we call $d\in R$ their gcd if $d|x$ and $d|y$ and if $c|x,c|y$ then $c|d$.\par
    $p$ is irreducible if $p=yz$ implies $y$ or $z$ is a unit. We call $R$ is a UFG if every nonzero nonunit factors into a product of irreducibles and the facrtotization is unique to order and units.
\end{definition}

\begin{proposition}
    If every nonzero nonunit factors have a factorization of a product of irreducible elements, then the factorization is unique up to order and units iff every irreducible element is prime.
\end{proposition}
\Pf\par
    
\begin{lemma}
    Let $\phi:R\to R'$ be a ring homomorphism, and $T\subset R'$ a subset. If $T$ is multiplicative, then $\phi^{-1}T$ is multiplicative; the converse holds if $\phi$ is surjective.
\end{lemma}
\Pf\par

\begin{proposition}
    Let $\phi:R\to R'$ be a ring map, and $J\subset R'$ an ideal. Set $I:=\phi^{-1}J$. If $J$ is prime, then $I$ is prime; the converse holds if $\phi$ is surjective.
\end{proposition}

\begin{corollary}
    Let $R$ be a ring, $I$ an ideal. Then $I$ is prime iff $R/I$ is a domain.
\end{corollary}
\Pf\par
    Consider
    \[
    \kappa:R\to R/I
    \]
    the quotient map and $I$ prime implies $\langle 0 \rangle$ is prime in $R/I$ and hence $R/I$ is a domain.

\begin{definition}
    (Maximal ideal)\par
    Let $R$ be a ring. An ideal $I$ is sai to be maximal if $I$ is proper and there is no proper ideal $J$ such that $I\subset J, I\neq J$.
\end{definition}

\begin{proposition}
    A ring $R$ is a field iff $\langle 0 \rangle$ is a maximal ideal.
\end{proposition}

\begin{corollary}
    Let $R$ be a ring, $I$ an ideal. Then $I$ is maximal iff $R/I$ is a field.
\end{corollary}
\Pf\par
    Only need to check $\langle 0 \rangle $ is maximal in $R/I$.

\begin{corollary}
    In a ring, every maximal ideal is prime.
\end{corollary}

\begin{definition}
    (Coprime)\par
    Let $R$ be a ring, and $x,y\in R$. We say $x$ and $y$ are coprime if their ideals $\langle x\rangle$ and $\langle y$ are comaximal.\par
    $x$ and $y$ are coprime if and only if there are $a,b\in R$ such that $ax+by = 1$.\par
\end{definition}

\begin{definition}
    A domain $R$ is called a Principal Ideal Domain if every ideal is principal. A PID is a UFD.
\end{definition}

\begin{theorem}
    Let $R$ be a PID. Let $P:=R[X]$ be the polynomial ring in one variable $X$, and $I$ a nonzero prime ideal of $P$. Then $P = \langle F\rangle$ with $F$ prime, or $P$ is maximal. Assume $P$ is maximal. Then either $P = \langle F\rangle $ with $F$ prime, or $P=\langle p, G\rangle$ with $p\in R$ prime, $pR = P\cap R$ and $G\in P$ prime with iamge $G'\in (R/pR)[X]$ prime.
\end{theorem}

\begin{theorem}
    Every proper ideal $I$ is contained in some maximal ideal.
\end{theorem}

\begin{corollary}
    Let $R$ be a ring, $x\in R$. Then $x$ is a unit iff $x$ belongs to non maximal ideal.
\end{corollary}

\subsection{Radicals}

\begin{definition}
    (Radical)\par
    Let $R$ be a ring. Its radical $\text{rad}(R)$ is defined to be the intersection of all its maximal ideals. 
\end{definition}

\begin{proposition}
    Let $R$ be a ring, $I$ an ideal, $x\in R$ and $u\in R^{\times}$. Then $x\in \text{rad}(R)$ iff $u-xy\in R^{\times}$ for all $y\in R$. In particular, the sum of an element of $\text{rad}(R)$ and a unit is a unit, and $I\subset \text{rad}(R)$ if $1-I\subset R^{\times}$.
\end{proposition}
\Pf\par
    For a maximal ideal $J$, if $u-xy\in J$, then $u\in J$ which is a contradiction and hence $u-xy$ is a unit. Conversely, if there exists $J$ maximal such that $x\in J$, then $\langle x\rangle + J = R$ and hence there exists $m\in J$ such that $u-xy = m$ for some unit $u$, which is a contradiction.\par

\begin{corollary}
    Let $R$ be a ring, $I$ an ideal, $\kappa:R\to R/I$ the quotient map. Assume $I\subset \text{rad}(R)$, then $\kappa$ is injective on $\text{Idem}(R)$.
\end{corollary}
\Pf\par
    For $e,e'\in \text{Idem}(R)$ and $x = e-e'$, if $\kappa(x) = 0$, then $x^3 = x$ and hence $x(1-x^2) = 0$, so $1-x^2$ is a unit and hence $x$ is $0$ and we are done.

\begin{definition}
    (Local ring)\par
    A ring is called local if it has exactly one maximal ideal, and semilocal if it has at least one and at most finitely many.\par
    By the residue field of a local ring $A$, we mean the field $A/M$ where $M$ is the maximal ideal of $A$.
\end{definition}

\begin{lemma}
    Let $A$ be a ring, $N$ the set of nonunits. Then $A$ is local iff $N$ is an ideal, if so, then $N$ is the maximal idal.
\end{lemma}
\Pf\par
    Only need to check the sufficiency, if $A$ is local, then we know $M$ is contained in $N$, and if there is $y\in M-N$, then $\langle y\rangle$ is a proper ideal and hence $y\in N$, which is a contradiction and hence $M=N$ and we are done.

\begin{proposition}
    Let $R$ be a ring, $S$ a multiplicative subset, and $I$ an ideal with $I\cap S = \emptyset$. Set $\mathcal{S} :=\{J,J\supset I, J\cap S = \emptyset\}$, then $\mathcal{S}$ has a maximal element $P$ and every such $P$ is prime.
\end{proposition}
\Pf\par
    By Zorn's lemma, their is a maximal element $P$ in $S$, for $x,y \in R - P$, there  exists $p,q\in P, a,b\in R$ such that $p+ax\in S, q+by \in S$ and hence $pq+pby+qax+abxy \in S$, and hence $xy\notin P$ and we are done.

\begin{definition}
    (Saturated multiplicative subsets)\par
    Let $R$ be a ring, and $S$ a multiplicative subset. We say $S$ is saturated if for $x,y\in R, xy\in S$, then $x,y\in S$.
\end{definition}

\begin{lemma}
    Let $R$ be a ring, $I$ a subset of $R$ that is stable under addition and multiplication, and $P_1,\cdots,P_n$ ideals such that $P_3,\cdots,P_n$ are prime. If $I$ is not contained in $P_j$ for all $j$, then there is an $x\in I$ such that $x\in P_j$ for $j$ or equivalently, if $I\subset \bigcup_{i=1}^n P_i$, then $I\subset I_i$ for some $i$.
\end{lemma}
\Pf\par
    If $n=1$ then we are done. We may use the induction, assume that $n\geq 2$, then by induction, for each $i$, there is $x_i\in I$ such that $x_i$ is not in $P_j, i\neq j$ and $x_i\in P_i$, so then $x_1+x_2\notin P_2$ if $n=2$. For other $n$, we will know $(x_1\cdots,x_{n-1})\notin P_j$ for all $j$.

\begin{definition}
    Let $R$ be a ring, $S$ a subset, its radical $\sqrt{S}$ is the set
    \[
    \sqrt{S}:=\{x\in R|x^n\in S\text{ for some }n\}
    \]
    If $I$ is an ideal and $I=\sqrt{I}$, then call $I$ to be radical.\par
    We call $\sqrt{0}$ is the nilradical and denoted as $\text{nil}(R)$. We call $x\in R$ nilpotent if $x\in \text{nil}(0)$, we call an ideal $I$ nilpotent if $a^n = 0$ for some $n\geq 1$.
\end{definition}

\begin{theorem}
    Let $R$ be a ring, $I$ an ideal, then
    \[
    \sqrt{I} = \cap_{P\supset I,P\text { prime}} P\]
\end{theorem}
\Pf\par
    For $x\notin\sqrt{I}$, let $S$ contains all the expotents of $x$ and $S$ is multiplicative, then $I\cap S = \emptyset$ and then there is an $P$ prime containing $I$ with not containing $x$ and hence $\sqrt{a}$ contains the union.\par
    Converse direction is easy.

\begin{proposition}
    Let $R$ be a ring, $I$ an ideal. Then $\sqrt{I}$ is an ideal.
\end{proposition}

\begin{definition}
    (Minimal primes)\par
    Let $R$ be a ring, $I$ an ideal and $P$ prime. We call $P$ a minimal prime of $I$ if $P$ is minimal in the set of primes containing $I$, we all $P$ a minimal prime of $R$ if $P$ is a minimal prime of $\langle 0 \rangle$.
\end{definition}

\begin{proposition}
A ring $R$ is reduced, i.e. $0$ is the only nilpotent, and has only one minial prime iff $R$ is a domain.
\end{proposition}
\Pf\par
    Converse direction is obvious. If $0$ is the only nilpotent elements, $Q$ is a minimal prime ideal, then $Q = 0$ since $0$ is the intersection of all the minimal primes, and we are done. 

\subsection{Modules}

\begin{definition}
    (Modules)\par
    Let $R$ be a ring. An $R$-module $M$ is an abelian group with a scalar multiplication
    $R\times M \to M$ which is
    \begin{itemize}
        \item $x(m+n) = xm+xn$ and $(x+y)m = xm + ym$
        \item $x(ym) = (xy)m$
        \item $1m = m$
    \end{itemize}\par
    A submodule $N$ of $M$ closed under scalar multiplication.\par
    Given $m\in M$, its annihilator
    \[
    \text{Ann}(m):=\{x\in R|xm = 0\}
    \]
    and the annilhilator of $M$ is
    \[
    \text{Ann}(M):=\{x\in R|xm = 0\text{ for all }m\in M\}
    \]
    We call the intersection of all maximal ideals containing $Ann(M)$ the radical of $M$, denoted as $\text{rad}(M)$.\par   
\end{definition}

\begin{proposition}
    There is a bijection between the maximal ideals containing $\text{Ann}(M)$ and the maximal ideals of $R/\text{Ann}(M)$, and hence
    \[
    \text{rad}(R/\text{Ann}(M)) = \text{rad}(M) /\text{Ann}(M)
    \]
\end{proposition}

\begin{proposition}
    Given a submodule $N$ of $M$, and then $\text{Ann}(M) \subset \text{Ann}(N)$ and we also have $\text{Ann}(M) \subset \text{Ann}(M/N)$.
\end{proposition}

\begin{definition}
    (Semilocal)\par
    We call $M$ semilocal if there are only finitely many maximal ideals containing $\text{Ann}(M)$. If $R$ is semilocal, so is $M$ and we will know $M$ is semilocal iff $R/\text{Ann}(M)$ is a semilocal ring.
\end{definition}

\begin{definition}
    (Polynomials)\par
    The sets of polynomials
    \[
    M[X] :=\{\sum\limits_{i=0}^n m_iM_i, M_i\text{ monomials}\}
    \]
    and then $M[X]$ is an $R[X]-module$.
\end{definition}

\begin{definition}
    (Homomorphisms)\par
    Let $R$ be aring, $M$ and $N$ modules. A $R$-linear map is a map $\alpha:M\to N$ such that
    \[
    \alpha(xm+yn) = x\alpha m + y\alpha n
    \]
    Let $\iota:\ker\alpha \to M$ be the inclusion and then $\ker\alpha$ has the UMP: $\alpha\iota = 0$ and for a homomorphism $\beta:K\to M$ with $\alpha\beta = 0$, there is a unique homomorphism $\gamma:K\to \ker\alpha$ with $\iota\gamma = \gamma$ as shown below
    \[
    \begin{tikzcd}
        \ker\alpha\arrow[r,"\iota"] & M\arrow[r,"\alpha"] & N \\
        &K\arrow[lu,"\gamma"]\arrow[u,"\beta"]\arrow[ru,"0"]& \\
    \end{tikzcd}
    \]
\end{definition}

\begin{definition}
    (Endomorphism)\par
    An endomorphism of $M$ a self-homomorphism denoted as $\text{End}_R(M) \subset \text{End}_{\mathbb{Z}}(M)$.\par
    For $x\in R$, let $\mu_x$ the self map of multiplication by $x$ and then $x\mapsto \mu_x$ denoted as
    \[\mu_R:R\to \text{End}_R(M)\]
    and note that $\ker\mu_R = \text{Ann}(M)$.
    We call $M$ faithful if $\mu_R$ is injective.
\end{definition}

\begin{definition}
    For two rings $R$ and $R'$, suppose $R'$ is an $R$-algebra and $M'$ an $R'$-module, then $M'$ is also an $R$-module by $xm:= \phi(x)m$.\par
    A subalgebra $R''$ of $R'$ is a subring such that the structure map owning image in $R''$. The subalgebra generated by $x_{\lambda} \in R'$ for $\lambda \in \Lambda$ is the smallest $R$-subalgebra containing $x_{\lambda}$ and we denote it by $R[\{x_{\lambda}\}]$ and we call $x_{\lambda}$ the generators.\par
    We say $R'$ is a finitely generated $R$-algebra if there exists $x_i, 1\leq i \leq n$ such that $R' = R[x_1,\cdots,x_n]$.
\end{definition}

\begin{definition}
    (Residue modules)\par
    Let $R$ be a ring, $M $a module and $M'\subset M$ a submodule. Then
    \[
    M/M':=\{m+M'|m\in M\}
    \]
    which is the residue module or M modulo M', form the quotien map
    \[\kappa:M\to M/M',\quad m\mapsto m+M'\]
\end{definition}

\begin{definition}
    (Cyclic Modules)\par
    Let $R$ be a ring. A module $M$ is said to be cyclic if there exists $m\in M$ such that $m = Rm$, then $\alpha:x\mapsto xm$ induces an isomorphism $R/\text{Ann}(m) \cong M$.
\end{definition}

\begin{definition}
    (Noether Isomorphisms)\par
    Let $R$ be a ring, $N$ a module, and $L$ and $M$ submodules.\par
    Assume $L\subset M$, and
    \[
    \alpha:N\to N/L \to (N/L)/(M/L)
    \]
    and we may know $\ker \alpha = M$. then $\alpha$ factors through the isomorphism $\beta$ in $N\to N/M \to (N/L)/(M/L)$ since $\alpha$ is surjective and $\ker\alpha = M$, so
    \[
    \begin{tikzcd}
        N\arrow[r]\arrow[d] & N/M\arrow[d,"\beta"] \\
        N/L\arrow[r] & (N/L)/(M/L) 
    \end{tikzcd}
    \]
    Assume $L$ not in $M$ and
    \[
    L+M:=\{l+m, l\in  L, m\in M\}
    \]
    and it will be a submodule, then similarly
    \[
    \begin{tikzcd}
        L\arrow[r]\arrow[d] & L/(L\cap M)\arrow[d,"\beta"] \\
        L+M\arrow[r] & (L+M)/M 
    \end{tikzcd}
    \]
 \end{definition}

\begin{definition}
    (Cokernels, coimages)\par
    Let $R$ be a ring, $\alpha:M\to N$ linear. Associated to $\alpha$ there are its cokernel and its coimage
    \[
    \text{Coker}(\alpha) := N/\text{Im}(\alpha)\quad \text{Coim}(\alpha):= M/\ker{\alpha}
    \]
\end{definition}

\begin{definition}
    (Generators, free modules)\par
    Let $R$ be a ring, $M$ a module. Given some submodules $N_{\lambda}$, by the sum $\sum N_{\lambda}$, we mean the set of all finite linear combinations $\sum x_{\lambda}m_{\lambda}, m_{\lambda} \in N_{\lambda}$.\par
    Elements $m_{\lambda}$ are said to be free of linearly independent if the linear combination equals to zero implies zero coefficients. If $m_{\lambda}$ are said to be form a (free) basis of $M$, then they are free and generate $M$ and we say $M$ is free on $m_{\lambda}$.\par
    We say $M$ is finitely generated if it has a finite set of generators and $M$ is free if it has a free basis.
\end{definition}

\begin{theorem}
    Let $R$ be a PID, $E$ a free module with $e_{\lambda}$ a basis, and $F$ a submodule, then $F$ is free and has a basis indexed by a subset of $\lambda$.
\end{theorem}

\begin{definition}
    Let $R$ be a ring, $\Lambda$ a set, $M_{\lambda}$ a module for $\lambda\in\Lambda$. The direct product of $M_{\lambda}$ is the set of any vectors
    \[
    \prod M_{\lambda} := \{(m_{m_{\lambda}})\}
    \] 
    which is a module under componentwise addition and scalar multiplication.\par
    The direct sum of $M_{\lambda}$ is the subset of restricted vectors:
    \[
    \bigoplus M_{\lambda} := \{(m_{m_{\lambda}}), m_{\lambda}\text{ nonzero for only finite elements}\}
    \]
\end{definition}

\begin{proposition}
    $\prod M_{\lambda}$ has the UMP,  for $R$-homomorphism $\alpha_{\kappa}:L\to M_{kappa}$, there is a unique $R$-homomorphism $L\to \prod M_{\lambda}$ such that $\pi_{\kappa}\alpha = \alpha_{\kappa}$, in other words, $\pi_{\lambda}$ induce a bijection of
    \[\text{Hom}(L,\prod M_{\lambda}) \cong \prod \text{Hom}(L,M_{\lambda})\]
    Similarly, the direct sum comes equipped with injections
    \[\iota_{\kappa} \to \bigoplus M_{\lambda}\]
    and it has the UMP: given $\beta_{\kappa}:M_{\kappa} \to N$, there is a unique $R$-homomorphism $\beta:\bigoplus M_{\lambda} \to N$ such that $\beta \iota_{\kappa} = \beta_{\kappa}$ and $\iota_{\kappa}$ iduce the bijection:
    \[\text{Hom}(\bigoplus, N) \to \prod \text{Hom}(M_{\lambda, N})\]
\end{proposition}

\subsection{Exact Sequences}

\begin{definition}
    (Exact)\par
    A sequence of module homomorphisms
    \[
    \cdots \to M_{k-1}\overset{\alpha_{k-1}}{\to} M_k \overset{\alpha_k}{\to} M_{k+1} \to \cdots
    \]
    is said to be exact at $M_k$ if $\ker\alpha_k = \text{Im}(\alpha_k)$. The sequence is said to be exact if it is exact at every $M_k$, except an initial source of final target.
\end{definition}

\begin{definition}
    (Short exact sequences)\par
    A sequence $0\to L \overset{\alpha}{\to} M \overset{\beta}{\to} N \to 0$ is exact if and only if $\alpha$ is injective and $N \cong \text{Coker}\alpha$ or dually if and only if $\beta$ is surjective and $L =\ker\beta$. Then the sequence is called short exact and we often regard $L$ as a submodule of $M$ and $N$ the quotient $M/L$.
\end{definition}
\Pf\par

\begin{proposition}
    For $\lambda \in \Lambda$, let $M_{\lambda}' \to M_{\lambda} \to M_{\lambda}''$ be sequence of module homomorphisms. If every sequence is exact, then so are the two induced sequences
    \[
    \bigoplus M_{\lambda}' \to \bigoplus M_{\lambda} \to \bigoplus M_{\lambda}'', \quad \prod M_{\lambda}' \to \prod M_{\lambda} \to \prod M_{\lambda}'' 
    \]
    Conversely, if either induced sequence is exact then so is every original one.
\end{proposition}
\Pf\par

\begin{proposition}
    Let $0\to M'\overset{\alpha}{\to} M \overset{\beta}{\to} M'' \to 0$ be a short exact sequence, and $N\subset M$ a submodule. Set $N':=\alpha^{-1}(N)$ and $N'':= \beta(N)$. Then the induced sequence $0 \to N' \to N \to N'' \to 0$ is short exact.
\end{proposition}

\begin{definition}
    (Retraction, section, splits)\par
    A linear map $\rho: M \to M'$ is a retraction of another $\alpha:M'\to M$ if $\rho\alpha = 1_{M'}$, then $\alpha$ is injective and $\rho$ is surjective.\par
    Dually, we call $\sigma:M''\to M$ a section of another $\beta:M\to M''$ if $\beta\sigma = 1_{M''}$, then $\beta$ is surjective and $\sigma$ is injective.\par
    We call a $3$-term exact sequence $M' \overset{\alpha}{\to} M \overset{\beta}{\to} M''$ splits if there is an isomorphism $\phi:M\cong M'\oplus M''$ with $\phi\alpha = \iota_{M'}$ and $\beta = \pi_{M''}\phi$.
\end{definition}
    
\begin{proposition}
    Let $M'\overset{\alpha}{\to}M\overset{\beta}{\to}M''$ be a $3$-term exact sequence. Then the following conditions are equivalent
    \begin{itemize}
        \item The sequence splits
        \item There exists a retraction $\rho:M\to M'$ of $\alpha$ and $\beta$ is surjective.
        \item There exists a section $\sigma:M'' \to M$ of $\beta$ and $\alpha$ is injective
    \end{itemize}
\end{proposition}
\Pf\par
    Assume the sequence is splits, then we have the commuting diagram
    \[
    \begin{tikzcd}
        M'\arrow[r,"\alpha"]\arrow[rd,"\iota_{M'}"] & M\arrow[r,"\beta"]\arrow[d,"\phi(\cong)"] & M'' \\
        & M'\oplus M''\arrow[ru,"\pi_{M''}"] & \\
    \end{tikzcd}
    \]
    then let $\rho = \pi_{M'}\phi$, then $\rho\alpha = \pi_{M'}\phi \phi^{-1}\iota_{M'} = 1_{M'}$. Let $\sigma = \phi^{-1}\iota_{M''}$ and then $\beta\sigma = \pi_{M''}\phi\phi^{-1}\iota_{M''} = 1_{M''}$ and then $\beta$ is surjective and $\alpha$ is injective.\par
    Now assume there is such a retraction $\rho$ and $\beta$ is surjective, then define $\sigma = 1_M - \alpha\rho$ and $\phi: M \to M'\oplus M''$ by $m\mapsto (\rho(m),\beta\sigma(m))$., if $\phi(m) = 0$, then $\rho(m) = 0$ and $\sigma(m) = m$, which means $\beta(m) = 0$. There exists $a\in M'$ such that $m = \alpha(a)$ and hence $a = 0$ which means $m = 0$, so $\ker\phi = 0$. For $(a,b)\in M'\oplus M''$, assume $\beta(m) = b$, then $\phi(\alpha(a) + \sigma(m)) = (a+\rho(m-\alpha\rho(m)), \beta(\alpha(a)) + \beta(\sigma(m))) = (a,b)$ and hence $\phi$ is surjective. And $\phi\alpha(a) = (a,\beta\sigma\alpha(a)) = (a,0)$ and $\pi_{M''}\phi(m) = \beta(\sigma(m)) = \beta(m)$ and we are done.\par
    
% ------------------------------------------------------------------------------
% Reference and Cited Works
% ------------------------------------------------------------------------------

\bibliographystyle{IEEEtran}
\bibliography{References.bib}

% ------------------------------------------------------------------------------

\end{document}
