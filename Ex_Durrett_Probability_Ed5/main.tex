%!TEX program = xelatex
\documentclass[lang=en,11pt,a4paper,citestyle =authoryear]{elegantpaper}

% 标题
\title{Solutions for Exercises in Durrett Edition 5}
\author{Boren(Wells) Guan}

% 本文档命令
\usepackage{array,url,stix}
\usepackage{subfigure}
\newcommand{\ccr}[1]{\makecell{{\color{#1}\rule{1cm}{1cm}}}}
\newcommand{\code}[1]{\lstinline{#1}}
\newcommand{\prvd}{$\hfill \qedsymbol$}
\newcommand{\Z}{\mathbb{Z}}
\newcommand{\R}{\mathbb{R}}
\newcommand{\N}{\mathbb{N}}
\newcommand{\C}{\mathbb{C}}
\newcommand{\Q}{\mathbb{Q}}
\newcommand{\M}{\mathcal{M}}
\newcommand{\B}{\mathcal{B}}
\newcommand{\X}{\mathcal{X}}

% 文档区
\begin{document}

% 标题
\maketitle

\subsection*{Section 3.2 Ex.6} 
\textbf{The Levy Metric} Show that
\[\rho(F,G) = \inf \{\epsilon: F(x-\epsilon)-\epsilon \leq G(x) \leq F(x+\epsilon)+\epsilon \text{ for all }x\}\]
defines a metric on the space of distributions and $\rho(F_n,F) \to 0$ if and only if $F_n \Rightarrow F$.
\vspace{0.5em}\\
\textbf{Sol.} \par
Firstly, consider $\rho(F,G) = 0$ implies that $G(x) \in [F(x-n^{-1})-n^{-1},F(x+n^{-1})+n^{-1}]$ for all $n\in\N$, and hence $G(x)=F(x)$ at all continuities $x$ of $F$, which means $G=F$ a.e. on $\R$. Then for any $\epsilon > \rho(F,G)$, then we know
\[F(x-\epsilon)-\epsilon \leq G(x) \leq F(x+\epsilon)+\epsilon\]
for all $x\in\R$, which means
\[\begin{aligned}
    G(x+\epsilon) + \epsilon &\geq F(x) \\
    G(x-\epsilon) - \epsilon &\leq F(x) \\
\end{aligned}\]
and hence $\epsilon \geq \rho(G,F)$, so $\rho(F,G) \geq \rho(G,F)$ and hence $\rho(G,F) = \rho(F,G)$ by symmetry. Now check the triangular inequality, consider any $\epsilon > \rho(F,G)$ and $\delta > \rho(G,H)$, then
\[
\begin{aligned}
    F(x) &\leq G(x+\epsilon) + \epsilon \leq H(x+\epsilon+\delta) +(\epsilon+\delta) \\
    F(x) &\geq G(x-\epsilon) - \epsilon \geq H(x-\epsilon-\delta) - (\epsilon+\delta) \\
\end{aligned}
\]
and hence $\rho(F,H)\leq\rho(F,G)+\rho(G,H)$.\par
Then if $F_n \Rightarrow F$, then we know there exists $\epsilon_n > 0$ for any $n\in\N$ and $\epsilon_n \to 0, n\to\infty$ such that
\[
\begin{aligned}
    F_n(x) &\geq F(x-\epsilon_n)-\epsilon_n \\
    F_n(x) &\leq F(x+\epsilon_n)+\epsilon_n \\
\end{aligned}
\]
for all $x\in \R$, then
\[
\begin{aligned}
    \liminf F_n(x) &\geq \liminf F(x-\epsilon_n) \\
    \limsup F_n(x) &\leq \limsup F(x+\epsilon_n)
\end{aligned}
\]
and hence $ \lim F_n(x) = F(x)$ at $x$ the continuity of $F$, which means $F_n \Rightarrow F$.\par
Then if $F_n \Rightarrow F$ we only should prove for any $\epsilon > 0$, there exists an integer $N$ such $\rho(F_n,F)< \epsilon$ for all $n\geq N$. Assume the conclusion is false, then there exists a subsequence of $\{F_n\}_1^{\infty}$ and $\{x_n\}_1^{\infty}\subset \R$ such that
\[F_{n_k}(x_k)\notin [F(x_k-\epsilon)-\epsilon,F(x_k+\epsilon)+\epsilon]\]
where we can assume $F_{n_k}(x_k) > F(x_k+\epsilon)+\epsilon$ WLOG and if $\{x_n\}$ bounded, we know there exists a subsequence $x_{m_k} \to x$ for some $x\in\R$, then we know there exists an integer $M$ such that $x_{m_k}<x+\delta$ for any $k\geq M$ for some $\delta>0$ where $\delta\leq \epsilon$ and $x+\delta$ is a continuity of $F$. Then we know
\[
F_{n_{m_k}}(x_{m_k}) \leq F_{n_{m_k}}(x+\delta) \to F(x+\delta) < F(x+\epsilon)
\]
and
\[
F_{n_{m_k}}(x_{m_k}) > F(x_{m_k}+\epsilon)+\epsilon \to F(x+\epsilon)+\epsilon
\]
for $k\geq M$ which is a contradiction. If $x_n$ not bounded, then assume $x_n\to \infty$ and if $F_{n}(x_n) > F(x_n+\epsilon)+\epsilon$ WLOG, we know $\limsup F_n(x_n) > 1+\epsilon$ which is a contradiction, so we assume $F_{n}(x_n) < F(x_n-\epsilon)-\epsilon\leq 1-\epsilon$. Notice we know $F_n$ has to be tight by theorem 3.2.13, then we know there is an $M>0$ such 
\[ \limsup(1-F_n(x_n)) \leq \limsup(1-F_n(M)+F_n(-M)) < \epsilon\]
which is a contradiction. For $x_n\to-\infty$, the proof is similar.
\prvd
\vspace{0.5em}

\subsection*{Section 3.2 Ex.7} 
\textbf{The Ky Fan Metric} on random variables is defined by
\[\alpha(X,Y) =\inf\{\epsilon \geq 0: P(|X-Y|>\epsilon)\leq \epsilon\}\]
Show that if $\alpha(X,Y)= \alpha$ then the corresponding distributions have Levy distance $\rho(F,G)\leq \alpha$.
\vspace{0.5em}\\
\textbf{Sol.} \par
Except for the required conclusion, we show $\alpha$ is a metric at first. Notice if $\alpha(X,Y) = 0$, we know for any $\epsilon > 0$,
\[ P (|X-Y|>\epsilon) \leq P(|X-Y|>\epsilon/n) \leq \epsilon.n\]
for any $n\in\N$ and hence $X=Y$ a.s. The symmetry is trivial, then we only need to show the triangular inequality. Consider $\epsilon>\alpha(X,Y)$ and $\delta>\alpha(Y,Z)$, then we know
\[P(|X-Z|>\epsilon+\delta) \leq P(|X-Y|>\epsilon)+P(|Y-Z|>\delta) \leq \epsilon + \delta\]
and hence $\alpha(X,Z)\leq \alpha(X,Y)+\alpha(Y,Z)$.\par
If $\alpha(X,Y)=\alpha$, then for any $\epsilon > \alpha$, we know
\[P(|X-Y|>\epsilon) \leq \epsilon\]
and hence
\[
\begin{aligned}
F(x) &\leq P(X\leq x, Y>x+\epsilon) + G(x+\epsilon) \leq \epsilon + G(x+\epsilon) \\
1- F(x) &\leq P(X>x, Y\leq x-\epsilon) + (1-G(x-\epsilon)) \leq \epsilon + 1 - G(x-\epsilon) \\ 
\end{aligned}
\]
for any $x\in\R$, and hence $\rho(F,G) \leq \alpha$.
\prvd
\vspace{0.5em}

\subsection*{Section 3.2 Ex.8} 
Let $\alpha(X,Y)$ be the Ky Fan metric and $\beta(X,Y)=E(|X-Y|/(1+|X-Y|))$. If $\alpha(X,Y)=a$, then
\[
a^2/(1+a)\leq \beta(X,Y)\leq a+(1-a)a/(1+a)
\]
\vspace{0.5em}\\
\textbf{Sol.} \par
We know $\alpha(X,Y)$ implies that $\rho(F,G)\leq \alpha$ where $\rho$ is the Levy metric and $F,G$ the distribution of $X,Y$, then we know for any $\epsilon>0$
\[
\begin{aligned}
\beta(X,Y) = E(|X-Y|/(1+|X-Y|)) &\geq \epsilon P(|X-Y|>\epsilon)/(1+\epsilon) \\
E(|X-Y|/(1+|X-Y|)) &\leq \epsilon P(|X-Y|\leq \epsilon)/(1+\epsilon) + P(|X-Y| > \epsilon)
\end{aligned}
\]
we know there exists $a_n\uparrow a$ such that $P(|X-Y|>a_n)\geq a_n$, then we know
\[\beta(X,Y)\geq a_nP(|X-Y|>a_n)/(1+a_n) \geq a_n^2/(1+a_n)\]
for any $n\in\N$ and hence $\beta(X,Y)\geq a^2/(1+a)$.\par
Consider $\epsilon > a$, then we have
\[
\beta(X,Y) \leq \epsilon/(1+\epsilon) + 1/(1+\epsilon)P(|X-Y|>\epsilon) \leq 2\epsilon/(1+\epsilon) 
\] 
and hence $\beta(X,Y) \leq 2a/(1+a) = a +(1-a)a/(1+a)$.
\prvd
\vspace{0.5em}

\subsection*{Section 3.2 Ex.9} 
If $F_n\Rightarrow F$ and $F$ is continuous then $\sup_x|F_n(x)-F(X)|\to 0$.
\vspace{0.5em}\\
\textbf{Sol.} \par
We consider the Levy metric and we know $\rho(F_n,F)\to 0$, and it is not hard to see we only need to show $F$ is uniformly continuous on $\R$. If not, consider there is $\epsilon>0$ and $\{s_n,t_n\}_1^{\infty}$ such that $|F(s_n,t_n) > \epsilon|$. If $s_n$ is bounded, consider one of its convergent subsequence and assume $s_{n_k},t_{n_k} \to x \in \R$, then by $F$ is continuous at $x$ we may know for $k$ large enough, $|F(s_{n_k})-F(t_{n_k})|<\epsilon$ which is a contradiction. Then assume $s_n$ is not bounded, assume there is a subsequence converging to $\infty$, i.e. $s_{n_k},t_{n_k} \to \infty$, then we only need to find $M>0$ such that $F(M)>1-\epsilon/3$ will induce a contradiction; the proof is similar when $s_{n_k}\to -\infty$.
\prvd
\vspace{0.5em}

\addappheadtotoc

\end{document}
